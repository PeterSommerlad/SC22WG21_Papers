\documentclass[ebook,11pt,article]{memoir}
\usepackage{geometry}  % See geometry.pdf to learn the layout options. There are lots.
\geometry{a4paper}  % ... or a4paper or a5paper or ... 
%\geometry{landscape}  % Activate for for rotated page geometry
%\usepackage[parfill]{parskip}  % Activate to begin paragraphs with an empty line rather than an indent


\usepackage[final]
           {listings}     % code listings
\usepackage{color}        % define colors for strikeouts and underlines
\usepackage{underscore}   % remove special status of '_' in ordinary text
\usepackage{xspace}
\pagestyle{myheadings}

\newcommand{\papernumber}{p0408r0}
\newcommand{\paperdate}{2016-07-01}

\markboth{\papernumber{} \paperdate{}}{\papernumber{} \paperdate{}}

\title{\papernumber{} - Efficient Access to basic\_stringbuf's Buffer}
\author{Peter Sommerlad}
\date{\paperdate}                        % Activate to display a given date or no date
\input{macros}
\setsecnumdepth{subsection}

\begin{document}
\maketitle
\begin{tabular}[t]{|l|l|}\hline 
Document Number: \papernumber & \\\hline
Date: & \paperdate \\\hline
Project: & Programming Language C++\\\hline 
Audience: & LWG/LEWG\\\hline
\end{tabular}

\chapter{Motivation}
Streams have been the oldest part of the C++ standard library and their specification doesn't take into account many things introduced since C++11. One of  the oversights is that there is no non-copying access to the internal buffer of a \tcode{basic_stringbuf} which makes at least the obtaining of the output results from an \tcode{ostringstream} inefficient, because a copy is always made. I personally speculate that this was also the reason why \tcode{basic_strbuf} took so long to get deprecated with its \tcode{char *} access.

With move semantics and \tcode{basic_string_view} there is no longer a reason to keep this pessimissation alive on \tcode{basic_stringbuf}.


\chapter{Introduction}
This paper proposes to adjust the API of \tcode{basic_stringbuf} and the corresponding stream class templates to allow accessing the underlying string more efficiently.

C++17 and library TS have \tcode{basic_string_view} allowing an efficient read-only access to a contiguous sequence of characters which I believe \tcode{basic_stringbuf} has to guarantee about its internal buffer, even if it is not implemented using \tcode{basic_string} obtaining a \tcode{basic_string_view} on the internal buffer should work sidestepping the copy overhead of calling \tcode{str()}. 

On the other hand, there is no means to construct a \tcode{basic_string} and move from it into a \tcode{basic_stringbuf} via a constructor or a move-enabled overload of \tcode{str(basic_string \&\&)}.

\chapter{Acknowledgements}
\begin{itemize}
\item Daniel Kr\"ugler encouraged me to pursue this track.
\end{itemize}


%\chapter{example code}


\chapter{Impact on the Standard}
This is an extension to the API of \tcode{basic_stringbuf}, \tcode{basic_stringstream}, \tcode{basic_istringstream}, and \tcode{basic_ostringstream} class templates.
\chapter{Design Decisions}
After experimentation I decided that substituting the \tcode{(basic_string<charT,traits,Allocator const \&)} constructors in favor of passing a \tcode{basic_string_view} would lead to ambiguities with the new move-from-string constructors.
\section{Open Issues to be Discussed by LEWG / LWG}
\begin{itemize}
\item Is the name of the \tcode{str_view()} member function ok?
\item Should the \tcode{str()\&\&} overload be provided for move-out?
\item Should \tcode{str()\&\&} empty the character sequence or leave it in an unspecified but valid state?
\end{itemize}

\chapter{Technical Specifications}
The following is relative to n4594.
\section{27.8.2 Adjust synopsis of basic\_stringbuf [stringbuf]}
Add a new constructor overload:
\begin{codeblock}
      explicit basic_stringbuf(
        basic_string<charT, traits, Allocator>&& s,
        ios_base::openmode which = ios_base::in | ios_base::out);
\end{codeblock}

Change the \tcode{const}-overload of \tcode{str()} member function to add a reference qualification. This avoids ambiguities with the rvalue-ref overload of \tcode{str()}.  
\begin{codeblock}
basic_string<charT,traits,Allocator> str() const &;
\end{codeblock}


Add two overloads of the \tcode{str()} member function and add the \tcode{str_view()} member function:
\begin{codeblock}
void str(basic_string<charT, traits, Allocator>&& s);
basic_string<charT,traits,Allocator> str() &&;
basic_string_view<charT, traits> str_view() const;
\end{codeblock}


\subsection{27.8.2.1 basic\_stringbuf constructors [stringbuf.cons]}
Add the following constructor specification:
\begin{itemdecl}
      explicit basic_stringbuf(
        basic_string<charT, traits, Allocator>&& s,
        ios_base::openmode which = ios_base::in | ios_base::out);
\end{itemdecl}
\begin{itemdescr}
\pnum
\effects Constructs an object of class \tcode{basic_stringbuf}, initializing the base class with \tcode{basic_streambuf()} (27.6.3.1), and initializing mode with which. Then calls \tcode{str(std::move(s))}.
\end{itemdescr}

\subsection{27.8.2.3 Member functions [stringbuf.members]}
Change the \tcode{const}-overload of \tcode{str()} member function specification to add a reference qualification. This avoids ambiguities with the rvalue-ref overload of \tcode{str()}.  
\begin{codeblock}
basic_string<charT,traits,Allocator> str() const &;
\end{codeblock}

Add the following specifications and adjust the wording of \tcode{str() const \&} according to the wording given for \tcode{str_view() const} member function.:
\begin{itemdecl}
void str(basic_string<charT, traits, Allocator>&& s);
\end{itemdecl}
\begin{itemdescr}
\pnum
\effects 
Moves the content of \tcode{s} into the \tcode{basic_stringbuf} underlying character sequence and initializes the input and output sequences according to \tcode{mode}.
%% mode is an exposition-only member of basic_strinbuf

\pnum
\postconditions
Let \tcode{size} denote the original value of \tcode{s.size()} before the move.
If \tcode{mode \& ios_base::out} is true, \tcode{pbase()} points to the first underlying character and \tcode{epptr() >= pbase() + size} holds; in addition, if \tcode{mode \& ios_base::ate} is true, \tcode{pptr() == pbase() + size} holds, otherwise \tcode{pptr() == pbase()} is true. If \tcode{mode \& ios_base::in} is true, \tcode{eback()} points to the first underlying character, and both \tcode{gptr() == eback()} and \tcode{egptr() == eback() + size} hold.
\end{itemdescr}

\begin{itemdecl}
basic_string<charT, traits, Allocator> str() &&;
\end{itemdecl}
\begin{itemdescr}

\pnum
\returns A \tcode{basic_string} object moved from the \tcode{basic_stringbuf} underlying character sequence. If the \tcode{basic_stringbuf} was created only in input mode, \tcode{basic_string(eback(), egptr()-eback())}. If the \tcode{basic_stringbuf} was created with \tcode{which \& ios_base::out} being true then \tcode{basic_string(pbase(), high_mark-pbase())}, where \tcode{high_mark} represents the position one past the highest initialized character in the buffer. Characters can be initialized by writing to the stream, by constructing the \tcode{basic_stringbuf} with a \tcode{basic_string}, or by calling one of the \tcode{str(basic_string)} member functions. In the case of calling one of the \tcode{str(basic_string)} member functions, all characters initialized prior to the call are now considered uninitialized (except for those characters re-initialized by the new \tcode{basic_string}). Otherwise the \tcode{basic_stringbuf} has been created in neither input nor output mode an empty \tcode{basic_string} is returned. 

\pnum
\postcondition The underlying character sequence is empty.
%% or " in an unspecified but valid state."
%% this might not be the right way to say that
\end{itemdescr}

\begin{itemdecl}
basic_string_view<charT, traits> str_view() const;
\end{itemdecl}
\begin{itemdescr}
\pnum
\returns A \tcode{basic_string_view} object referring to the \tcode{basic_stringbuf} underlying character sequence. If the \tcode{basic_stringbuf} was created only in input mode,  \tcode{basic_string_view(eback(), egptr()-eback())}. If the \tcode{basic_stringbuf} was created with \tcode{which \& ios_base::out} being true then \tcode{basic_string_view(pbase(), high_mark-pbase())}, where \tcode{high_mark} represents the position one past the highest initialized character in the buffer. Characters can be initialized by writing to the stream, by constructing the \tcode{basic_stringbuf} with a \tcode{basic_string}, or by calling one of the \tcode{str(basic_string)} member functions. In the case of calling one of the \tcode{str(basic_string)} member functions, all characters initialized prior to the call are now considered uninitialized (except for those characters re-initialized by the new \tcode{basic_string}). Otherwise the \tcode{basic_stringbuf} has been created in neither input nor output mode a \tcode{basic_string_view} referring to an empty range is returned. 

\pnum
\enternote
Using the returned object after destruction or any modification of \tcode{*this}, such as output on the holding stream, will cause undefined behavior, because the internal string referred by the return value might have changed or re-allocated. 
\exitnote
\end{itemdescr}
%% istream
\section{27.8.3 Adjust synopsis of basic\_istringstream [istringstream]}
Add a new constructor overload:
\begin{codeblock}
           explicit basic_istringstream(
             basic_string<charT, traits, Allocator>&& str,
             ios_base::openmode which = ios_base::in);
\end{codeblock}

Change the \tcode{const}-overload of \tcode{str()} member function to add a reference qualification. This avoids ambiguities with the rvalue-ref overload of \tcode{str()}.  
\begin{codeblock}
basic_string<charT,traits,Allocator> str() const &;
\end{codeblock}


Add an overload of the \tcode{str()} member function and add the \tcode{str_view()} member function:
\begin{codeblock}
void str(basic_string<charT, traits, Allocator>&& s);
basic_string<charT,traits,Allocator> str() &&;
basic_string_view<charT, traits> str_view() const;
\end{codeblock}

\subsection{27.8.3.1 basic\_istringstream constructors [istringstream.cons]}
Add the following constructor specification:
\begin{itemdecl}
explicit basic_istringstream(
  const basic_string<charT, traits, Allocator>&& str,
  ios_base::openmode which = ios_base::in);
\end{itemdecl}
\begin{itemdescr}
\pnum
\effects Constructs an object of class \tcode{basic_istringstream<charT, traits>}, initializing the base class with \tcode{basic_istream(\&sb)} and initializing \tcode{sb} with \tcode{basic_stringbuf<charT, traits, Allocator>(std::move(str), which | ios_base::in))} (27.8.2.1).
\end{itemdescr}
\subsection{27.8.3.3 Member functions [istringstream.members]}
Change the \tcode{const}-overload of \tcode{str()} member function specification to add a reference qualification. This avoids ambiguities with the rvalue-ref overload of \tcode{str()}.  
\begin{codeblock}
basic_string<charT,traits,Allocator> str() const &;
\end{codeblock}

Add the following specifications and adjust the wording of \tcode{str() const} according to the wording given for \tcode{str_view() const} member function.:
\begin{itemdecl}
void str(basic_string<charT, traits, Allocator>&& s);
\end{itemdecl}
\begin{itemdescr}
\pnum
\effects \tcode{rdbuf()->str(std::move(s))}.
\end{itemdescr}
\begin{itemdecl}
basic_string<charT,traits,Allocator> str() &&;
\end{itemdecl}
\begin{itemdescr}
\pnum
\returns \tcode{std::move(*rdbuf()).str()}.
\end{itemdescr}
\begin{itemdecl}
basic_string_view<charT, traits> str_view() const;
\end{itemdecl}
\begin{itemdescr}
\pnum
\returns \tcode{rdbuf()->str_view()}.
\end{itemdescr}

%%ostream

\section{27.8.4 Adjust synopsis of basic\_ostringstream [ostringstream]}
Add a new constructor overload:
\begin{codeblock}
           explicit basic_ostringstream(
             basic_string<charT, traits, Allocator>&& str,
             ios_base::openmode which = ios_base::in);
\end{codeblock}

Change the \tcode{const}-overload of \tcode{str()} member function to add a reference qualification. This avoids ambiguities with the rvalue-ref overload of \tcode{str()}.  
\begin{codeblock}
basic_string<charT,traits,Allocator> str() const &;
\end{codeblock}


Add an overload of the \tcode{str()} member function and add the \tcode{str_view()} member function:
\begin{codeblock}
void str(basic_string<charT, traits, Allocator>&& s);
basic_string<charT,traits,Allocator> str() &&;
basic_string_view<charT, traits> str_view() const;
\end{codeblock}

\subsection{27.8.4.1 basic\_ostringstream constructors [ostringstream.cons]}
Add the following constructor specification:
\begin{itemdecl}
explicit basic_ostringstream(
  const basic_string<charT, traits, Allocator>&& str,
  ios_base::openmode which = ios_base::in);
\end{itemdecl}
\begin{itemdescr}
\pnum
\effects Constructs an object of class \tcode{basic_ostringstream<charT, traits>}, initializing the base class with \tcode{basic_ostream(\&sb)} and initializing \tcode{sb} with \tcode{basic_stringbuf<charT, traits, Allocator>(std::move(str), which | ios_base::in))} (27.8.2.1).
\end{itemdescr}
\subsection{27.8.4.3 Member functions [ostringstream.members]}
Change the \tcode{const}-overload of \tcode{str()} member function specification to add a reference qualification. This avoids ambiguities with the rvalue-ref overload of \tcode{str()}.  
\begin{codeblock}
basic_string<charT,traits,Allocator> str() const &;
\end{codeblock}

Add the following specifications and adjust the wording of \tcode{str() const} according to the wording given for \tcode{str_view() const} member function.:
\begin{itemdecl}
void str(basic_string<charT, traits, Allocator>&& s);
\end{itemdecl}
\begin{itemdescr}
\pnum
\effects \tcode{rdbuf()->str(std::move(s))}.
\end{itemdescr}
\begin{itemdecl}
basic_string<charT,traits,Allocator> str() &&;
\end{itemdecl}
\begin{itemdescr}
\pnum
\returns \tcode{std::move(*rdbuf()).str()}.
\end{itemdescr}
\begin{itemdecl}
basic_string_view<charT, traits> str_view() const;
\end{itemdecl}
\begin{itemdescr}
\pnum
\returns \tcode{rdbuf()->str_view()}.
\end{itemdescr}

%%stringstream
\section{27.8.5 Adjust synopsis of basic\_stringstream [stringstream]}
Add a new constructor overload:
\begin{codeblock}
           explicit basic_stringstream(
             basic_string<charT, traits, Allocator>&& str,
             ios_base::openmode which = ios_base::in);
\end{codeblock}

Change the \tcode{const}-overload of \tcode{str()} member function to add a reference qualification. This avoids ambiguities with the rvalue-ref overload of \tcode{str()}.  
\begin{codeblock}
basic_string<charT,traits,Allocator> str() const &;
\end{codeblock}

Add an overload of the \tcode{str()} member function and add the \tcode{str_view()} member function:
\begin{codeblock}
void str(basic_string<charT, traits, Allocator>&& s);
basic_string<charT,traits,Allocator> str() &&;
basic_string_view<charT, traits> str_view() const;
\end{codeblock}

\subsection{27.8.4.1 basic\_stringstream constructors [stringstream.cons]}
Add the following constructor specification:
\begin{itemdecl}
explicit basic_stringstream(
  const basic_string<charT, traits, Allocator>&& str,
  ios_base::openmode which = ios_base::in);
\end{itemdecl}
\begin{itemdescr}
\pnum
\effects Constructs an object of class \tcode{basic_stringstream<charT, traits>}, initializing the base class with \tcode{basic_stream(\&sb)} and initializing \tcode{sb} with \tcode{basic_stringbuf<charT, traits, Allocator>(std::move(str), which | ios_base::in))} (27.8.2.1).
\end{itemdescr}
\subsection{27.8.4.3 Member functions [stringstream.members]}
Change the \tcode{const}-overload of \tcode{str()} member function specification to add a reference qualification. This avoids ambiguities with the rvalue-ref overload of \tcode{str()}.  
\begin{codeblock}
basic_string<charT,traits,Allocator> str() const &;
\end{codeblock}

Add the following specifications and adjust the wording of \tcode{str() const} according to the wording given for \tcode{str_view() const} member function.:
\begin{itemdecl}
void str(basic_string<charT, traits, Allocator>&& s);
\end{itemdecl}
\begin{itemdescr}
\pnum
\effects \tcode{rdbuf()->str(std::move(s))}.
\end{itemdescr}
\begin{itemdecl}
basic_string<charT,traits,Allocator> str() &&;
\end{itemdecl}
\begin{itemdescr}
\pnum
\returns \tcode{std::move(*rdbuf()).str()}.
\end{itemdescr}
\begin{itemdecl}
basic_string_view<charT, traits> str_view() const;
\end{itemdecl}
\begin{itemdescr}
\pnum
\returns \tcode{rdbuf()->str_view()}.
\end{itemdescr}



\chapter{Appendix: Example Implementations}

The given specification has been implemented within a recent version of the sstream header of gcc6. Here are some definitions taken from there:
\begin{codeblock}
// basic_stringbuf:
      explicit
      basic_stringbuf(__string_type&& __str,
		      ios_base::openmode __mode = ios_base::in | ios_base::out)
      : __streambuf_type(), _M_mode(), _M_string(std::move(__str))
      { _M_stringbuf_init(__mode); }
  	using __string_view_type=experimental::basic_string_view<_CharT,_Traits>;

      __string_view_type str_view() const {
      	__string_view_type __ret{};
      	if ( this->pptr())  {
      	    // The current egptr() may not be the actual string end.
      	    if (this->pptr() > this->egptr())
      	      __ret = __string_view_type(this->pbase(), this->pptr()-this->pbase());
      	    else
       	      __ret = __string_view_type(this->pbase(), this->egptr()-this->pbase());
      	  }
      	else {
      		__ret = _M_string;
      	}
      	return __ret;
      }
     void
      str(__string_type&& __s)
      {
	_M_string.assign(std::move(__s));
	_M_stringbuf_init(_M_mode);
      }

//basic_istringstream
      explicit
      basic_istringstream(__string_type&& __str,
			  ios_base::openmode __mode = ios_base::in)
      : __istream_type(), _M_stringbuf(std::move(__str), __mode | ios_base::in)
      { this->init(&_M_stringbuf); }
  	using __string_view_type=experimental::basic_string_view<_CharT,_Traits>;
      __string_view_type
      str_view() const
      { return _M_stringbuf.str_view(); }
      void
      str( __string_type&& __s)
      { _M_stringbuf.str(std::move(__s)); }

//basic_ostringstream
      explicit
      basic_ostringstream(__string_type&& __str,
			  ios_base::openmode __mode = ios_base::out)
      : __ostream_type(), _M_stringbuf(std::move(__str), __mode | ios_base::out)
      { this->init(&_M_stringbuf); }
  	using __string_view_type=experimental::basic_string_view<_CharT,_Traits>;
      __string_view_type
      str_view() const
      { return _M_stringbuf.str_view(); }
      void
      str( __string_type&& __s)
      { _M_stringbuf.str(std::move(__s)); }

//basic_stringstream
      explicit
      basic_stringstream( __string_type&& __str,
			 ios_base::openmode __m = ios_base::out | ios_base::in)
      : __iostream_type(), _M_stringbuf(std::move(__str), __m)
      { this->init(&_M_stringbuf); }
  	using __string_view_type=experimental::basic_string_view<_CharT,_Traits>;
      __string_view_type
      str_view() const
      { return _M_stringbuf.str_view(); }
      void
      str( __string_type&& __s)
      { _M_stringbuf.str(std::move(__s)); }

\end{codeblock}


\end{document}

