\documentclass[ebook,11pt,article]{memoir}
\usepackage{geometry}  % See geometry.pdf to learn the layout options. There are lots.
\geometry{a4paper}  % ... or a4paper or a5paper or ... 
%\geometry{landscape}  % Activate for for rotated page geometry
%%% from std.tex
%\usepackage[american]
%           {babel}        % needed for iso dates
%\usepackage[iso,american]
%           {isodate}      % use iso format for dates
\usepackage[final]
           {listings}     % code listings
%\usepackage{longtable}    % auto-breaking tables
%\usepackage{ltcaption}    % fix captions for long tables
\usepackage{relsize}      % provide relative font size changes
%\usepackage{textcomp}     % provide \text{l,r}angle
\usepackage{underscore}   % remove special status of '_' in ordinary text
%\usepackage{parskip}      % handle non-indented paragraphs "properly"
%\usepackage{array}        % new column definitions for tables
\usepackage[normalem]{ulem}
\usepackage{enumitem}
\usepackage{color}        % define colors for strikeouts and underlines
%\usepackage{amsmath}      % additional math symbols
%\usepackage{mathrsfs}     % mathscr font
\usepackage[final]{microtype}
%\usepackage{multicol}
\usepackage{xspace}
%\usepackage{lmodern}
\usepackage[T1]{fontenc} % makes tilde work! and is better for umlauts etc.
%\usepackage[pdftex, final]{graphicx}
\usepackage[pdftex,
%            pdftitle={C++ International Standard},
%            pdfsubject={C++ International Standard},
%            pdfcreator={Richard Smith},
            bookmarks=true,
            bookmarksnumbered=true,
            pdfpagelabels=true,
            pdfpagemode=UseOutlines,
            pdfstartview=FitH,
            linktocpage=true,
            colorlinks=true,
            linkcolor=blue,
            plainpages=false
           ]{hyperref}
%\usepackage{memhfixc}     % fix interactions between hyperref and memoir
%\usepackage[active,header=false,handles=false,copydocumentclass=false,generate=std-gram.ext,extract-cmdline={gramSec},extract-env={bnftab,simplebnf,bnf,bnfkeywordtab}]{extract} % Grammar extraction
%
\renewcommand\RSsmallest{5.5pt}  % smallest font size for relsize


%%%% reuse all three from std.tex:
\input{macros}
\input{layout}
\input{styles}

\pagestyle{myheadings}

\newcommand{\papernumber}{p0573r0}
\newcommand{\paperdate}{2017-07-15}

\markboth{\papernumber{} \paperdate{}}{\papernumber{} \paperdate{}}

\title{\papernumber{} - Manipulators for C++ Synchronized Buffered Ostream (see p0053)}
\author{Peter Sommerlad, Pablo Halpern}
\date{\paperdate}                % Activate to display a given date or no date
\setsecnumdepth{subsection}

\begin{document}
\maketitle
\begin{center}
\begin{tabular}[t]{|l|l|}\hline 
Document Number:&  \papernumber \\\hline
Date: & \paperdate \\\hline
Project: & Programming Language C++\\\hline 
Audience: & LEWG (LWG to re-check the wording)\\\hline
\end{tabular}
\end{center}
\chapter{Introduction}
After Kona, Pablo asked me to add \tcode{ostream} manipulators for \tcode{basic_osyncstream} to allow users of such streams to modify their flushing behavior, when those stream objects are only know via \tcode{ostream\&} down the call chain.

The wording for these manipulators was reviewed by LWG in Toronto (p0053r5), but their names were never discussed in LEWG, therefore I followed Jeffrey's suggestion to split them from p0053r6. For more information see that paper.


\section{Items to be discussed by LEWG}
\begin{itemize}
\item Naming of the manipulators
\item Should the manipulators be in header \tcode{<osyncstream>} instead of globally available in \tcode{<ostream>} as are \tcode{flush} and \tcode{endl}? Putting them in \tcode{<osyncstream>} (only), will increase dependence on \tcode{basic_osyncstream}, where \tcode{basic_syncbuf} would suffice for inline implementation of the manipulators. That dependency could even be mitigated by non-inline implementations of the manipulators (providing their instantiations for the supported character types as is done with many other things in the iostream implementaions).
\item re-check wording (done be LWG in Toronto, but minor adaptations were made, because of LWG's feedback. Pablo is OK with the edits)
\item What should be the delivery vehicle for this feature: C++20 or the concurrency TS? I believe both should be addressed when moved, like with p0053.
\end{itemize}


\chapter{Wording}

This wording is relative to the current C++ working draft and refers to the specification in p0053r6. It could be integrated into a Concurrency TS  accordingly when p0053 gets adopted.

\section{30.7.5.4 Standard \tcode{basic_ostream} manipulators [ostream.manip]}
Add the following three manipulators.
\begin{addedblock}
\begin{itemdecl}
template <class charT, class traits>
  basic_ostream<charT, traits>& emit_on_flush(basic_ostream<charT, traits>& os);
\end{itemdecl}

\begin{itemdescr}
\pnum
\effects
If \tcode{os.rdbuf()} is a \tcode{basic_osyncbuf<charT, traits, Allocator>} pointer \tcode{buf}, calls \tcode{buf->set_emit_on_sync(true)}. Otherwise this manipulator has no effect. 
\begin{note}
To work around the issue that the \tcode{Allocator} template argument can not be deduced, implementations can introduce an intermediate base class to \tcode{basic_osyncbuf} that takes care its \tcode{emit_on_sync} flag.
\end{note}

\pnum
\returns
\tcode{os}.
\end{itemdescr}

\begin{itemdecl}
template <class charT, class traits>
  basic_ostream<charT, traits>& noemit_on_flush(basic_ostream<charT, traits>& os);
\end{itemdecl}

\begin{itemdescr}
\pnum
\effects
If \tcode{os.rdbuf()} is a \tcode{basic_osyncbuf<charT, traits, Allocator>} pointer \tcode{buf}, calls \tcode{buf->set_emit_on_sync(false)}. Otherwise this manipulator has no effect. 

\pnum
\returns
\tcode{os}.
\end{itemdescr}

\begin{itemdecl}
template <class charT, class traits>
  basic_ostream<charT, traits>& flush_emit(basic_ostream<charT, traits>& os);
\end{itemdecl}

\begin{itemdescr}
\pnum
\effects
\tcode{flush(os)}. Further
if \tcode{os.rdbuf()} is a \tcode{basic_osyncbuf<charT, traits, Allocator>} pointer \tcode{buf}, 
calls \tcode{buf->emit()}. 
 
\pnum
\returns
\tcode{os}.
\end{itemdescr}
\end{addedblock}

%%%%%%
\section{Implementation}
An example implementation is availabile on \url{https://github.com/PeterSommerlad/SC22WG21_Papers/tree/master/workspace/p0053_basic_osyncstreambuf}

\end{document}

