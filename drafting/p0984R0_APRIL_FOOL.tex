\documentclass[ebook,11pt,article]{memoir}
\usepackage{geometry}  % See geometry.pdf to learn the layout options. There are lots.
\geometry{a4paper}  % ... or a4paper or a5paper or ... 
%\geometry{landscape}  % Activate for for rotated page geometry
%\usepackage[parfill]{parskip}  % Activate to begin paragraphs with an empty line rather than an indent
\usepackage{hyperref}
%\usepackage[final]
%           {listings}     % code listings
%\usepackage{color}        % define colors for strikeouts and underlines
%\usepackage{underscore}   % remove special status of '_' in ordinary text
%\usepackage{xspace}
%\usepackage[normalem]{ulem}
\usepackage{enumitem}
%%%%% from std.tex:
\usepackage[final]
           {listings}     % code listings
\usepackage{longtable}    % auto-breaking tables
\usepackage{ltcaption}    % fix captions for long tables
\usepackage{relsize}      % provide relative font size changes
\usepackage{textcomp}     % provide \text{l,r}angle
\usepackage{underscore}   % remove special status of '_' in ordinary text
\usepackage{parskip}      % handle non-indented paragraphs "properly"
\usepackage{array}        % new column definitions for tables
\usepackage[normalem]{ulem}
\usepackage{color}        % define colors for strikeouts and underlines
\usepackage{amsmath}      % additional math symbols
\usepackage{mathrsfs}     % mathscr font
\usepackage{microtype}
\usepackage{multicol}
\usepackage{xspace}
\usepackage{lmodern}
\usepackage[T1]{fontenc}
\usepackage[pdftex, final]{graphicx}
\input{macros}
\input{styles}
\input{layout}
%%%%%
\pagestyle{myheadings}

\newcommand{\papernumber}{P0984R0}
\newcommand{\paperdate}{2018-04-01}
\newcommand{\atat}{\makeatletter{}@\makeatother}

%\definecolor{insertcolor}{rgb}{0,0.5,0.25}
%\newcommand{\del}[1]{\textcolor{red}{\sout{#1}}}
%\newcommand{\ins}[1]{\textcolor{insertcolor}{\underline{#1}}}
%
%\newenvironment{insrt}{\color{insertcolor}}{\color{black}}


\markboth{\papernumber{} \paperdate{}}{\papernumber{} \paperdate{}}

\title{\papernumber{} - All (*)()-Pointers Replaced by Ideal Lambdas\\
Functions Objects Obsoleted by Lambdas}
\author{Peter Sommerlad}
\date{\paperdate}                        % Activate to display a given date or no date
\setsecnumdepth{subsection}

\begin{document}
\maketitle
\begin{tabular}[t]{|l|l|}\hline 
Document Number:& \papernumber  \\\hline
Date: & \paperdate \\\hline
Project: & Programming Language C++\\\hline 
Audience: & EWG/LEWG\\\hline
Target: & C++20\\\hline
\end{tabular}

\begin{abstract}
Ban non-overloaded functions and \tcode{<functional>} function object class templates from the standard = simpler syntax
\end{abstract}

\chapter{Motivation}

%Lambdas Over Ordinary Functions

Recent discussions around abbreviated concept syntax not needing braces or parentheses (P0956) or angle brackets, as well as the ban to take function addresses of standard functions (P0551) and passing them to algorithms, and the ugliness of \tcode{std::multiplies<>{}} as an algorithm argument led me to propose this change to the standard.

In addition it addresses issues many programmers have with ADL and allows for safer code, because it is easier to employ those functions in functional code without the ugliness or uncertainty one faces when using a function name that might degenerate to a function pointer (or not in generic code accepting references).

As a third issue, I am addressing that the standard committee often failed to "eat their own dog food" when specifying the next revision of the C++ standard, leading to holes in the library or even the language.


\section{Inability to take the address of a function defined in the standard library}
Please note the following example taken from P0052R7. It shows how to sidestep taking the address of a standard library function (\tcode{\&::fclose}) by wrapping it in a (unnecessarily) generic lambda:
\begin{example}
The following example shows its use to avoid calling \tcode{fclose} when \tcode{fopen} fails
\begin{codeblock}
auto file = make_unique_resource_checked(
  ::fopen("potentially_nonexistent_file.txt", "r"), 
  nullptr, [](auto fptr){::fclose(fptr);});
\end{codeblock}
\end{example}

Almost all examples on stack overflow are wrong with the current wording, e.g. \url{https://stackoverflow.com/questions/26360916/using-custom-deleter-with-unique-ptr} that claims \tcode{unique_ptr<FILE, int(*)(FILE*)>(fopen("file.txt", "rt"), \&fclose);} is the way to employ \tcode{unique_ptr} for \tcode{FILE *}. 

\section{Ugly syntax when using standard function objects}

Teaching a functional approach has become a treat with the introduction of lambdas in C++11, especially the relaxed rules for lambda bodies and generic lambdas of more recent standard revisions cry out for simplifying the syntax. Just consider the burden and teachability of the accumulate call in multiply_vector versus the multiply_vector_new.

\begin{codeblock}
void multiply_vector(){
	std::vector v{1,2,3,4,5,6};
	auto res=accumulate(begin(v),end(v),1,std::multiplies<>{});
	ASSERT_EQUAL(720,res);
}
\end{codeblock}
\begin{codeblock}
void multiply_vector_new(){
	std::vector v{1,2,3,4,5,6};
	auto res=accumulate(begin(v),end(v),1,Times);
	ASSERT_EQUAL(720,res);
}
\end{codeblock}

In addition the standard library functional forgot to provide corresponding function object class templates for a whole bunch of operators. Also using C++17 fold expressions one could do even more by providing n-ary function objects for the corresponding binary operators. This will also add to the versatility of the corresponding function objects.


\chapter{April's Impact on the standard}

Core language change:
\section{8.4.5.1 Closure Types [expr.prim.lambda.closure]}

In [expr.prim.lambda.closure] change paragraph 7 as follows

\pnum
The closure type for a non-generic \grammarterm{lambda-expression} with no
\grammarterm{lambda-capture}
whose constraints (if any) are satisfied
\added{is called a \textit{Ideal Lambda}.
An Ideal Lambda }
has a conversion function to pointer to
function with \Cpp{} language linkage(10.5) %\iref{dcl.link} 
having
the same parameter and return types as the closure type's function call operator.
The conversion is to ``pointer to \tcode{noexcept} function''
if the function call operator
has a non-throwing exception specification.
The value returned by this conversion function
is the address of a function \tcode{F} that, when invoked,
has the same effect as invoking the closure type's function call operator.
\tcode{F} is a constexpr function
if the function call operator is a constexpr function.
For a generic lambda with no \grammarterm{lambda-capture}, the closure type has a
conversion function template to
pointer to function. The conversion function template has the same invented
template parameter list, and the pointer to function has the same
parameter types, as the function call operator template.  The return type of
the pointer to function shall behave as if it were a
\grammarterm{decltype-specifier} denoting the return type of the corresponding
function call operator template specialization.
\begin{addedblock}
An Ideal Lambda furthermore defines an overload for the unary \tcode{operator\&()} that returns the result of the said conversion to function pointer. 
\begin{note}
That operator overload guarantees that existing code bases that invalidly take the address of a standard library function continue to work as expected.
\end{note}
\end{addedblock}

\section{20.5.1.2 Headers [headers]}
Change paragraph 6 of section [headers] as follows:

Names that are defined as functions in C shall be defined as \removed{functions}\added{\tcode{constexpr inline auto} variables initialized from an Ideal Lambda} in the
\Cpp{} standard library\added{, unless the \Cpp{} standard defines overloads of said function. In that case the names defined as functions in C shall be defined as functions}. \footnote{This disallows the practice, allowed in C, of
providing a masking macro in addition to the function prototype. The only way to
achieve equivalent inline behavior in \Cpp{} is to provide a definition \removed{as an
extern inline function}\added{within the Ideal Lambda's body}.}

\section{20.5.2.3 Linkage [using.linkage]}
Change paragraph 2 as follows:

\removed{Whether a}\added{A} name from the C standard library declared with
external linkage has
%\indextext{linkage!external}%
%\indextext{header!C library}%
%\indextext{\idxcode{extern ""C""}}%
\removed{\tcode{extern "C"}
or}
%\indextext{\idxcode{extern ""C++""}}%
\tcode{extern "C++"}
linkage\removed{ is \impldef{linkage of names from C standard library}. It is recommended that an
implementation use
\tcode{extern "C++"}
linkage for this purpose.}\footnote{The only reliable way to declare an object or
function signature from the C standard library is by including the header that
declares it, notwithstanding the latitude granted in 7.1.4 of the C
Standard.}


\section{20.5.5.4 Non-member functions [global.functions]}

Change paragraph 1 of [global.functions] as follows

\pnum
\removed{It is unspecified whether any}
\added{All non-overloaded non-template }non-member
functions in the \Cpp{} standard library 
\added{shall be defined as \tcode{constexpr inline auto} variables initialized from an Ideal Lambda. 
For the purpose of this standard these variables are called \emph{FOOL} (Function ObsOleted by Lambda).}
\begin{addedblock}
\begin{note}
This mechanism allows many wrong programs that take the address of a 
standard library function to conform to this standard.
\end{note}
\end{addedblock}
\added{It is unspecified wether any overloaded or templated non-member functions }
are defined as
inline(10.1.6).%\iref{dcl.inline}.

\section{20.5.5.6 
Constexpr functions and constructors
[constexpr.functions]}

Extend paragraph 1 of [constexpr.functions] as follows:

This document explicitly requires that certain standard library functions are
\tcode{constexpr}(10.1.5).
%\iref{dcl.constexpr}. 
An implementation shall not declare
any standard library function signature as \tcode{constexpr} except for those where
it is explicitly required.
Within any header that provides any non-defining declarations of constexpr
functions or constructors an implementation shall provide corresponding definitions.
\added{A FOOL's ([global.functions]) call operator shall be defined as \tcode{constexpr} or not accordingly.}
%\iref{dcl.constexpr}


\chapter{Fool's Impact on the standard}

Note my plan is to deprecate all existing function object class templates and replace them by the generic version of a lambda which is not only the syntactical convenient thing, but also the only thing that should work{\textsuperscript{{\scriptsize TM}}. Names of the lambda variables are uppercased at the moment to avoid confusion, but that is LEWG's bike shedding's call. 

As a side effect, disregarding the space for the deprecated features section 23.14 becomes significantly shorter. One could also attempt to deprecate \tcode{bind}, but I leave that with LEWG and do not propose it right now.
 
\section{23.14.1 Header <functional> synopsis [functional.syn]}
Adjust the synopsis of [functional.syn] as follows. Move all removed definitions and declarations to appendix D.

\begin{codeblock}
namespace std {
  // \ref{func.invoke}, invoke
  template<class F, class... Args>
    invoke_result_t<F, Args...> invoke(F&& f, Args&&... args)
      noexcept(is_nothrow_invocable_v<F, Args...>);

  // \ref{refwrap}, \tcode{reference_wrapper}
  template<class T> class reference_wrapper;

  template<class T> reference_wrapper<T> ref(T&) noexcept;
  template<class T> reference_wrapper<const T> cref(const T&) noexcept;
  template<class T> void ref(const T&&) = delete;
  template<class T> void cref(const T&&) = delete;

  template<class T> reference_wrapper<T> ref(reference_wrapper<T>) noexcept;
  template<class T> reference_wrapper<const T> cref(reference_wrapper<T>) noexcept;
 
\end{codeblock}
\begin{addedblock}
\begin{codeblock}
  // unary operators
  constexpr inline auto Deref = @\seebelow@ ; // unary *
  constexpr inline auto Address = @\seebelow@ ; // unary \&  
  constexpr inline auto Negate = @\seebelow@ ; // unary -
  constexpr inline auto Posate = @\seebelow@ ; // unary +  
  constexpr inline auto Not = @\seebelow@ ; // ! not
  constexpr inline auto Cmpl = @\seebelow@ ; // \~ cmpl
  
 
  // left associative binary operators
  constexpr inline auto PtrMemb = @\seebelow@ ; // ->*
  constexpr inline auto RefMemb = @\seebelow@ ; // .*
  
  constexpr inline auto Plus = @\seebelow@ ; // +
  constexpr inline auto Minus = @\seebelow@ ; // -
  constexpr inline auto Times = @\seebelow@ ; // *
  constexpr inline auto Divide = @\seebelow@ ; // /
  constexpr inline auto Remainder = @\seebelow@ ; // \&
  
  constexpr inline auto Equal = @\seebelow@ ; // ==
  constexpr inline auto Not_eq = @\seebelow@ ; // !=
  constexpr inline auto Bigger = @\seebelow@ ; // >
  constexpr inline auto Smaller = @\seebelow@ ; // <
  constexpr inline auto Maybe_bigger = @\seebelow@ ; // >=
  constexpr inline auto Sometimes_smaller = @\seebelow@ ; // <=
  constexpr inline auto Spaceship = @\seebelow@ ; // <=>

  constexpr inline auto And = @\seebelow@ ; // \&\& and
  constexpr inline auto Or = @\seebelow@ ; // || or

  constexpr inline auto Bitand = @\seebelow@ ; // \& bit_and
  constexpr inline auto Bitor = @\seebelow@ ; // | bit_or
  constexpr inline auto Xor = @\seebelow@ ; // \^{} xor
  constexpr inline auto Lshift = @\seebelow@ ; // <<
  constexpr inline auto Rshift = @\seebelow@ ; // >>
  
  // right associative binary operators
  constexpr inline auto Assign = @\seebelow@ ; // =
  constexpr inline auto PlusAssign = @\seebelow@ ; // +=
  constexpr inline auto MinusAssign = @\seebelow@ ; // -=
  constexpr inline auto TimesAssign = @\seebelow@ ; // *=
  constexpr inline auto DivideAssign = @\seebelow@ ; // /=
  constexpr inline auto RemainderAssign = @\seebelow@ ; // \%=
  constexpr inline auto AndAssign = @\seebelow@ ; // \&\&=
  constexpr inline auto OrAssign = @\seebelow@ ; // ||=
  constexpr inline auto And_eq = @\seebelow@ ; // \&=
  constexpr inline auto Or_eq = @\seebelow@ ; // |=
  constexpr inline auto Xor_eq = @\seebelow@ ; // \^{}=
  constexpr inline auto LshiftAssign = @\seebelow@ ; // <{}<=
  constexpr inline auto RrhiftAssign = @\seebelow@ ; // >{}>=
  
  // ternary operator
  constexpr inline auto Wtf = @\seebelow@ ; // ?:

  
  
\end{codeblock}
\end{addedblock}
\begin{removedblock}
\begin{codeblock}
  // \ref{arithmetic.operations}, arithmetic operations
  template<class T = void> struct plus;
  template<class T = void> struct minus;
  template<class T = void> struct multiplies;
  template<class T = void> struct divides;
  template<class T = void> struct modulus;
  template<class T = void> struct negate;
  template<> struct plus<void>;
  template<> struct minus<void>;
  template<> struct multiplies<void>;
  template<> struct divides<void>;
  template<> struct modulus<void>;
  template<> struct negate<void>;
  
  // \ref{comparisons}, comparisons
  template<class T = void> struct equal_to;
  template<class T = void> struct not_equal_to;
  template<class T = void> struct greater;
  template<class T = void> struct less;
  template<class T = void> struct greater_equal;
  template<class T = void> struct less_equal;
  template<> struct equal_to<void>;
  template<> struct not_equal_to<void>;
  template<> struct greater<void>;
  template<> struct less<void>;
  template<> struct greater_equal<void>;
  template<> struct less_equal<void>;

  // \ref{logical.operations}, logical operations
  template<class T = void> struct logical_and;
  template<class T = void> struct logical_or;
  template<class T = void> struct logical_not;
  template<> struct logical_and<void>;
  template<> struct logical_or<void>;
  template<> struct logical_not<void>;

  // \ref{bitwise.operations}, bitwise operations
  template<class T = void> struct bit_and;
  template<class T = void> struct bit_or;
  template<class T = void> struct bit_xor;
  template<class T = void> struct bit_not;
  template<> struct bit_and<void>;
  template<> struct bit_or<void>;
  template<> struct bit_xor<void>;
  template<> struct bit_not<void>;
\end{codeblock}
\end{removedblock}
\begin{codeblock}

  // \ref{func.notfn}, function template \tcode{not_fn}
  template<class F> @\unspec@ not_fn(F&& f);

  // \ref{func.bind}, bind
  template<class T> struct is_bind_expression;
  template<class T> struct is_placeholder;

  template<class F, class... BoundArgs>
    @\unspec@ bind(F&&, BoundArgs&&...);
  template<class R, class F, class... BoundArgs>
    @\unspec@ bind(F&&, BoundArgs&&...);

  namespace placeholders {
    // \tcode{\placeholder{M}} is the \impldef{number of placeholders for bind expressions} number of placeholders
    @\seebelownc@ _1;
    @\seebelownc@ _2;
               .
               .
               .
    @\seebelownc@ _@\placeholdernc{M}@;
  }
 
  // \ref{func.memfn}, member function adaptors
  template<class R, class T>
    @\unspec@ mem_fn(R T::*) noexcept;

  // \ref{func.wrap}, polymorphic function wrappers
  class bad_function_call;

  template<class> class function; // not defined
  template<class R, class... ArgTypes> class function<R(ArgTypes...)>;

  template<class R, class... ArgTypes>
    void swap(function<R(ArgTypes...)>&, function<R(ArgTypes...)>&) noexcept;

  template<class R, class... ArgTypes>
    bool operator==(const function<R(ArgTypes...)>&, nullptr_t) noexcept;
  template<class R, class... ArgTypes>
    bool operator==(nullptr_t, const function<R(ArgTypes...)>&) noexcept;
  template<class R, class... ArgTypes>
    bool operator!=(const function<R(ArgTypes...)>&, nullptr_t) noexcept;
  template<class R, class... ArgTypes>
    bool operator!=(nullptr_t, const function<R(ArgTypes...)>&) noexcept;

  // \ref{func.search}, searchers
  template<class ForwardIterator, class BinaryPredicate = equal_to<>>
    class default_searcher;

  template<class RandomAccessIterator,
           class Hash = hash<typename iterator_traits<RandomAccessIterator>::value_type>,
           class BinaryPredicate = equal_to<>>
    class boyer_moore_searcher;

  template<class RandomAccessIterator,
           class Hash = hash<typename iterator_traits<RandomAccessIterator>::value_type>,
           class BinaryPredicate = equal_to<>>
    class boyer_moore_horspool_searcher;

  // \ref{unord.hash}, hash function primary template
  template<class T>
    struct hash;

  // \ref{func.bind}, function object binders
  template<class T>
    inline constexpr bool is_bind_expression_v = is_bind_expression<T>::value;
  template<class T>
    inline constexpr int is_placeholder_v = is_placeholder<T>::value;
}
\end{codeblock}

Move the sections 23.14.6 [arithmetic.operations], 23.14.7 [comparisons], 23.14.8 [logical.operations], and 23.14.9 [bitwise.operations]j to Appendix D and replace them by the following new section 23.14.6.

\begin{addedblock}
\section{23.14.6 Function Objects for Operators as Lambdas [functional.fool]}


According to \tref{functional.unaryops} provide the corresponding definition of the constexpr variable by initializing it with a lambda expression. For each unary operator \tcode{\atat} in the \tref{functional.unaryops} with the name \tcode{\textit{Name}} define the variable initialized with a unary lambda expression as follows:

\begin{itemdecl}
  constexpr inline auto @\textit{Name}@  = 
    [](auto&& x) constexpr
      noexcept(noexcept( @\textit{\atat}@ std::declval<decltype(x)>()))
      -> decltype(@\textit{\atat}@ std::forward<decltype(x)>(x))
    {
      return @\textit{\atat}@ std::forward<decltype(x)>(x);
    };
\end{itemdecl}

\begin{table}[htp]
\caption{Unary operation function objects}
\begin{center}
\begin{tabular}{|c|c|}
\hline
name & operator \\
\hline
\tcode{Deref} & \tcode{*}\\
\tcode{Address} & \tcode{\&} \\
\tcode{Negate} & \tcode{-}\\
\tcode{Posate} & \tcode{+}\\
\tcode{Not} & \tcode{!} \\
\tcode{Cmpl} & \tcode{\~} \\
\hline
\end{tabular}
\end{center}
\label{functional.unaryops}
\end{table}%




According to \tref{functional.leftbinops} provide the corresponding definition of the constexpr variable by initializing it with a lambda expression. For each binary operator \tcode{\textit{\atat}} in \tref{functional.leftbinops} with the name \tcode{\textit{Name}} define the variable initialized with a lambda expression as follows:
\begin{itemdecl}
  constexpr inline auto @\textit{Name}@ = 
    [](auto&& ... x) constexpr
      noexcept(noexcept(( ... @\textit{\atat}@ std::declval<decltype(x)>()))
      -> decltype((... @\textit{\atat}@ std::forward<decltype(x)>(x)))
    {
      return ( ... @\textit{\atat}@ std::forward<decltype(x)>(x));
    };

\end{itemdecl}
\begin{table}[htp]
\caption{Binary left associative operation function objects}
\begin{center}
\begin{tabular}{|c|c|}
\hline
name & operator \\
\hline
\tcode{PtrMemb} & \tcode{->*}\\
\tcode{RefMemb} & \tcode{.*}\\
\hline
\tcode{Plus} & \tcode{+}\\
\tcode{Minus} & \tcode{-}\\
\tcode{Times} & \tcode{*}\\
\tcode{Divide} & \tcode{/} \\
\tcode{Remainder} & \tcode{\%} \\
\hline
\tcode{Equal} & \tcode{==}\\
\tcode{Unequal} & \tcode{!=}\\
\tcode{Bigger} & \tcode{>}\\
\tcode{Smaller} & \tcode{<} \\
\tcode{Maybe_bigger} & \tcode{>=} \\
\tcode{Sometimes_smaller} & \tcode{<=} \\
\tcode{Spaceship} & \tcode{<=>} \\
\hline
\tcode{And} & \tcode{\&\&}\\
\tcode{Or} & \tcode{||}\\
\hline
\tcode{Bitand} & \tcode{\&}\\
\tcode{Bitor} & \tcode{|}\\
\tcode{Xor} & \tcode{\^}\\
\tcode{Lshift} & \tcode{<<}\\
\tcode{Rshift} & \tcode{>>}\\
\hline
\end{tabular}
\end{center}
\label{functional.leftbinops}
\end{table}%
%%%%%%%%%%%%%%%%
\newpage
According to \tref{functional.rightbinops} provide the corresponding definition of the constexpr variable by initializing it with a lambda expression. For each binary operator \tcode{\textit{\atat}} in \tref{functional.rightbinops} with the name \tcode{\textit{Name}} define the variable initialized with a lambda expression as follows:
\begin{itemdecl}
  constexpr inline auto @\textit{Name}@ = 
    [](auto&& ... x) constexpr
      noexcept(noexcept((std::declval<decltype(x)>() @\textit{\atat}@  ... ))
      -> decltype((std::forward<decltype(x)>(x) @\textit{\atat}@ ... ))
    {
      return (std::forward<decltype(x)>(x) @\textit{\atat}@ ...);
    };
\end{itemdecl}
\begin{table}[htp]
\caption{Binary right associative operation function objects}
\begin{center}
\begin{tabular}{|c|c|}
\hline
name & operator \\
\hline
\tcode{Assign} & \tcode{=} \\
\tcode{PlusAssign} & \tcode{+=} \\
\tcode{MinusAssign} & \tcode{-=} \\
\tcode{TimesAssign} & \tcode{*=} \\
\tcode{DivideAssign} & \tcode{/=} \\
\tcode{RemainderAssign} & \tcode{\%=} \\
\tcode{AndAssign} & \tcode{\&\&=} \\
\tcode{OrAssign} & \tcode{||=} \\
\tcode{And_eq} & \tcode{\&=} \\
\tcode{Or_eq} & \tcode{|=} \\
\tcode{Xor_eq} & \tcode{\^{}=} \\
\tcode{LshiftAssign} & \tcode{<<=} \\
\tcode{RshiftAssign} & \tcode{>>=} \\
\hline
\end{tabular}
\end{center}
\label{functional.rightbinops}
\end{table}%

\newpage
Define the ternary operator's lambda object as follows:
\begin{itemdecl}
constexpr auto Wtf=[](auto&&c, auto&& l, auto&& r) constexpr
noexcept(noexcept(std::declval<decltype(c)>() ? std::declval<decltype(l)>() : std::declval<decltype(r)>()))
-> decltype(std::declval<decltype(c)>() ? std::declval<decltype(l)>() : std::declval<decltype(r)>())
{
	return std::forward<decltype(c)>(c) ?
			std::forward<decltype(l)>(l) : std::forward<decltype(r)>(r);
};
\end{itemdecl}

\end{addedblock}
\end{document}
