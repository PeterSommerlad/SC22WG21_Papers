\documentclass[ebook,11pt,article]{memoir}
\usepackage{geometry}  % See geometry.pdf to learn the layout options. There are lots.
\geometry{a4paper}  % ... or a4paper or a5paper or ... 
%\geometry{landscape}  % Activate for for rotated page geometry
%\usepackage[parfill]{parskip}  % Activate to begin paragraphs with an empty line rather than an indent

%\usepackage[final]
%           {listings}     % code listings
%\usepackage{color}        % define colors for strikeouts and underlines
%\usepackage{underscore}   % remove special status of '_' in ordinary text
%\usepackage{xspace}
%\usepackage[normalem]{ulem}
\usepackage{enumitem}
%%%%% from std.tex:
\usepackage[final]
           {listings}     % code listings
\usepackage{longtable}    % auto-breaking tables
\usepackage{ltcaption}    % fix captions for long tables
\usepackage{relsize}      % provide relative font size changes
\usepackage{textcomp}     % provide \text{l,r}angle
\usepackage{underscore}   % remove special status of '_' in ordinary text
\usepackage{parskip}      % handle non-indented paragraphs "properly"
\usepackage{array}        % new column definitions for tables
\usepackage[normalem]{ulem}
\usepackage{color}        % define colors for strikeouts and underlines
\usepackage{amsmath}      % additional math symbols
\usepackage{mathrsfs}     % mathscr font
\usepackage{microtype}
\usepackage{multicol}
\usepackage{xspace}
\usepackage{lmodern}
\usepackage[T1]{fontenc}
\usepackage[pdftex, final]{graphicx}
\input{macros}
\input{styles}
\input{layout}
%%%%%
\pagestyle{myheadings}

\newcommand{\papernumber}{dXXXXr0}
\newcommand{\paperdate}{2017-05-19}

%\definecolor{insertcolor}{rgb}{0,0.5,0.25}
%\newcommand{\del}[1]{\textcolor{red}{\sout{#1}}}
%\newcommand{\ins}[1]{\textcolor{insertcolor}{\underline{#1}}}
%
%\newenvironment{insrt}{\color{insertcolor}}{\color{black}}


\markboth{\papernumber{} \paperdate{}}{\papernumber{} \paperdate{}}

\title{\papernumber{} - Towards std::units}
\author{Peter Sommerlad}
\date{\paperdate}                        % Activate to display a given date or no date
\setsecnumdepth{subsection}

\begin{document}
\maketitle
\begin{tabular}[t]{|l|l|}\hline 
Document Number:& \papernumber  \\\hline
Date: & \paperdate \\\hline
Project: & Programming Language C++\\\hline 
Audience: & LWG/LEWG\\\hline
Target: & C++20\\\hline
\end{tabular}

\chapter{Motivation}

\chapter{Acknowledgements}
\begin{itemize}
\item All people who inspired me to work on this, because they created and used units libraries.
\item C++now 2017 participants who worked on this during the "library in a week" workshop: Billy Baker, Charles Wilson, Daniel Pfeifer, Dave Jenkins, Manuel Bergler, Morris Hafner, Nicolas Holthaus, Peter Bindels, Steven Watanabe, Tuan Tran.
\end{itemize}

\chapter{Components and Relationships}

\begin{description}
\item [Dimension] 7 physical base dimensions (length, mass, time, current, temperature, amount of substance, luminous intensity), combined, user-defined, special dimensionless (e.g., angles)
\item [Unit] unit of measurement for a given dimension (m,kg,s,A,K,mol,cd) , can be scaled (kilo, micro), 
\item [Quantity] a measurement in a given unit, convertible to other quantity of same dimension
\item [Units System] A set of dimensions and units useful for a domain, e.g., SI units. Combines UDL-suffixes for the units, formatting/IO for quantities, conversions
\item [Conversion] A rule/computation to convert a quantity in one unit into a corresponding quantity of the same dimension in another unit, potentially from another unit system.
\end{description}


\section{Dimensions and Dimensional Analysis}

\subsection{Operational Combinations}

\section{Units}

\section{Quantities}

\section{Units Systems}


\chapter{Specification}


\rSec2[bla]{Bla}

\indexlibrary{\idxcode{bla}}%
\begin{itemdecl}
bla
\end{itemdecl}

\begin{itemdescr}
\pnum
\requires
bla

\pnum
\effects
bla
\end{itemdescr}



\end{document}
