\documentclass[ebook,11pt,article]{memoir}
\usepackage{geometry}                % See geometry.pdf to learn the layout options. There are lots.
\geometry{a4paper}                   % ... or a4paper or a5paper or ... 
%\geometry{landscape}                % Activate for for rotated page geometry
%\usepackage[parfill]{parskip}    % Activate to begin paragraphs with an empty line rather than an indent

\usepackage[final]
           {listings}     % code listings
\usepackage{color}        % define colors for strikeouts and underlines
\usepackage{underscore}   % remove special status of '_' in ordinary text
\usepackage{xspace}
\pagestyle{myheadings}
\markboth{D3915 2014-02-12}{D3915X 2014-02-12}
%%TODO
% namespace std { inline namespace literals { inline namespace chrono_literals, string_literals, complex_literals
% put b on the side
% complex: if, i, il optional i_l

\title{apply() call a function with arguments from a tuple (V3)
}
\author{Peter Sommerlad}
\date{2014-02-12}                                           % Activate to display a given date or no date
\input{macros}
\setsecnumdepth{subsection}

\begin{document}
\maketitle
\begin{tabular}[t]{|l|l|}\hline 
Document Number: & D3915 (update to N3829) \\\hline
Date: & 2014-02-12 \\\hline
Project: & Programming Language C++\\\hline 
\end{tabular}
\chapter{History}
\section{integer_sequence}
This is a slightly update from the wording of N3829 as proposed by LEWG in Issaquah, adding \tcode{std::} again where necessary and \tcode{constexpr}.

N3658 and its predecessor N3493 introduced \tcode{integer_sequence} facility and provide application of this features, for example \tcode{apply()} that is proposed in this paper.
\section{Observations}
There is a lot of history I am unaware of and several implementations posted on StackOverflow. Also the C++14 CD contains an implementation of \tcode{apply()} as an example of \tcode{std::integer_sequence} in [intseq.general].
\section{Using tuplevar... or operator...}
Mike Spertus made me aware of the proposed language extension to form a parameter pack from a tuple, i.e., by overloading an \tcode{operator...} which might make the provision of \tcode{apply()} moot. However, up to now, no such feature has been proposed to the standard committee and it is unclear if it would make it into C++17. Even if it would, it would just make the implementation of apply trivial.
\chapter{Introduction}
Tuples are great for generic programming with variadic templates. However, the standard does not define a general purpose facility that allows to call a function/functor/lambda with the tuple elements as arguments. Such a feature should be provided, because it is useful (at least for me). It even is given as an example of \tcode{std::integer_sequence} in [intseq.general] coming from N3658. 
\section{Rationale}
It is easy to create tuples from variadic templates either from types directly as \tcode{std::tuple<PACK...>} or  with \tcode{std::make_tuple()} or \tcode{std::forward_as_tuple()} the opposite mechanism of passing a tuple's elements as function arguments is not available.

Some suggested to not use the name \tcode{apply() }and reserve that to a mechanism like \emph{\tcode{INVOKE}} in the standard library and use \tcode{apply_from_tuple()} instead.

\section{Open Issues to be Discussed}
It is open, if \tcode{apply()} should get a \tcode{noexcept(noexcept(xxx))} specification delegating to the underlying effect. Advice from library working group was requested, but it seems that it shouldn't get one, because library only provides noexcept specifications for "special" functions, like ctors or swap.

Should \tcode{apply()} be \tcode{constexpr} to make it applicable in meta programming or concepts? LEWG advice is required.

\section{Acknowledgements}
Acknowledgements go to Jonathan Wakely for providing \tcode{integer_sequence} and providing \tcode{apply()} as the example in the working draft. He also suggested a big improvement to the resulting standard change from N3802.

Acknowledgements go to Mike Spertus for making me aware of the ... pack formation approaches.

Acknowledgements to the following persons on the c++std-lib reflector for their feedback and encouragement: Jonathan Wakely, Andy Sawyer, Stephan T. Lavavej, Jared Hoberock, and Tony van Eerd.
%more

\chapter{Possible Implementation}
The following implementation suggestion was derived from N3658, N3690, and StackOverflow (http://stackoverflow.com/a/12650100) and some simplification. It actually seems to work with current \tcode{clang -std=c++1y}.

\begin{codeblock}
template <typename F, typename Tuple, size_t... I>
decltype(auto) apply_impl(F&& f, Tuple&& t, index_sequence<I...>) {
	return forward<F>(f)(get<I>(forward<Tuple>(t))...);
}
template <typename F, typename Tuple>
decltype(auto) apply(F&& f, Tuple&& t) {
  using Indices = make_index_sequence<tuple_size<decay_t<Tuple>>::value>;
  return apply_impl(forward<F>(f), forward<Tuple>(t), Indices{});
}
\end{codeblock}

\chapter{Proposed Library Additions}
%%%%%%%%%%% C++ EDITOR %%%%% 

Add the following declaration in [tuple.general] in the synopsis under the group \emph{element access}:

\begin{codeblock}
template <typename F, typename Tuple>
constexpr decltype(auto) apply(F&& f, Tuple&& t);
\end{codeblock}


Append the following to section [tuple.elem] after paragraph 11.

\begin{itemdecl}
template <typename F, typename Tuple>
constexpr decltype(auto) apply(F&& f, Tuple&& t);
\end{itemdecl}
\begin{itemdescr}
%\pnum
%\requires \tcode{t} provides tuple-like access. \tcode{f} is a function object.

\pnum
\effects Given the exposition only function 
\begin{codeblock}
template <typename F, typename Tuple, size_t... I>
constexpr decltype(auto) 
apply_impl(F&& f, Tuple&& t, index_sequence<I...>) // exposition only
{
	return std::forward<F>(f)(std::get<I>(std::forward<Tuple>(t))...); 
}
\end{codeblock}
%	return INVOKE(std::forward<F>(f),std::get<I>(std::forward<Tuple>(t))...); 
% this INVOKE is doing what INVOKE in the standard does. more general solution not proposed.
\pnum Equivalent to 
\begin{codeblock}
return apply_impl(std::forward<F>(f), std::forward<Tuple>(t),
 make_index_sequence<tuple_size<decay_t<Tuple>>::value>{});
\end{codeblock}
\end{itemdescr}

Delete the example code in [intseq.general] p 2.



%%%%%%%%%%% C++ EDITOR %%%%% 



%%%%
%%%%%

\end{document}   