\documentclass[ebook,11pt,article]{memoir}
\usepackage{geometry}  % See geometry.pdf to learn the layout options. There are lots.
\geometry{a4paper}  % ... or a4paper or a5paper or ... 
%\geometry{landscape}  % Activate for for rotated page geometry
%%% from std.tex
%\usepackage[american]
%           {babel}        % needed for iso dates
%\usepackage[iso,american]
%           {isodate}      % use iso format for dates
\usepackage[final]
           {listings}     % code listings
%\usepackage{longtable}    % auto-breaking tables
%\usepackage{ltcaption}    % fix captions for long tables
\usepackage{relsize}      % provide relative font size changes
%\usepackage{textcomp}     % provide \text{l,r}angle
\usepackage{underscore}   % remove special status of '_' in ordinary text
%\usepackage{parskip}      % handle non-indented paragraphs "properly"
%\usepackage{array}        % new column definitions for tables
\usepackage[normalem]{ulem}
\usepackage{enumitem}
\usepackage{color}        % define colors for strikeouts and underlines
%\usepackage{amsmath}      % additional math symbols
%\usepackage{mathrsfs}     % mathscr font
%\usepackage{microtype}
%\usepackage{multicol}
\usepackage{xspace}
%\usepackage{lmodern}
\usepackage[T1]{fontenc} % makes tilde work! and is better for umlauts etc.
%\usepackage[pdftex, final]{graphicx}
\usepackage[pdftex,
%            pdftitle={C++ International Standard},
%            pdfsubject={C++ International Standard},
%            pdfcreator={Richard Smith},
            bookmarks=true,
            bookmarksnumbered=true,
            pdfpagelabels=true,
            pdfpagemode=UseOutlines,
            pdfstartview=FitH,
            linktocpage=true,
            colorlinks=true,
            linkcolor=blue,
            plainpages=false
           ]{hyperref}
%\usepackage{memhfixc}     % fix interactions between hyperref and memoir
%\usepackage[active,header=false,handles=false,copydocumentclass=false,generate=std-gram.ext,extract-cmdline={gramSec},extract-env={bnftab,simplebnf,bnf,bnfkeywordtab}]{extract} % Grammar extraction
%
\renewcommand\RSsmallest{5.5pt}  % smallest font size for relsize


%%%% reuse all three from std.tex:
\input{macros}
\input{layout}
\input{styles}

\pagestyle{myheadings}


\newcommand{\papernumber}{p0408r1}
\newcommand{\paperdate}{2017-02-03}

%\definecolor{insertcolor}{rgb}{0,0.5,0.25}
%\newcommand{\del}[1]{\textcolor{red}{\sout{#1}}}
%\newcommand{\ins}[1]{\textcolor{insertcolor}{\underline{#1}}}
%
%\newenvironment{insrt}{\color{insertcolor}}{\color{black}}


\markboth{\papernumber{} \paperdate{}}{\papernumber{} \paperdate{}}

\title{\papernumber{} - Efficient Access to basic\_stringbuf's Buffer}
\author{Peter Sommerlad}
\date{\paperdate}                        % Activate to display a given date or no date
\setsecnumdepth{subsection}

\begin{document}
\maketitle
\begin{tabular}[t]{|l|l|}\hline 
Document Number: & \papernumber  \\\hline
Date: & \paperdate \\\hline
Project: & Programming Language C++\\\hline 
Audience: & LWG/LEWG\\\hline
Target: & C++20\\\hline
\end{tabular}

\chapter{Motivation}
Streams have been the oldest part of the C++ standard library and their specification doesn't take into account many things introduced since C++11. One of  the oversights is that there is no non-copying access to the internal buffer of a \tcode{basic_stringbuf} which makes at least the obtaining of the output results from an \tcode{ostringstream} inefficient, because a copy is always made. I personally speculate that this was also the reason why \tcode{basic_strbuf} took so long to get deprecated with its \tcode{char *} access.

With move semantics and \tcode{basic_string_view} there is no longer a reason to keep this pessimissation alive on \tcode{basic_stringbuf}.

\added{I also believe we should remove \tcode{basic_strbuf} from the standard's appendix [depr.str.strstreams].}

\chapter{Introduction}
This paper proposes to adjust the API of \tcode{basic_stringbuf} and the corresponding stream class templates to allow accessing the underlying string more efficiently.

C++17 and library TS have \tcode{basic_string_view} allowing an efficient read-only access to a contiguous sequence of characters which I believe \tcode{basic_stringbuf} has to guarantee about its internal buffer, even if it is not implemented using \tcode{basic_string} obtaining a \tcode{basic_string_view} on the internal buffer should work sidestepping the copy overhead of calling \tcode{str()}. 

On the other hand, there is no means to construct a \tcode{basic_string} and move from it into a \tcode{basic_stringbuf} via a constructor or a move-enabled overload of \tcode{str(basic_string \&\&)}.

\section{History}
Discussed in LEWG Issaquah. Answering some questions and raising more. Reflected in this paper.
\subsection{Changes from r0}
\begin{itemize}
\item Added more context to synopsis sections to see all overloads (Thanks Alisdair).
\item rename \tcode{str_view()} to just \tcode{view()}. There was discussion on including an explicit conversion operator as well, but I didn't add it yet (my implementation has it).
\item renamed r-value-ref qualified \tcode{str()} to \tcode{pilfer()} and removed the reference qualification from it and remaining \tcode{str()} member.
\item Added allocator parameter for the \tcode{basic_string} parameter/result to member functions (see p0407 for allocator support for stringstreams in general)
\end{itemize}


\chapter{Acknowledgements}
\begin{itemize}
\item Daniel Kr\"ugler encouraged me to pursue this track.
\item \added{Alisdair Meredith for telling me to include context in the synopsis showing all overloads. That is the only change in this version, no semantic changes!}
\end{itemize}


%\chapter{example code}


\chapter{Impact on the Standard}
This is an extension to the API of \tcode{basic_stringbuf}, \tcode{basic_stringstream}, \tcode{basic_istringstream}, and \tcode{basic_ostringstream} class templates.

\added{This paper addresses both Library Fundamentals TS 3 and C++Next (2020?). 
When added to the standard draft section [depr.str.strstreams] should be removed from it.}
\chapter{Design Decisions}
After experimentation I decided that substituting the \tcode{(basic_string<charT,traits,Allocator const \&)} constructors in favor of passing a \tcode{basic_string_view} would lead to ambiguities with the new move-from-string constructors.
\section{Open Issues discussed by LEWG in Issquah}
\begin{itemize}
\item Is the name of the \tcode{str_view()} member function ok? No. Renamed to \tcode{view()}
\item Should the \tcode{str()\&\&} overload be provided for move-out? No. give it another name (\tcode{pilfer}) and remove rvalue-ref-qualification.
\item Should \tcode{str()\&\&} empty the character sequence or leave it in an unspecified but valid state? Empty it, and specify.
\item Provide guidance on validity lifetime of of the obtained \tcode{string_view} object.
\end{itemize}

\chapter{Technical Specifications}
The following is relative to n4604.

Remove section on \tcode{char*} streams [depr.str.strstreams] and all its subsections from appendix D.

\section{27.8.2 Adjust synopsis of basic\_stringbuf [stringbuf]}
Add a new constructor overload:
\begin{codeblock}
    // \ref{stringbuf.cons}, constructors:
    explicit basic_stringbuf(
      ios_base::openmode which = ios_base::in | ios_base::out);
    @\added{template<class SAlloc=Allocator>}@
    explicit basic_stringbuf(
      const basic_string<charT, traits, @\added{S}@Alloc@\removed{ator}@>& str,
      ios_base::openmode which = ios_base::in | ios_base::out);
\end{codeblock}
\begin{addedblock}\begin{codeblock}
    explicit basic_stringbuf(
      basic_string<charT, traits, Allocator>&& s,
      ios_base::openmode which = ios_base::in | ios_base::out);
\end{codeblock}\end{addedblock}
\begin{codeblock}
    basic_stringbuf(const basic_stringbuf& rhs) = delete;
    basic_stringbuf(basic_stringbuf&& rhs);
\end{codeblock}

Change the getting \tcode{str()} overload to take an Allocator for the returned string. 
Change the \tcode{str()} overload copying into the string buffer to take an allocator template parameter that could differ from the buffer's own \tcode{Allocator}. 
Add a \tcode{str()} overload that moves from its string rvalue-reference argument into the internal buffer.
Add the \tcode{pilfer()} member function obtaining a string from the internal buffer by moving from it. 
Add the \tcode{view()} member function obtaining a \tcode{string_view} to the underlying internal buffer.

\begin{codeblock}
    // \ref{stringbuf.members}, get and set:
    @\added{template<class SAlloc=Allocator>}@
    basic_string<charT,traits,@\added{S}@Alloc@\removed{ator}@> str(@\added{const SAlloc\& sa=SAlloc()}@) const;
    @\added{template<class SAlloc=Allocator>}@
    void str(const basic_string<charT, traits, @\added{S}@Alloc@\removed{ator}@>& s);
\end{codeblock}
\begin{addedblock}
\begin{codeblock}
    void str(basic_string<charT, traits, Allocator>&& s);
    basic_string<charT,traits,Allocator> pilfer();
    basic_string_view<charT, traits> view() const;
\end{codeblock}
\end{addedblock}

\subsection{27.8.2.1 basic\_stringbuf constructors [stringbuf.cons]}
Modify the following constructor specification:
\begin{itemdecl}
@\added{template<class SAlloc=Allocator>}@
explicit basic_stringbuf(
  const basic_string<charT, traits, @\added{S}@Alloc@\removed{ator}@>& str,
  ios_base::openmode which = ios_base::in | ios_base::out);
\end{itemdecl}

\begin{itemdescr}
\pnum
\effects
Constructs an object of class
\tcode{basic_stringbuf},
initializing the base class with
\tcode{basic_streambuf()}~(\ref{streambuf.cons}), and initializing
\tcode{mode}
with \tcode{which}.
Then calls \tcode{str(s)}.
\end{itemdescr}

Add the following constructor specification:
\begin{insrt}
\begin{itemdecl}
      explicit basic_stringbuf(
        basic_string<charT, traits, Allocator>&& s,
        ios_base::openmode which = ios_base::in | ios_base::out);
\end{itemdecl}
\begin{itemdescr}
\pnum
\effects Constructs an object of class \tcode{basic_stringbuf}, initializing the base class with \tcode{basic_streambuf()} (27.6.3.1), and initializing \tcode{mode} with \tcode{which}. Then calls \tcode{str(std::move(s))}.
\end{itemdescr}
\end{insrt}

Note to editors: if p0407 is accepted the changes there for allocators apply here as well. However, different allocators for \tcode{s} and the \tcode{basic_stringbuf} will result in a copy instead of a move.

\subsection{27.8.2.3 Member functions [stringbuf.members]}
Add an allocator parameter for the copied from string to allow having a different allocator than the underlying stream.
\begin{codeblock}
@\added{template<class SAlloc=Allocator>}@
basic_string<charT,traits,@\added{S}@Alloc@\removed{ator}@> str(@\added{const SAlloc\& sa=SAlloc()}@) const;
\end{codeblock}

Change p1 to use plural for "\tcode{str(basic_string)} member functions" and refer to the allocator:

\begin{itemdescr}
\pnum
\returns
A
\tcode{basic_string}
object \added{with allocator \tcode{sa} }whose content is equal to the
\tcode{basic_stringbuf}
underlying character sequence.
If the \tcode{basic_stringbuf} was created only in input mode, the resultant
\tcode{basic_string} contains the character sequence in the range
\range{eback()}{egptr()}. If the \tcode{basic_stringbuf} was created with
\tcode{which \& ios_base::out} being true then the resultant \tcode{basic_string}
contains the character sequence in the range \range{pbase()}{high_mark}, where
\tcode{high_mark} represents the position one past the highest initialized character
in the buffer. Characters can be initialized by writing to the stream, by constructing
the \tcode{basic_stringbuf} with a \tcode{basic_string}, or by calling 
\added{one of }
the
\tcode{str(basic_string)} member function\added{s}. In the case of calling 
\added{one of }
the
\tcode{str(basic_string)} member function\added{s}, all characters initialized prior to
the call are now considered uninitialized (except for those characters re-initialized
by the new \tcode{basic_string}). Otherwise the \tcode{basic_stringbuf} has been created
in neither input nor output mode and a zero length \tcode{basic_string} is returned. 
\end{itemdescr}

Add the following specifications and adjust the wording of \tcode{str() const} according to the wording given for \tcode{view() const} member function.:
\begin{insrt}
\begin{itemdecl}
void str(basic_string<charT, traits, Allocator>&& s);
\end{itemdecl}
\begin{itemdescr}
\pnum
\effects 
Moves the content of \tcode{s} into the \tcode{basic_stringbuf} underlying character sequence and initializes the input and output sequences according to \tcode{mode}.
%% mode is an exposition-only member of basic_strinbuf

\pnum
\postconditions
Let \tcode{size} denote the original value of \tcode{s.size()} before the move.
If \tcode{mode \& ios_base::out} is true, \tcode{pbase()} points to the first underlying character and \tcode{epptr() >= pbase() + size} holds; in addition, if \tcode{mode \& ios_base::ate} is true, \tcode{pptr() == pbase() + size} holds, otherwise \tcode{pptr() == pbase()} is true. If \tcode{mode \& ios_base::in} is true, \tcode{eback()} points to the first underlying character, and both \tcode{gptr() == eback()} and \tcode{egptr() == eback() + size} hold.
\end{itemdescr}

\begin{itemdecl}
basic_string<charT, traits, Allocator> pilfer();
\end{itemdecl}
\begin{itemdescr}

\pnum
\returns A \tcode{basic_string} object moved from the \tcode{basic_stringbuf} underlying character sequence. If the \tcode{basic_stringbuf} was created only in input mode, \tcode{basic_string(eback(), egptr()-eback())}. If the \tcode{basic_stringbuf} was created with \tcode{which \& ios_base::out} being true then \tcode{basic_string(pbase(), high_mark-pbase())}, where \tcode{high_mark} represents the position one past the highest initialized character in the buffer. Characters can be initialized by writing to the stream, by constructing the \tcode{basic_stringbuf} with a \tcode{basic_string}, or by calling one of the \tcode{str(basic_string)} member functions. In the case of calling one of the \tcode{str(basic_string)} member functions, all characters initialized prior to the call are now considered uninitialized (except for those characters re-initialized by the new \tcode{basic_string}). Otherwise the \tcode{basic_stringbuf} has been created in neither input nor output mode and an empty \tcode{basic_string} is returned. 

\pnum
\postconditions The underlying character sequence is empty.
\end{itemdescr}

\begin{itemdecl}
basic_string_view<charT, traits> view() const;
\end{itemdecl}
\begin{itemdescr}
\pnum
\returns A \tcode{basic_string_view} object referring to the \tcode{basic_stringbuf} underlying character sequence. If the \tcode{basic_stringbuf} was created only in input mode,  \tcode{basic_string_view(eback(), egptr()-eback())}. If the \tcode{basic_stringbuf} was created with \tcode{which \& ios_base::out} being true then \tcode{basic_string_view(pbase(), high_mark-pbase())}, where \tcode{high_mark} represents the position one past the highest initialized character in the buffer. Characters can be initialized by writing to the stream, by constructing the \tcode{basic_stringbuf} with a \tcode{basic_string}, or by calling one of the \tcode{str(basic_string)} member functions. In the case of calling one of the \tcode{str(basic_string)} member functions, all characters initialized prior to the call are now considered uninitialized (except for those characters re-initialized by the new \tcode{basic_string}). Otherwise the \tcode{basic_stringbuf} has been created in neither input nor output mode and a \tcode{basic_string_view} referring to an empty range is returned. 

\pnum
\begin{note}
Using the returned \tcode{basic_string_view} object after destruction or any modification of the character sequence underlying \tcode{*this}, such as output on the holding stream, will cause undefined behavior, because the internal string referred by the return value might have changed or re-allocated. 
\end{note}
\end{itemdescr}

\end{insrt}
%% istream
\section{27.8.3 Adjust synopsis of basic\_istringstream [istringstream]}
Add a new constructor overload and change the one taking the string by copy to allow a different allocator for the copied from string:
\begin{codeblock}
    // \ref{istringstream.cons}, constructors:
    explicit basic_istringstream(
      ios_base::openmode which = ios_base::in);
    @\added{template<class SAlloc=Allocator>}@
    explicit basic_istringstream(
      const basic_string<charT, traits, @\added{S}@Alloc@\removed{ator}@>& str,
      ios_base::openmode which = ios_base::in);
\end{codeblock}
\begin{addedblock}
\begin{codeblock}
    explicit basic_istringstream(
      basic_string<charT, traits, Allocator>&& str,
      ios_base::openmode which = ios_base::in);
\end{codeblock}
\end{addedblock}
\begin{codeblock}
    basic_istringstream(const basic_istringstream& rhs) = delete;
    basic_istringstream(basic_istringstream&& rhs);
\end{codeblock}

Change the \tcode{str()} member function to allow different allocator argument for the new string to be used or the obtained string copy.
Add an overload of the \tcode{str(s)} member function that moves from a string and add \tcode{pilfer()} and \tcode{view()} member function:

\begin{codeblock}
    // \ref{istringstream.members}, members:
    basic_stringbuf<charT, traits, Allocator>* rdbuf() const;

    @\added{template<class SAlloc=Allocator>}@
    basic_string<charT,traits,@\removed{Allocator}\added{SAlloc}@> str(@\added{const SAlloc\& sa=SAlloc()}@) const;
    @\added{template<class SAlloc=Allocator>}@
    void str(const basic_string<charT, traits, @\added{S}@Alloc@\removed{ator}@>& s);
\end{codeblock}
\begin{addedblock}
\begin{codeblock}
    void str(basic_string<charT, traits, Allocator>&& s);
    basic_string<charT,traits,Allocator> pilfer();
    basic_string_view<charT, traits> view() const;
\end{codeblock}
\end{addedblock}

\subsection{27.8.3.1 basic\_istringstream constructors [istringstream.cons]}
Change the constructor specification to allow a string copy with a different allocator.
\begin{itemdecl}
@\added{template<class SAlloc=Allocator>}@
explicit basic_istringstream(
  const basic_string<charT, traits, @\added{S}@Alloc@\removed{ator}@>& str,
  ios_base::openmode which = ios_base::in);
\end{itemdecl}

\begin{itemdescr}
\pnum
\effects
Constructs an object of class
\tcode{basic_istringstream<charT, traits>},
initializing the base class with
\tcode{basic_istream(\&sb)}
and initializing \tcode{sb} with
\tcode{basic_stringbuf<charT, traits, Alloca\-tor>(str, which | ios_base::in))}~(\ref{stringbuf.cons}).
\end{itemdescr}


Add the following constructor specification:

\begin{addedblock}
\begin{itemdecl}
explicit basic_istringstream(
  const basic_string<charT, traits, Allocator>&& str,
  ios_base::openmode which = ios_base::in);
\end{itemdecl}
\begin{itemdescr}
\pnum
\effects Constructs an object of class \tcode{basic_istringstream<charT, traits>}, initializing the base class with \tcode{basic_istream(\&sb)} and initializing \tcode{sb} with \tcode{basic_stringbuf<charT, traits, Allocator>(std::move(str), which | ios_base::in))} (27.8.2.1).
\end{itemdescr}
\end{addedblock}


\subsection{27.8.3.3 Member functions [istringstream.members]}
Add the allocator parameter to the following str() overloads:
\begin{itemdecl}
@\added{template<class SAlloc=Allocator>}@
basic_string<charT,traits,@\added{S}@Alloc@\removed{ator}@> str(@\added{const SAlloc\& sa=SAlloc()}@) const;
\end{itemdecl}
\begin{itemdescr}
\pnum
\returns
\tcode{rdbuf()->str(\added{sa})}.
\end{itemdescr}

\begin{itemdecl}
    @\added{template<class SAlloc=Allocator>}@
    void str(const basic_string<charT, traits, @\added{S}@Alloc@\removed{ator}@>& s);
\end{itemdecl}

\begin{itemdescr}
\pnum
\effects
Calls
\tcode{rdbuf()->str(s)}.
\end{itemdescr}


Add the following specifications:

\begin{addedblock}
\begin{itemdecl}
void str(basic_string<charT, traits, Allocator>&& s);
\end{itemdecl}
\begin{itemdescr}
\pnum
\effects \tcode{rdbuf()->str(std::move(s))}.
\end{itemdescr}
\begin{itemdecl}
basic_string<charT,traits,Allocator> pilfer();
\end{itemdecl}
\begin{itemdescr}
\pnum
\returns \tcode{rdbuf()->pilfer()}.
\end{itemdescr}
\begin{itemdecl}
basic_string_view<charT, traits> view() const;
\end{itemdecl}
\begin{itemdescr}
\pnum
\returns \tcode{rdbuf()->view()}.
\end{itemdescr}
\end{addedblock}

%%ostream

\section{27.8.4 Adjust synopsis of basic\_ostringstream [ostringstream]}
Add a new constructor overload and change the one taking the string by copy to allow a different allocator for the copied from string:
\begin{codeblock}
    // \ref{ostringstream.cons}, constructors:
    explicit basic_ostringstream(
      ios_base::openmode which = ios_base::out);
    @\added{template<class SAlloc=Allocator>}@
    explicit basic_ostringstream(
      const basic_string<charT, traits, @\added{S}@Alloc@\removed{ator}@>& str,
      ios_base::openmode which = ios_base::out);
\end{codeblock}
\begin{addedblock}
\begin{codeblock}
    explicit basic_ostringstream(
      basic_string<charT, traits, Allocator>&& str,
      ios_base::openmode which = ios_base::out);
\end{codeblock}
\end{addedblock}
\begin{codeblock}
    basic_ostringstream(const basic_ostringstream& rhs) = delete;
    basic_ostringstream(basic_ostringstream&& rhs);
\end{codeblock}

Change the \tcode{str()} member function to allow different allocator argument for the new string to be used or the obtained string copy.
Add an overload of the \tcode{str(s)} member function that moves from a string and add \tcode{pilfer()} and \tcode{view()} member function:

\begin{codeblock}
    // \ref{ostringstream.members}, members:
    basic_stringbuf<charT, traits, Allocator>* rdbuf() const;

    @\added{template<class SAlloc=Allocator>}@
    basic_string<charT,traits,@\removed{Allocator}\added{SAlloc}@> str(@\added{const SAlloc\& sa=SAlloc()}@) const;
    @\added{template<class SAlloc=Allocator>}@
    void str(const basic_string<charT, traits, @\added{S}@Alloc@\removed{ator}@>& s);
\end{codeblock}
\begin{addedblock}
\begin{codeblock}
    void str(basic_string<charT, traits, Allocator>&& s);
    basic_string<charT,traits,Allocator> pilfer();
    basic_string_view<charT, traits> view() const;
\end{codeblock}
\end{addedblock}

\subsection{27.8.4.1 basic\_ostringstream constructors [ostringstream.cons]}
Change the constructor specification to allow a string copy with a different allocator.
\begin{itemdecl}
@\added{template<class SAlloc=Allocator>}@
explicit basic_ostringstream(
  const basic_string<charT, traits, @\added{S}@Alloc@\removed{ator}@>& str,
  ios_base::openmode which = ios_base::out);
\end{itemdecl}

\begin{itemdescr}
\pnum
\effects
Constructs an object of class
\tcode{basic_ostringstream<charT, traits>},
initializing the base class with
\tcode{basic_ostream(\&sb)}
and initializing \tcode{sb} with
\tcode{basic_stringbuf<charT, traits, Alloca\-tor>(str, which | ios_base::out))}~(\ref{stringbuf.cons}).
\end{itemdescr}

Add the following constructor specification:
\begin{insrt}
\begin{itemdecl}
explicit basic_ostringstream(
  const basic_string<charT, traits, Allocator>&& str,
  ios_base::openmode which = ios_base::out);
\end{itemdecl}
\begin{itemdescr}
\pnum
\effects Constructs an object of class \tcode{basic_ostringstream<charT, traits>}, initializing the base class with \tcode{basic_ostream(\&sb)} and initializing \tcode{sb} with \tcode{basic_stringbuf<charT, traits, Allocator>(std::move(str), which | ios_base::out))} (27.8.2.1).
\end{itemdescr}
\end{insrt}

\subsection{27.8.4.3 Member functions [ostringstream.members]}
Add the allocator parameter to the following str() overloads:
\begin{itemdecl}
@\added{template<class SAlloc=Allocator>}@
basic_string<charT,traits,@\added{S}@Alloc@\removed{ator}@> str(@\added{const SAlloc\& sa=SAlloc()}@) const;
\end{itemdecl}
\begin{itemdescr}
\pnum
\returns
\tcode{rdbuf()->str(\added{sa})}.
\end{itemdescr}

\begin{itemdecl}
    @\added{template<class SAlloc=Allocator>}@
    void str(const basic_string<charT, traits, @\added{S}@Alloc@\removed{ator}@>& s);
\end{itemdecl}

\begin{itemdescr}
\pnum
\effects
Calls
\tcode{rdbuf()->str(s)}.
\end{itemdescr}


Add the following specifications:

\begin{addedblock}
\begin{itemdecl}
void str(basic_string<charT, traits, Allocator>&& s);
\end{itemdecl}
\begin{itemdescr}
\pnum
\effects \tcode{rdbuf()->str(std::move(s))}.
\end{itemdescr}
\begin{itemdecl}
basic_string<charT,traits,Allocator> pilfer();
\end{itemdecl}
\begin{itemdescr}
\pnum
\returns \tcode{rdbuf()->pilfer()}.
\end{itemdescr}
\begin{itemdecl}
basic_string_view<charT, traits> view() const;
\end{itemdecl}
\begin{itemdescr}
\pnum
\returns \tcode{rdbuf()->view()}.
\end{itemdescr}
\end{addedblock}


%%stringstream
\section{27.8.5 Adjust synopsis of basic\_stringstream [stringstream]}
Add a new constructor overload and change the one taking the string by copy to allow a different allocator for the copied from string:
\begin{codeblock}
    // \ref{stringstream.cons}, constructors:
    explicit basic_stringstream(
      ios_base::openmode which = ios_base::out | ios_base::in);
    @\added{template<class SAlloc=Allocator>}@
    explicit basic_stringstream(
      const basic_string<charT, traits, @\added{S}@Alloc@\removed{ator}@>& str,
      ios_base::openmode which = ios_base::out | ios_base::in);
\end{codeblock}
\begin{addedblock}
\begin{codeblock}
    explicit basic_stringstream(
      basic_string<charT, traits, Allocator>&& str,
      ios_base::openmode which = ios_base::in | ios_base::out);
\end{codeblock}
\end{addedblock}
\begin{codeblock}
    basic_stringstream(const basic_stringstream& rhs) = delete;
    basic_stringstream(basic_stringstream&& rhs);
\end{codeblock}

Change the \tcode{str()} member function to allow different allocator argument for the new string to be used or the obtained string copy.
Add an overload of the \tcode{str(s)} member function that moves from a string and add \tcode{pilfer()} and \tcode{view()} member function:

\begin{codeblock}
    // \ref{stringstream.members}, members:
    basic_stringbuf<charT, traits, Allocator>* rdbuf() const;

    @\added{template<class SAlloc=Allocator>}@
    basic_string<charT,traits,@\removed{Allocator}\added{SAlloc}@> str(@\added{const SAlloc\& sa=SAlloc()}@) const;
    @\added{template<class SAlloc=Allocator>}@
    void str(const basic_string<charT, traits, @\added{S}@Alloc@\removed{ator}@>& s);
\end{codeblock}
\begin{addedblock}
\begin{codeblock}
    void str(basic_string<charT, traits, Allocator>&& s);
    basic_string<charT,traits,Allocator> pilfer();
    basic_string_view<charT, traits> view() const;
\end{codeblock}
\end{addedblock}

\subsection{27.8.4.1 basic\_stringstream constructors [stringstream.cons]}
Change the constructor specification to allow a string copy with a different allocator.
\begin{itemdecl}
@\added{template<class SAlloc=Allocator>}@
explicit basic_stringstream(
  const basic_string<charT, traits, @\added{S}@Alloc@\removed{ator}@>& str,
  ios_base::openmode which = ios_base::out | ios_base::in);
\end{itemdecl}

\begin{itemdescr}
\pnum
\effects
Constructs an object of class
\tcode{basic_stringstream<charT, traits>},
initializing the base class with
\tcode{basic_iostream(\&sb)}
and initializing
\tcode{sb}
with
\tcode{basic_stringbuf<charT, traits, Alloca\-tor>(str, which)}.
\end{itemdescr}

Add the following constructor specification:
\begin{insrt}
\begin{itemdecl}
explicit basic_stringstream(
  const basic_string<charT, traits, Allocator>&& str,
  ios_base::openmode which = ios_base::in | ios_base::out);
\end{itemdecl}
\begin{itemdescr}
\pnum
\effects Constructs an object of class \tcode{basic_stringstream<charT, traits>}, initializing the base class with \tcode{basic_stream(\&sb)} and initializing \tcode{sb} with \tcode{basic_stringbuf<charT, traits, Allocator>(std::move(str), which))} (27.8.2.1).
\end{itemdescr}
\end{insrt}

\subsection{27.8.4.3 Member functions [stringstream.members]}
Add the allocator parameter to the following str() overloads:
\begin{itemdecl}
@\added{template<class SAlloc=Allocator>}@
basic_string<charT,traits,@\added{S}@Alloc@\removed{ator}@> str(@\added{const SAlloc\& sa=SAlloc()}@) const;
\end{itemdecl}
\begin{itemdescr}
\pnum
\returns
\tcode{rdbuf()->str(\added{sa})}.
\end{itemdescr}

\begin{itemdecl}
    @\added{template<class SAlloc=Allocator>}@
    void str(const basic_string<charT, traits, @\added{S}@Alloc@\removed{ator}@>& s);
\end{itemdecl}

\begin{itemdescr}
\pnum
\effects
Calls
\tcode{rdbuf()->str(s)}.
\end{itemdescr}


Add the following specifications:

\begin{addedblock}
\begin{itemdecl}
void str(basic_string<charT, traits, Allocator>&& s);
\end{itemdecl}
\begin{itemdescr}
\pnum
\effects \tcode{rdbuf()->str(std::move(s))}.
\end{itemdescr}
\begin{itemdecl}
basic_string<charT,traits,Allocator> pilfer();
\end{itemdecl}
\begin{itemdescr}
\pnum
\returns \tcode{rdbuf()->pilfer()}.
\end{itemdescr}
\begin{itemdecl}
basic_string_view<charT, traits> view() const;
\end{itemdecl}
\begin{itemdescr}
\pnum
\returns \tcode{rdbuf()->view()}.
\end{itemdescr}
\end{addedblock}



\chapter{Appendix: Example Implementations}

The given specification has been implemented within a recent version of the sstream header of gcc6. Modified version of the headers and some tests are available at

{https://github.com/PeterSommerlad/SC22WG21_Papers/tree/master/workspace/Test_basic_stringbuf_efficient/src}.


\end{document}

