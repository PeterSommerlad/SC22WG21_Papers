\documentclass[ebook,11pt,article]{memoir}
\usepackage{geometry}  % See geometry.pdf to learn the layout options. There are lots.
\geometry{a4paper}  % ... or a4paper or a5paper or ... 
%\geometry{landscape}  % Activate for for rotated page geometry
%\usepackage[parfill]{parskip}  % Activate to begin paragraphs with an empty line rather than an indent


\usepackage[final]
           {listings}     % code listings
\usepackage{color}        % define colors for strikeouts and underlines
\usepackage{underscore}   % remove special status of '_' in ordinary text
\usepackage{xspace}
\usepackage[normalem]{ulem} % for insertion marks with \uline
\usepackage{url}

\pagestyle{myheadings}

\newcommand{\papernumber}{p0448r0}
\newcommand{\paperdate}{2016-10-10}

\markboth{\papernumber{} \paperdate{}}{\papernumber{} \paperdate{}}

\title{\papernumber{} - A strstream replacement using span\textless{}charT\textgreater{} as buffer}
\author{Peter Sommerlad}
\date{\paperdate}                % Activate to display a given date or no date
\input{macros}
\setsecnumdepth{subsection}

\begin{document}
\maketitle
\begin{tabular}[t]{|l|l|}\hline 
Document Number: \papernumber & (N2065 done right?)\\\hline
Date: & \paperdate \\\hline
Project: & Programming Language C++\\\hline 
Audience: & LWG/LEWG\\\hline
\end{tabular}

\chapter{History}
Streams have been the oldest part of the C++ standard library and especially strstreams that can use pre-allocated buffers have been deprecated for a long time now, waiting for a replacement. p0407 and p0408 provide the efficient access to the underlying buffer for stringstreams that strstream provided solving half of the problem that strstreams provide a solution for. The other half is using a fixed size pre-allocated buffer, e.g., allocated on the stack, that is used as the stream buffers internal storage.

A combination of external-fixed and internal-growing buffer allocation that strstreambuf provides is IMHO a doomed approach and very hard to use right.

There had been a proposal for the pre-allocated external memory buffer streams in N2065 but that went nowhere. Today, with \tcode{span<T>} we actually have a library type representing such buffers views we can use for specifying (and implementing) such streams. They can be used in areas where dynamic (re-)allocation of stringstreams is not acceptable but the burden of caring for a pre-existing buffer during the lifetime of the stream is manageable. 

\chapter{Introduction}
This paper proposes a class template \tcode{basic_spanbuf} and the corresponding stream class templates to enable the use of streams on externally provided memory buffers. No ownership or re-allocation support is given. For those features we have string-based streams.

\chapter{Acknowledgements}
\begin{itemize}
\item Thanks go to Jonathan Wakely who pointed the problem of \tcode{strstream} out to me and to Neil Macintosh to provide the span library type specification.
\end{itemize}

\chapter{Motivation}
To finally get rid of the deprecated \tcode{strstream} in the C++ standard we need a replacement. p0407/p0408 provide one for one half of the needs for \tcode{strstream}. This paper provides one for the second half: fixed sized buffers. 

\begin{example} reading input from a fixed pre-arranged character buffer:
\begin{codeblock}
char input[] = "10 20 30";
ispanstream is{span<char>{input}};
int i;
is >> i;
ASSERT_EQUAL(10,i);
is >> i ;
ASSERT_EQUAL(20,i);
is >> i;
ASSERT_EQUAL(30,i);
is >>i;
ASSERT(!is);
\end{codeblock}
\end{example}
\begin{example} writing to a fixed pre-arranged character buffer:
\begin{codeblock}
char  output[30]{}; // zero-initialize array
ospanstream os{span<char >{output}};
os << 10 << 20 << 30 ;
auto const sp = os.span();
ASSERT_EQUAL(6,sp.size());
ASSERT_EQUAL("102030",std::string(sp.data(),sp.size()));
ASSERT_EQUAL(static_cast<void*>(output),sp.data()); // no copying of underlying data!
ASSERT_EQUAL("102030",output); // initialization guaranteed NUL termination
\end{codeblock}
\end{example}

\chapter{Impact on the Standard}
This is an extension to the standard library to enable deletion of the deprecated \tcode{strstream} classes by providing \tcode{basic_spanbuf}, \tcode{basic_spanstream}, \tcode{basic_ispanstream}, and \tcode{basic_ospanstream} class templates that take an object of type \tcode{span<charT>} which provides an external buffer to be used by the stream. 

\chapter{Design Decisions}
\section{General Principles}
\section{Open Issues to be Discussed by LEWG / LWG}
\begin{itemize}
\item Should arbitrary types as template arguments to \tcode{span} be allowed to provide the underlying buffer by using the \tcode{byte} sequence representation \tcode{span} provides. (I do not think so, but someone might have a usecase.)
\item Should the \tcode{basic_spanbuf} be copy-able? It doesn't own any resources, so copying like with handles or \tcode{span} might be fine.
\end{itemize}

\chapter{Technical Specifications}
Insert a new section 27.x in chapter 27 after section 27.8 [string.streams]

\section{27.x Span-based Streams [span.streams]}
This section introduces a stream interface for user-provided fixed-size buffers. 
\subsection{27.x.1 Overview [span.streams.overview]}
The header \tcode{<spanstream>} defines four class templates and eight types that associate stream buffers with objects of class \tcode{span} as described in [span].

\paragraph{Header \tcode{<spanstream>} synobsis}

\begin{codeblock}
namespace std {
namespace experimental {
  template <class charT, class traits = char_traits<charT> >
    class basic_spanbuf;
  typedef basic_spanbuf<char>     spanbuf;
  typedef basic_spanbuf<wchar_t> wspanbuf;
  template <class charT, class traits = char_traits<charT> >
    class basic_ispanstream;
  typedef basic_ispanstream<char>     ispanstream;
  typedef basic_ispanstream<wchar_t> wispanstream;
  template <class charT, class traits = char_traits<charT> >
    class basic_ospanstream;
  typedef basic_ospanstream<char>     ospanstream;
  typedef basic_ospanstream<wchar_t> wospanstream;
  template <class charT, class traits = char_traits<charT> >
    class basic_spanstream;
  typedef basic_spanstream<char>     spanstream;
  typedef basic_spanstream<wchar_t> wspanstream;
}}
\end{codeblock}
\section{27.x.2 Class template \tcode{basic_spanbuf} [spanbuf]}
%\rSec2[spanbuf]{Class template \tcode{basic_spanbuf}}
%\indexlibrary{\idxcode{basic_spanbuf}}%
\begin{codeblock}
namespace std {
  template <class charT, class traits = char_traits<charT> >
  class basic_spanbuf
    : public basic_streambuf<charT, traits> {
  public:
    using char_type      = charT;
    using int_type       = typename traits::int_type;
    using pos_type       = typename traits::pos_type;
    using off_type       = typename traits::off_type;
    using traits_type    = traits;

    // \ref{spanbuf.cons}, constructors:
    template <ptrdiff_t Extent>
    explicit basic_spanbuf(
      span<charT, Extent> span,
      ios_base::openmode which = ios_base::in | ios_base::out);
    basic_spanbuf(const basic_spanbuf& rhs) = delete;
    basic_spanbuf(basic_spanbuf&& rhs) noexcept;

    // \ref{spanbuf.assign}, assign and swap:
    basic_spanbuf& operator=(const basic_spanbuf& rhs) = delete;
    basic_spanbuf& operator=(basic_spanbuf&& rhs) noexcept;
    void swap(basic_spanbuf& rhs) noexcept;

    // \ref{spanbuf.members}, get and set:
    span<charT> span() const noexcept;
    void span(span<charT> s) noexcept;

  protected:
    // \ref{spanbuf.virtuals}, overridden virtual functions:
    int_type underflow() override;
    int_type pbackfail(int_type c = traits::eof()) override;
    int_type overflow (int_type c = traits::eof()) override;
    basic_streambuf<charT, traits>* setbuf(charT*, streamsize) override;

    pos_type seekoff(off_type off, ios_base::seekdir way,
                     ios_base::openmode which
                      = ios_base::in | ios_base::out) override;
    pos_type seekpos(pos_type sp,
                     ios_base::openmode which
                      = ios_base::in | ios_base::out) override;

  private:
    ios_base::openmode mode;  // \expos
  };

  template <class charT, class traits>
    void swap(basic_spanbuf<charT, traits>& x,
              basic_spanbuf<charT, traits>& y) noexcept;
}
\end{codeblock}

\pnum
The class
\tcode{basic_spanbuf}
is derived from
\tcode{basic_streambuf}
to associate possibly the input sequence and possibly
the output sequence with a sequence of arbitrary
\term{characters}.
The sequence is provided by an object of class
\tcode{span<charT>}.

\pnum
For the sake of exposition, the maintained data is presented here as:
\begin{itemize}
\item
\tcode{ios_base::openmode mode},
has
\tcode{in}
set if the input sequence can be read, and
\tcode{out}
set if the output sequence can be written.
\end{itemize}
%%%

\section{27.x.2.1 \tcode{basic_spanbuf} constructors [spanbuf.cons]}
%\rSec3[spanbuf.cons]{\tcode{basic_spanbuf}  constructors}

\indexlibrary{\idxcode{basic_spanbuf}!constructor}%
\begin{itemdecl}
template <ptrdiff_t Extent>
explicit basic_spanbuf(
  basic_span<charT, Extent> s,
  ios_base::openmode which = ios_base::in | ios_base::out);
\end{itemdecl}

\begin{itemdescr}
\pnum
\effects
Constructs an object of class
\tcode{basic_spanbuf},
initializing the base class with
\tcode{basic_streambuf()}~(\ref{streambuf.cons}), and initializing
\tcode{mode}
with \tcode{which}. 
Initializes the internal pointers as if calling \tcode{span(s)}.
\end{itemdescr}

\indexlibrary{\idxcode{basic_spanbuf}!constructor}%
\begin{itemdecl}
basic_spanbuf(basic_spanbuf&& rhs) noexcept;
\end{itemdecl}

\begin{itemdescr}
\pnum
\effects Move constructs from the rvalue \tcode{rhs}. 
Both \tcode{basic_spanbuf} objects share the same underlying \tcode{span}.
The sequence pointers in 
\tcode{*this}
(\tcode{eback()}, \tcode{gptr()}, \tcode{egptr()},
\tcode{pbase()}, \tcode{pptr()}, \tcode{epptr()}) obtain
the values which \tcode{rhs} had. 
%It
%is
%\impldef{whether sequence pointers are copied by \tcode{basic_spanbuf} move
%constructor} whether the sequence pointers in \tcode{*this}
%(\tcode{eback()}, \tcode{gptr()}, \tcode{egptr()},
%\tcode{pbase()}, \tcode{pptr()}, \tcode{epptr()}) obtain
%the values which \tcode{rhs} had. Whether they do or not, \tcode{*this}
%and \tcode{rhs} reference the same \tcode{span} after the
%construction. 
The openmode, locale and any other state of \tcode{rhs} is
also copied.

\pnum
\postconditions Let \tcode{rhs_p} refer to the state of
\tcode{rhs} just prior to this construction.

\begin{itemize}
\item \tcode{span() == rhs_p.span()}
\item \tcode{eback() == rhs_p.eback()}
\item \tcode{gptr() == rhs_p.gptr()}
\item \tcode{egptr() == rhs_p.egptr()}
\item \tcode{pbase() == rhs_p.pbase()}
\item \tcode{pptr() == rhs_p.pptr()}
\item \tcode{epptr() == rhs_p.epptr()}
\end{itemize}
\end{itemdescr}


\subsection{27.x.2.2 Assign and swap [spanbuf.assign]}

%\rSec3[spanbuf.assign]{Assign and swap}
%\indexlibrarymember{operator=}{basic_spanbuf}%
\begin{itemdecl}
basic_spanbuf& operator=(basic_spanbuf&& rhs) noexcept;
\end{itemdecl}

\begin{itemdescr}
\pnum
\effects After the move assignment \tcode{*this} has the observable state it would
have had if it had been move constructed from \tcode{rhs} (see~\ref{spanbuf.cons}).

\pnum
\returns \tcode{*this}.
\end{itemdescr}

%\indexlibrarymember{swap}{basic_spanbuf}%
\begin{itemdecl}
void swap(basic_spanbuf& rhs) noexcept;
\end{itemdecl}

\begin{itemdescr}
\pnum
\effects Exchanges the state of \tcode{*this}
and \tcode{rhs}.
\end{itemdescr}

%\indexlibrarymember{swap}{basic_spanbuf}%
\begin{itemdecl}
template <class charT, class traits, class Allocator>
  void swap(basic_spanbuf<charT, traits>& x,
            basic_spanbuf<charT, traits>& y) noexcept;
\end{itemdecl}

\begin{itemdescr}
\pnum
\effects As if by \tcode{x.swap(y)}.
\end{itemdescr}


\subsection{27.x.2.3 Member functions [spanbuf.members]}
%\rSec3[spanbuf.members]{Member functions}

%\indexlibrarymember{span}{basic_spanbuf}%
\begin{itemdecl}
span<charT> span() const;
\end{itemdecl}

\begin{itemdescr}
\pnum
\returns
A
\tcode{span}
object representing the  
\tcode{basic_spanbuf}
underlying character sequence.
If the \tcode{basic_spanbuf} was created only in output mode, the resultant
\tcode{span} represents the character sequence in the range
\range{pbase()}{pptr()}, otherwise in the range
\range{eback()}{egptr()}. 
\begin{note}
In constrast to \tcode{basic_stringbuf} the underlying sequence can never grow and will not be owned. An owning copy can be obtained by converting the result to \tcode{basic_string<charT>}.
\end{note}


\end{itemdescr}

%\indexlibrarymember{str}{basic_spanbuf}%
\begin{itemdecl}
template<ptrdiff_t Extent>
void span(span<charT,Extent> s);
\end{itemdecl}

\begin{itemdescr}
\pnum
\effects
Initializes the \tcode{basic_spanbuf} underlying character
sequence with \tcode{s} and initializes the input and output sequences according to \tcode{mode}.

\pnum
\postconditions If \tcode{mode \& ios_base::out} is true, \tcode{pbase()} points to the
first underlying character and \tcode{epptr()} \tcode{>= pbase() + s.size()} holds; in
addition, if \tcode{mode \& ios_base::ate} is true,
\tcode{pptr() == pbase() + s.size()}
holds, otherwise \tcode{pptr() == pbase()} is true. If \tcode{mode \& ios_base::in} is
true, \tcode{eback()} points to the first underlying character, and both \tcode{gptr()
== eback()} and \tcode{egptr() == eback() + s.size()} hold.

\begin{note}
Using append mode does not make sense for \tcode{span}-based streams.
\end{note}

\end{itemdescr}

\subsection{27.x.2.4 Overridden virtual functions [spanbuf.virtuals]}
%\rSec3[spanbuf.virtuals]{Overridden virtual functions}
\pnum
\begin{note}
Since the underlying buffer is of fixed size, neither \tcode{overflow}, \tcode{underflow} or \tcode{pbackfail} can provide useful behavior.
\end{note}

%\indexlibrarymember{underflow}{basic_spanbuf}%
\begin{itemdecl}
int_type underflow() override;
\end{itemdecl}

\begin{itemdescr}
\pnum
\returns
%If the input sequence has a read position available,
%returns
%\tcode{traits::to_int_type(*gptr())}.
%Otherwise, returns
\tcode{traits::eof()}.
%Any character in the underlying buffer which has been initialized is considered
%to be part of the input sequence. 
\end{itemdescr}

%\indexlibrarymember{pbackfail}{basic_spanbuf}%
\begin{itemdecl}
int_type pbackfail(int_type c = traits::eof()) override;
\end{itemdecl}

\begin{itemdescr}
\pnum
\returns
\tcode{traits::eof()}.
\end{itemdescr}

%\indexlibrarymember{overflow}{basic_spanbuf}%
\begin{itemdecl}
int_type overflow(int_type c = traits::eof()) override;
\end{itemdecl}

\begin{itemdescr}
\pnum
\returns
\tcode{traits::eof()}.

\end{itemdescr}

%\indexlibrarymember{seekoff}{basic_spanbuf}%
\begin{itemdecl}
pos_type seekoff(off_type off, ios_base::seekdir way,
                 ios_base::openmode which
                   = ios_base::in | ios_base::out) override;
\end{itemdecl}

\begin{itemdescr}
\pnum
\effects
Alters the stream position within one of the
controlled sequences, if possible, as indicated in Table~\ref{tab:iostreams.seekoff.positioning}.

%\begin{libtab2}{\tcode{seekoff} positioning}{tab:iostreams.seekoff.positioning}
%{p{2.5in}l}{Conditions}{Result}
%\tcode{(which \& ios_base\colcol{}in)}\tcode{ == ios_base::in}  &
% positions the input sequence \\ \rowsep
%\tcode{(which \& ios_base\colcol{}out)}\tcode{ == ios_base::out}  &
% positions the output sequence  \\ \rowsep
%\tcode{(which \& (ios_base::in |}\br
%\tcode{ios_base::out)) ==}\br
%\tcode{(ios_base::in) |}\br
%\tcode{ios_base::out))}\br
%and \tcode{way ==} either\br
%\tcode{ios_base::beg} or\br
%\tcode{ios_base::end}     &
% positions both the input and the output sequences  \\ \rowsep
%Otherwise &
% the positioning operation fails. \\
%\end{libtab2}

\pnum
For a sequence to be positioned, if its next pointer
(either
\tcode{gptr()}
or
\tcode{pptr()})
is a null pointer and the new offset \tcode{newoff} is nonzero, the positioning
operation fails. Otherwise, the function determines \tcode{newoff} as indicated in
Table~\ref{tab:iostreams.newoff.values}.

%\begin{libtab2}{\tcode{newoff} values}{tab:iostreams.newoff.values}
%{lp{2.0in}}{Condition}{\tcode{newoff} Value}
%\tcode{way == ios_base::beg}  &
% 0  \\ \rowsep
%\tcode{way == ios_base::cur}  &
% the next pointer minus the beginning pointer (\tcode{xnext - xbeg}). \\ \rowsep
%\tcode{way == ios_base::end}  &
% the high mark pointer minus the beginning pointer (\tcode{high_mark - xbeg}).   \\
%\end{libtab2}

\pnum
If
\tcode{(newoff + off) < 0},
or if \tcode{newoff + off} refers to an uninitialized
character outside the span (as defined in~\ref{spanbuf.members} paragraph 1),
the positioning operation fails.
Otherwise, the function assigns
\tcode{xbeg + newoff + off}
to the next pointer \tcode{xnext}.

\pnum
\returns
\tcode{pos_type(newoff)},
constructed from the resultant offset \tcode{newoff}
(of type
\tcode{off_type}),
that stores the resultant stream position, if possible.
If the positioning operation fails, or
if the constructed object cannot represent the resultant stream position,
the return value is
\tcode{pos_type(off_type(-1))}.
\end{itemdescr}

%\indexlibrarymember{seekpos}{basic_spanbuf}%
\begin{itemdecl}
pos_type seekpos(pos_type sp,
                 ios_base::openmode which
                   = ios_base::in | ios_base::out) override;
\end{itemdecl}

\begin{itemdescr}
\pnum
\effects
Equivalent to \tcode{seekoff(off_type(sp), ios_base::beg, which)}.

\pnum
\returns
\tcode{sp}
to indicate success, or
\tcode{pos_type(off_type(-1))}
to indicate failure.
\end{itemdescr}

%\indexlibrarymember{setbuf}{basic_streambuf}%
\begin{itemdecl}
basic_streambuf<charT, traits>* setbuf(charT* s, streamsize n);
\end{itemdecl}

\begin{itemdescr}
\pnum
\effects
If \tcode{s} and \tcode{n} denote a non-empty span
\tcode{this->span(span<charT>(s,n));}

\pnum
\returns
\tcode{this}.
\end{itemdescr}

%\rSec2[ispanstream]{Class template \tcode{basic_ispanstream}}
\section{27.x.3 Class template \tcode{basic_ispanstream} [ispanstream] }

%\indexlibrary{\idxcode{basic_ispanstream}}%
\begin{codeblock}
namespace std {
  template <class charT, class traits = char_traits<charT>>
  class basic_ispanstream
    : public basic_istream<charT, traits> {
  public:
    using char_type      = charT;
    using int_type       = typename traits::int_type;
    using pos_type       = typename traits::pos_type;
    using off_type       = typename traits::off_type;
    using traits_type    = traits;

    // \ref{ispanstream.cons}, constructors:
    template <ptrdiff_t Extent>
    explicit basic_ispanstream(
      span<charT, Extent> span,
      ios_base::openmode which = ios_base::in);
    basic_ispanstream(const basic_ispanstream& rhs) = delete;
    basic_ispanstream(basic_ispanstream&& rhs) noexcept;

    // \ref{ispanstream.assign}, assign and swap:
    basic_ispanstream& operator=(const basic_ispanstream& rhs) = delete;
    basic_ispanstream& operator=(basic_ispanstream&& rhs) noexcept;
    void swap(basic_ispanstream& rhs) noexcept;

    // \ref{ispanstream.members}, members:
    basic_spanbuf<charT, traits>* rdbuf() const noexcept;

    span<charT> span() const noexcept;
	template<ptrdiff_t Extent>
    void span(span<charT> s) noexcept;
  private:
    basic_spanbuf<charT, traits> sb; // \expos
  };

  template <class charT, class traits>
    void swap(basic_ispanstream<charT, traits>& x,
              basic_ispanstream<charT, traits>& y) noexcept;
}
\end{codeblock}

\pnum
The class
\tcode{basic_ispanstream<charT, traits>}
supports reading objects of class
\tcode{span<charT, traits>}.
It uses a
\tcode{basic_spanbuf<charT, traits>}
object to control the associated span.
For the sake of exposition, the maintained data is presented here as:
\begin{itemize}
\item
\tcode{sb}, the \tcode{spanbuf} object.
\end{itemize}

%\rSec3[ispanstream.cons]{\tcode{basic_ispanstream} constructors}
\subsection{27.x.3.1 \tcode{basic_ispanstream} constructors [ispanstream.cons]}
\label{ispanstream.cons}

%\indexlibrary{\idxcode{basic_ispanstream}!constructor}%
\begin{itemdecl}
template <ptrdiff_t Extent>
explicit basic_ispanstream(
  span<charT, Extent> span,
  ios_base::openmode which = ios_base::in);
\end{itemdecl}

\begin{itemdescr}
\pnum
\effects
Constructs an object of class
\tcode{basic_ispanstream<charT, traits>},
initializing the base class with
\tcode{basic_istream(\&sb)}
and initializing \tcode{sb} with
\tcode{basic_spanbuf<charT, traits>{span, which | ios_base::in})}~(\ref{spanbuf.cons}).
\end{itemdescr}

%\indexlibrary{\idxcode{basic_ispanstream}!constructor}%
\begin{itemdecl}
basic_ispanstream(basic_ispanstream&& rhs);
\end{itemdecl}

\begin{itemdescr}
\pnum
\effects Move constructs from the rvalue \tcode{rhs}. This
is accomplished by move constructing the base class, and the contained
\tcode{basic_spanbuf}.
Next \tcode{basic_istream<charT, traits>::set_rdbuf(\&sb)} is called to
install the contained \tcode{basic_spanbuf}.
\end{itemdescr}

%\rSec3[ispanstream.assign]{Assign and swap}
\subsection{27.x.3.2 Assign and swap [ispanstream.assign]}
\label{ispanstream.assign}

%\indexlibrarymember{operator=}{basic_ispanstream}%
\begin{itemdecl}
basic_ispanstream& operator=(basic_ispanstream&& rhs);
\end{itemdecl}

\begin{itemdescr}
\pnum
\effects Move assigns the base and members of \tcode{*this} from the base and corresponding
members of \tcode{rhs}.

\pnum
\returns \tcode{*this}.
\end{itemdescr}

%\indexlibrarymember{swap}{basic_ispanstream}%
\begin{itemdecl}
void swap(basic_ispanstream& rhs);
\end{itemdecl}

\begin{itemdescr}
\pnum
\effects Exchanges the state of \tcode{*this} and
\tcode{rhs} by calling
\tcode{basic_istream<charT, traits>::swap(rhs)} and
\tcode{sb.swap(rhs.sb)}.
\end{itemdescr}


%\indexlibrarymember{swap}{basic_ispanstream}%
\begin{itemdecl}
template <class charT, class traits, class Allocator>
  void swap(basic_ispanstream<charT, traits, Allocator>& x,
            basic_ispanstream<charT, traits, Allocator>& y);
\end{itemdecl}

\begin{itemdescr}
\pnum
\effects As if by \tcode{x.swap(y)}.
\end{itemdescr}

%\rSec3[ispanstream.members]{Member functions}
\subsection{27.x.3.3 Member functions [ispanstream.members]}
\label{ispanstream.members}

%\indexlibrarymember{rdbuf}{basic_ispanstream}%
\begin{itemdecl}
basic_spanbuf<charT>* rdbuf() const noexcept;
\end{itemdecl}

\begin{itemdescr}
\pnum
\returns
\tcode{const_cast<basic_spanbuf<charT>*>(\&sb)}.
\end{itemdescr}

%\indexlibrarymember{str}{basic_ispanstream}%
\begin{itemdecl}
span<charT> span() const noexcept;
\end{itemdecl}

\begin{itemdescr}
\pnum
\returns
\tcode{rdbuf()->span()}.
\end{itemdescr}

%\indexlibrarymember{str}{basic_ispanstream}%
\begin{itemdecl}
template<ptrdiff_t Extent>
void span(span<charT, Extent> s) noexcept;
\end{itemdecl}

\begin{itemdescr}
\pnum
\effects
Calls
\tcode{rdbuf()->span(s)}.
\end{itemdescr}

%\rSec2[ospanstream]{Class template \tcode{basic_ospanstream}}
\section{27.x.4 Class template \tcode{basic_ospanstream} [ospanstream] }

%\indexlibrary{\idxcode{basic_ospanstream}}%
\begin{codeblock}
namespace std {
  template <class charT, class traits = char_traits<charT>>
  class basic_ospanstream
    : public basic_istream<charT, traits> {
  public:
    using char_type      = charT;
    using int_type       = typename traits::int_type;
    using pos_type       = typename traits::pos_type;
    using off_type       = typename traits::off_type;
    using traits_type    = traits;

    // \ref{ospanstream.cons}, constructors:
    template <ptrdiff_t Extent>
    explicit basic_ospanstream(
      span<charT, Extent> span,
      ios_base::openmode which = ios_base::out);
    basic_ospanstream(const basic_ospanstream& rhs) = delete;
    basic_ospanstream(basic_ospanstream&& rhs) noexcept;

    // \ref{ospanstream.assign}, assign and swap:
    basic_ospanstream& operator=(const basic_ospanstream& rhs) = delete;
    basic_ospanstream& operator=(basic_ospanstream&& rhs) noexcept;
    void swap(basic_ospanstream& rhs) noexcept;

    // \ref{ospanstream.members}, members:
    basic_spanbuf<charT, traits>* rdbuf() const noexcept;

    span<charT> span() const noexcept;
	template<ptrdiff_t Extent>
    void span(span<charT> s) noexcept;
  private:
    basic_spanbuf<charT, traits> sb; // \expos
  };

  template <class charT, class traits>
    void swap(basic_ospanstream<charT, traits>& x,
              basic_ospanstream<charT, traits>& y) noexcept;
}
\end{codeblock}

\pnum
The class
\tcode{basic_ospanstream<charT, traits>}
supports writing to objects of class
\tcode{span<charT, traits>}.
It uses a
\tcode{basic_spanbuf<charT, traits>}
object to control the associated span.
For the sake of exposition, the maintained data is presented here as:
\begin{itemize}
\item
\tcode{sb}, the \tcode{spanbuf} object.
\end{itemize}

%\rSec3[ospanstream.cons]{\tcode{basic_ospanstream} constructors}
\subsection{27.x.4.1 \tcode{basic_ospanstream} constructors [ospanstream.cons]}
\label{ospanstream.cons}

%\indexlibrary{\idxcode{basic_ospanstream}!constructor}%
\begin{itemdecl}
template <ptrdiff_t Extent>
explicit basic_ospanstream(
  span<charT, Extent> span,
  ios_base::openmode which = ios_base::out);
\end{itemdecl}

\begin{itemdescr}
\pnum
\effects
Constructs an object of class
\tcode{basic_ospanstream<charT, traits>},
initializing the base class with
\tcode{basic_istream(\&sb)}
and initializing \tcode{sb} with
\tcode{basic_spanbuf<charT, traits>{span, which | ios_base::out})}~(\ref{spanbuf.cons}).
\end{itemdescr}

%\indexlibrary{\idxcode{basic_ospanstream}!constructor}%
\begin{itemdecl}
basic_ospanstream(basic_ospanstream&& rhs) noexcept;
\end{itemdecl}

\begin{itemdescr}
\pnum
\effects Move constructs from the rvalue \tcode{rhs}. This
is accomplished by move constructing the base class, and the contained
\tcode{basic_spanbuf}.
Next \tcode{basic_ostream<charT, traits>::set_rdbuf(\&sb)} is called to
install the contained \tcode{basic_spanbuf}.
\end{itemdescr}

%\rSec3[ospanstream.assign]{Assign and swap}
\subsection{27.x.4.2 Assign and swap [ospanstream.assign]}
\label{ospanstream.assign}

%\indexlibrarymember{operator=}{basic_ospanstream}%
\begin{itemdecl}
basic_ospanstream& operator=(basic_ospanstream&& rhs) noexcept;
\end{itemdecl}

\begin{itemdescr}
\pnum
\effects Move assigns the base and members of \tcode{*this} from the base and corresponding
members of \tcode{rhs}.

\pnum
\returns \tcode{*this}.
\end{itemdescr}

%\indexlibrarymember{swap}{basic_ospanstream}%
\begin{itemdecl}
void swap(basic_ospanstream& rhs) noexcept;
\end{itemdecl}

\begin{itemdescr}
\pnum
\effects Exchanges the state of \tcode{*this} and
\tcode{rhs} by calling
\tcode{basic_istream<charT, traits>::swap(rhs)} and
\tcode{sb.swap(rhs.sb)}.
\end{itemdescr}


%\indexlibrarymember{swap}{basic_ospanstream}%
\begin{itemdecl}
template <class charT, class traits, class Allocator>
  void swap(basic_ospanstream<charT, traits, Allocator>& x,
            basic_ospanstream<charT, traits, Allocator>& y) noexcept;
\end{itemdecl}

\begin{itemdescr}
\pnum
\effects As if by \tcode{x.swap(y)}.
\end{itemdescr}

%\rSec3[ospanstream.members]{Member functions}
\subsection{27.x.4.3 Member functions [ospanstream.members]}
\label{ospanstream.members}

%\indexlibrarymember{rdbuf}{basic_ospanstream}%
\begin{itemdecl}
basic_spanbuf<charT>* rdbuf() const noexcept;
\end{itemdecl}

\begin{itemdescr}
\pnum
\returns
\tcode{const_cast<basic_spanbuf<charT>*>(\&sb)}.
\end{itemdescr}

%\indexlibrarymember{str}{basic_ospanstream}%
\begin{itemdecl}
span<charT> span() const noexcept;
\end{itemdecl}

\begin{itemdescr}
\pnum
\returns
\tcode{rdbuf()->span()}.
\end{itemdescr}

%\indexlibrarymember{str}{basic_ospanstream}%
\begin{itemdecl}
template<ptrdiff_t Extent>
void span(span<charT, Extent> s) noexcept;
\end{itemdecl}

\begin{itemdescr}
\pnum
\effects
Calls
\tcode{rdbuf()->span(s)}.
\end{itemdescr}

%\rSec2[spanstream]{Class template \tcode{basic_spanstream}}
\section{27.x.5 Class template \tcode{basic_spanstream} [spanstream] }

%\indexlibrary{\idxcode{basic_spanstream}}%
\begin{codeblock}
namespace std {
  template <class charT, class traits = char_traits<charT>>
  class basic_spanstream
    : public basic_istream<charT, traits> {
  public:
    using char_type      = charT;
    using int_type       = typename traits::int_type;
    using pos_type       = typename traits::pos_type;
    using off_type       = typename traits::off_type;
    using traits_type    = traits;

    // \ref{spanstream.cons}, constructors:
    template <ptrdiff_t Extent>
    explicit basic_spanstream(
      span<charT, Extent> span,
      ios_base::openmode which = ios_base::out);
    basic_spanstream(const basic_spanstream& rhs) = delete;
    basic_spanstream(basic_spanstream&& rhs) noexcept;

    // \ref{spanstream.assign}, assign and swap:
    basic_spanstream& operator=(const basic_spanstream& rhs) = delete;
    basic_spanstream& operator=(basic_spanstream&& rhs) noexcept;
    void swap(basic_spanstream& rhs) noexcept;

    // \ref{spanstream.members}, members:
    basic_spanbuf<charT, traits>* rdbuf() const noexcept;

    span<charT> span() const noexcept;
	template<ptrdiff_t Extent>
    void span(span<charT> s) noexcept;
  private:
    basic_spanbuf<charT, traits> sb; // \expos
  };

  template <class charT, class traits>
    void swap(basic_spanstream<charT, traits>& x,
              basic_spanstream<charT, traits>& y) noexcept;
}
\end{codeblock}

\pnum
The class
\tcode{basic_spanstream<charT, traits>}
supports reading from and writing to objects of class
\tcode{span<charT, traits>}.
It uses a
\tcode{basic_spanbuf<charT, traits>}
object to control the associated span.
For the sake of exposition, the maintained data is presented here as:
\begin{itemize}
\item
\tcode{sb}, the \tcode{spanbuf} object.
\end{itemize}

%\rSec3[spanstream.cons]{\tcode{basic_spanstream} constructors}
\subsection{27.x.5.1 \tcode{basic_spanstream} constructors [spanstream.cons]}
\label{spanstream.cons}

%\indexlibrary{\idxcode{basic_spanstream}!constructor}%
\begin{itemdecl}
template <ptrdiff_t Extent>
explicit basic_spanstream(
  span<charT, Extent> span,
  ios_base::openmode which = ios_base::out | ios_bas::in);
\end{itemdecl}

\begin{itemdescr}
\pnum
\effects
Constructs an object of class
\tcode{basic_spanstream<charT, traits>},
initializing the base class with
\tcode{basic_istream(\&sb)}
and initializing \tcode{sb} with
\tcode{basic_spanbuf<charT, traits>{span, which})}~(\ref{spanbuf.cons}).
\end{itemdescr}

%\indexlibrary{\idxcode{basic_spanstream}!constructor}%
\begin{itemdecl}
basic_spanstream(basic_spanstream&& rhs) noexcept;
\end{itemdecl}

\begin{itemdescr}
\pnum
\effects Move constructs from the rvalue \tcode{rhs}. This
is accomplished by move constructing the base class, and the contained
\tcode{basic_spanbuf}.
Next \tcode{basic_ostream<charT, traits>::set_rdbuf(\&sb)} is called to
install the contained \tcode{basic_spanbuf}.
\end{itemdescr}

%\rSec3[spanstream.assign]{Assign and swap}
\subsection{27.x.5.2 Assign and swap [spanstream.assign]}
\label{spanstream.assign}

%\indexlibrarymember{operator=}{basic_spanstream}%
\begin{itemdecl}
basic_spanstream& operator=(basic_spanstream&& rhs) noexcept;
\end{itemdecl}

\begin{itemdescr}
\pnum
\effects Move assigns the base and members of \tcode{*this} from the base and corresponding
members of \tcode{rhs}.

\pnum
\returns \tcode{*this}.
\end{itemdescr}

%\indexlibrarymember{swap}{basic_spanstream}%
\begin{itemdecl}
void swap(basic_spanstream& rhs) noexcept;
\end{itemdecl}

\begin{itemdescr}
\pnum
\effects Exchanges the state of \tcode{*this} and
\tcode{rhs} by calling
\tcode{basic_istream<charT, traits>::swap(rhs)} and
\tcode{sb.swap(rhs.sb)}.
\end{itemdescr}


%\indexlibrarymember{swap}{basic_spanstream}%
\begin{itemdecl}
template <class charT, class traits, class Allocator>
  void swap(basic_spanstream<charT, traits, Allocator>& x,
            basic_spanstream<charT, traits, Allocator>& y) noexcept;
\end{itemdecl}

\begin{itemdescr}
\pnum
\effects As if by \tcode{x.swap(y)}.
\end{itemdescr}

%\rSec3[spanstream.members]{Member functions}
\subsection{27.x.5.3 Member functions [spanstream.members]}
\label{spanstream.members}

%\indexlibrarymember{rdbuf}{basic_spanstream}%
\begin{itemdecl}
basic_spanbuf<charT>* rdbuf() const noexcept;
\end{itemdecl}

\begin{itemdescr}
\pnum
\returns
\tcode{const_cast<basic_spanbuf<charT>*>(\&sb)}.
\end{itemdescr}

%\indexlibrarymember{str}{basic_spanstream}%
\begin{itemdecl}
span<charT> span() const noexcept;
\end{itemdecl}

\begin{itemdescr}
\pnum
\returns
\tcode{rdbuf()->span()}.
\end{itemdescr}

%\indexlibrarymember{str}{basic_spanstream}%
\begin{itemdecl}
template<ptrdiff_t Extent>
void span(span<charT, Extent> s) noexcept;
\end{itemdecl}

\begin{itemdescr}
\pnum
\effects
Calls
\tcode{rdbuf()->span(s)}.
\end{itemdescr}



\chapter{Appendix: Example Implementations}
An example implementation will become available under the author's github account at:
\url{https://github.com/PeterSommerlad/SC22WG21_Papers/tree/master/workspace/Test_basic_spanbuf}
\end{document}

