\documentclass[ebook,11pt,article]{memoir}
\usepackage{geometry}  % See geometry.pdf to learn the layout options. There are lots.
\geometry{a4paper}  % ... or a4paper or a5paper or ... 
%\geometry{landscape}  % Activate for for rotated page geometry
%%% from std.tex
%\usepackage[american]
%           {babel}        % needed for iso dates
%\usepackage[iso,american]
%           {isodate}      % use iso format for dates
\usepackage[final]
           {listings}     % code listings
%\usepackage{longtable}    % auto-breaking tables
%\usepackage{ltcaption}    % fix captions for long tables
\usepackage{relsize}      % provide relative font size changes
%\usepackage{textcomp}     % provide \text{l,r}angle
\usepackage{underscore}   % remove special status of '_' in ordinary text
%\usepackage{parskip}      % handle non-indented paragraphs "properly"
%\usepackage{array}        % new column definitions for tables
\usepackage[normalem]{ulem}
\usepackage{enumitem}
\usepackage{color}        % define colors for strikeouts and underlines
%\usepackage{amsmath}      % additional math symbols
%\usepackage{mathrsfs}     % mathscr font
\usepackage[final]{microtype}
%\usepackage{multicol}
\usepackage{xspace}
%\usepackage{lmodern}
\usepackage[T1]{fontenc} % makes tilde work! and is better for umlauts etc.
%\usepackage[pdftex, final]{graphicx}
\usepackage[pdftex,
%            pdftitle={C++ International Standard},
%            pdfsubject={C++ International Standard},
%            pdfcreator={Richard Smith},
            bookmarks=true,
            bookmarksnumbered=true,
            pdfpagelabels=true,
            pdfpagemode=UseOutlines,
            pdfstartview=FitH,
            linktocpage=true,
            colorlinks=true,
            linkcolor=blue,
            plainpages=false
           ]{hyperref}
%\usepackage{memhfixc}     % fix interactions between hyperref and memoir
%\usepackage[active,header=false,handles=false,copydocumentclass=false,generate=std-gram.ext,extract-cmdline={gramSec},extract-env={bnftab,simplebnf,bnf,bnfkeywordtab}]{extract} % Grammar extraction
%
\renewcommand\RSsmallest{5.5pt}  % smallest font size for relsize


%%%% reuse all three from std.tex:
\input{macros}
\input{layout}
\input{styles}
\newcommand{\iref}[1]{[#1]}
\pagestyle{myheadings}

\newcommand{\papernumber}{p0407r2}
\newcommand{\paperdate}{2017-11-16}

\markboth{\papernumber{} \paperdate{}}{\papernumber{} \paperdate{}}

\title{\papernumber{} - Allocator-aware basic\_stringbuf}
\author{Peter Sommerlad}
\date{\paperdate}                        % Activate to display a given date or no date
\setsecnumdepth{subsection}

\begin{document}
\maketitle
\begin{tabular}[t]{|l|l|}\hline 
Document Number: \papernumber &   (referring to n3172 and LWG issue 2429)\\\hline
Date: & \paperdate \\\hline
Project: & Programming Language C++\\\hline 
Audience: & LWG/LEWG\\\hline
\end{tabular}

\chapter{History}

%make the same changes to stuff introduced with p0408 ?


Streams have been the oldest part of the C++ standard library and their specification doesn't take into account many things introduced since C++11. One of  the oversights is that allocator support is only available through a template parameter but not really encouraged or allowed on a per-object basis. The second issue that there is no non-copying access to the internal buffer of a \tcode{basic_stringbuf} which makes at least the obtaining of the output results from an \tcode{ostringstream} inefficient, because a copy is always made. There is a second paper p0408 on the efficient internal buffer access that sidesteps the extra copy.
\section{n3172}
In Batavia 2010 n3172 was discussed and the issue LWG 2429 was closed with NAD but including an encouraging note that n3172 was just not enough (retrospectively this was due to the rush to get C++11 done). And there was not yet the allocator infrastructure in place that we aim for with C++17.
\section{p0407r0}
Since this paper has not yet really been discussed by LEWG or LWG this update mainly reflects the re-ordering of the standard's section numbers. It also adds access to the set allocator of \tcode{basic_stringbuf}.
\section{p0407r1}
Thanks to sweat-dripping effort by Pablo Halpern, I was able to come up with an easier way to correctly specify allocator support by introducing an addition exposition-only \tcode{basic_string} member variable to the \tcode{basic_stringbuf}. Also added an overload of the move constructor with an additional allocator parameter.

The paper p0408 (efficient buffer access) needs to be slightly adopted according to this paper if accepted before this paper (this paper passed p0408 in LEWG in Albuquerque).


\chapter{Introduction}
This paper proposes one adjustment to \tcode{basic_stringbuf} and the corresponding stream class templates to enable the actual use of allocators. It follows the direction of what \tcode{basic_string} provides and thus allows implementations who actually use \tcode{basic_string} as the internal buffer for \tcode{basic_stringbuf} to directly map the allocator to the underlying \tcode{basic_string}.

\chapter{Acknowledgements}
\begin{itemize}
\item Thanks go to Pablo Halpern who originally started this, helped with getting Allocator support simpler to specify correctly, and Daniel Kr\"ugler who pointed this out to me and told me to split the two issues into two independent papers. Also to Alisdair Meredith for feedback on earlier versions pointing out deficiencies with respect to my abilities to specify allocators.
\end{itemize}

\chapter{Motivation}
With the introduction of more useful allocator API in the recent editions of the standard including the planned C++17, it is more desirable to have the library classes that allocate and release memory to employ that infrastructure, e.g., to provide thread-specific allocation that can work without employing mutual exclusion. Unfortunately streams based on strings do not take allocator object arguments, whereas they already have the corresponding template parameter. This seems to be an easy to provide extension that almost looks overlooked by previous allocator-specific adaptations of the standard's text.
\begin{codeblock}
\end{codeblock}

\chapter{Impact on the Standard}
This is an extension to the constructor API of \tcode{basic_stringbuf}, \tcode{basic_stringstream}, \tcode{basic_istringstream}, and \tcode{basic_ostringstream} class templates to follow the constructors taking allocators from \tcode{basic_string}. Because each constructor is extended with a parameter as the last one and this parameter is provided with a default argument there should be minimal impact on existing client code. Regular usage should be completely unaffected. 

The class \tcode{basic_stringbuf} gets an additional member function to obtain a copy of the underlying allocator object, like \tcode{basic_string} provides.

\chapter{Design Decisions}
\section{General Principles}
Allocator support in the standard library is lacking for string-based streams and seems to be addable in a straightforward way, because all class templates already take it as template parameter.
\section{Open Issues to be Discussed by LEWG / LWG}
\begin{itemize}
\item Does it make sense to add \tcode{noexcept} specifications for \tcode{move()} and \tcode{swap()} members, since the base classes and other streams do not. At least it does not make sense so for stream objects, since the base classes do not specify that.
\item Are there other functions with respect to string streams that would require an allocator parameter? I do not think so, except move (from string) constructors in \href{https://wg21.link/P0408}{P0408} which require additional overloads along the lines of the constructors extended here.
\item There is some overlap with P0408, how to deal with that overlap?
\end{itemize}

\chapter{Technical Specifications}

\section{30.8.2 Adjust synopsis of \tcode{basic_stringbuf} [stringbuf]}
Change each of the non-moving, non-deleted constructors to add a const-ref \tcode{Allocator} parameter as last parameter with a default constructed \tcode{Allocator} as default argument. Add an overload for the move constructor adding an \tcode{Allocator} parameter. Add an exposition-only member variable \tcode{buf} to allow referring to it for specifying allocator behaviour. May be: Add noexcept specification, depending on allocator behavior, like with \tcode{basic_string}?
\begin{codeblock}
    explicit basic_stringbuf(
        ios_base::openmode which = ios_base::in | ios_base::out@\added{,}@
        @\added{const Allocator \&a = Allocator()}@);

    @\added{template<class SAllocator>}@
    explicit basic_stringbuf(
        const basic_string<charT, traits, @\added{S}@Allocator>& str,
        ios_base::openmode which = ios_base::in | ios_base::out@\added{,}@
        @\added{const Allocator\& a = Allocator()}@);
    @\added{template<class SAllocator>}@
    @\added{explicit basic_stringbuf(}@
        @\added{const basic_string<charT, traits, SAllocator>\& str,}@
        @\added{const Allocator\& a)}@
        @\added{: basic_stringbuf(std,ios_base::openmode(ios_base::in | ios_base::out),a)\{\}}@
    basic_stringbuf(const basic_stringbuf& rhs) = delete;
    basic_stringbuf(basic_stringbuf&& rhs);
    @\added{basic_stringbuf(basic_stringbuf\&\& rhs, const Allocator\& a);}@        

    // \ref{stringbuf.assign}, assign and swap
    basic_stringbuf& operator=(const basic_stringbuf& rhs) = delete;
    basic_stringbuf& operator=(basic_stringbuf&& rhs);
@% \added{noexcept(allocator_traits<Allocator>::propagate_on_container_move_assignment::value || allocator_traits<Allocator>::is_always_equal::value)};
@    void swap(basic_stringbuf& rhs);
\end{codeblock}
%    \added{noexcept(allocator_traits<Allocator>::propagate_on_container_swap::value ||allocator_traits<Allocator>::is_always_equal::value)};


\textit{Add the following declaration to the public section of synopsis of the class template \tcode{basic_stringbuf}:}
\begin{addedblock}
\begin{codeblock}
    allocator_type get_allocator() const noexcept;
\end{codeblock}
\end{addedblock}

\textit{Add the following exposition only member to the private section of synopsis of the class template \tcode{basic_stringbuf}. This allows to delegate all details of allocator-related behaviour on what \tcode{basic_string} is doing, simplifying this specification a lot.}
\begin{codeblock}
  private:
    ios_base::openmode mode;  // \expos
    @\added{basic_string<charT, traits, Allocator> buf; // \expos }
\end{codeblock}

\textit{May be: Add a conditional noexcept specification to swap based on Allocator's behaviour?:}
\begin{codeblock}
  template <class charT, class traits, class Allocator>
    void swap(basic_stringbuf<charT, traits, Allocator>& x,
              basic_stringbuf<charT, traits, Allocator>& y);
\end{codeblock}
%\added{noexcept(noexcept(x.swap(y)))};


\textit{Append a list item p2.2 and a paragraph p3 to the text following the synopsis:}

\pnum
The class
\tcode{basic_stringbuf}
is derived from
\tcode{basic_streambuf}
to associate possibly the input sequence and possibly
the output sequence with a sequence of arbitrary
\term{characters}.
The sequence can be initialized from, or made available as, an object of class
\tcode{basic_string}.

\begin{insrt}
\pnum
In every specialization \tcode{basic_stringbuf<charT, traits, Allocator>}, the type \\\tcode{allocator_traits<Allocator>::value_type} shall name the same type as \tcode{charT}. 
Every object of type \tcode{basic_stringbuf<charT, traits, Allocator>} shall use an object of type \tcode{Allocator} to allocate and free storage for the internal buffer of \tcode{charT} objects as needed. The \tcode{Allocator} object used shall be obtained as described in 23.2.1 [container.requirements.general].
%\begin{note}
%Implementations using \tcode{basic_string} internally, will simply pass the allocator parameter to the corresponding \tcode{basic_string} constructors.
%\end{note}
\end{insrt}

\pnum
For the sake of exposition, the maintained data is presented here as:
\begin{itemize}
\item
\tcode{ios_base::openmode mode},
has
\tcode{in}
set if the input sequence can be read, and
\tcode{out}
set if the output sequence can be written.
\item
\added{\tcode{basic}_\tcode{string<charT, traits, Allocator> buf}
holds the underlying character sequence.}
\end{itemize}


\subsection{30.8.2.1 \tcode{basic_stringbuf} constructors [stringbuf.cons]}
Adjust the constructor specifications taking the additional Allocator parameter and an overload for the move-constructor taking an Allocator:

\begin{itemdecl}
explicit basic_stringbuf(
  ios_base::openmode which = ios_base::in | ios_base::out@\added{,}@
  @\added{const Allocator \&a = Allocator()}@);
\end{itemdecl}

\begin{itemdescr}
\pnum
\effects
Constructs an object of class
\tcode{basic_stringbuf},
initializing the base class with
\tcode{basic_streambuf()}\iref{streambuf.cons}, \removed{and} initializing
\tcode{mode}
with \tcode{which}\added{, and \tcode{buf} with \tcode{a}}.

\pnum
\postconditions
\tcode{str() == ""}.
\end{itemdescr}

\begin{itemdecl}
@\added{template<class SAllocator>}@
explicit basic_stringbuf(
  const basic_string<charT, traits, @\added{S}@Allocator>& s,
  ios_base::openmode which = ios_base::in | ios_base::out@\added{,}@
  @\added{const Allocator\& a = Allocator()}@);
\end{itemdecl}

\begin{itemdescr}
\pnum
\effects
Constructs an object of class
\tcode{basic_stringbuf},
initializing the base class with
\tcode{basic_streambuf()}\iref{streambuf.cons}, \removed{and} initializing
\tcode{mode}
with \tcode{which}\added{, and initializing \tcode{buf} with \tcode{\{s,a\}}}.
\removed{Then calls \tcode{str(s)}.}
\end{itemdescr}

\begin{itemdecl}
basic_stringbuf(basic_stringbuf&& rhs);
@\added{basic_stringbuf(basic_stringbuf\&\& rhs, const Allocator\& a);}@
\end{itemdecl}

\begin{itemdescr}
\pnum
\effects Move constructs from the rvalue \tcode{rhs}. 
\added{In the first form \tcode{buf} is initialized from \tcode{\{std::move(rhs.buf)\}}. In the second form \tcode{buf} is initialized from \tcode{\{std::move(rhs.buf), a\}}.}
It
is
\impldef{whether sequence pointers are copied by \tcode{basic_stringbuf} move
constructor} whether the sequence pointers in \tcode{*this}
(\tcode{eback()}, \tcode{gptr()}, \tcode{egptr()},
\tcode{pbase()}, \tcode{pptr()}, \tcode{epptr()}) obtain
the values which \tcode{rhs} had. Whether they do or not, \tcode{*this}
and \tcode{rhs} reference separate buffers (if any at all) after the
construction. The openmode, locale and any other state of \tcode{rhs} is
also copied.

\pnum
\postconditions Let \tcode{rhs_p} refer to the state of
\tcode{rhs} just prior to this construction and let \tcode{rhs_a}
refer to the state of \tcode{rhs} just after this construction.

\begin{itemize}
\item \tcode{str() == rhs_p.str()}
\item \tcode{gptr() - eback() == rhs_p.gptr() - rhs_p.eback()}
\item \tcode{egptr() - eback() == rhs_p.egptr() - rhs_p.eback()}
\item \tcode{pptr() - pbase() == rhs_p.pptr() - rhs_p.pbase()}
\item \tcode{epptr() - pbase() == rhs_p.epptr() - rhs_p.pbase()}
\item \tcode{if (eback()) eback() != rhs_a.eback()}
\item \tcode{if (gptr()) gptr() != rhs_a.gptr()}
\item \tcode{if (egptr()) egptr() != rhs_a.egptr()}
\item \tcode{if (pbase()) pbase() != rhs_a.pbase()}
\item \tcode{if (pptr()) pptr() != rhs_a.pptr()}
\item \tcode{if (epptr()) epptr() != rhs_a.epptr()}
\end{itemize}
\end{itemdescr}

%\rSec3[stringbuf.assign]{Assign and swap}
\section{30.8.2.2 Assign and swap [stringbuf.assign]}

\indexlibrarymember{operator=}{basic_stringbuf}%
\begin{itemdecl}
basic_stringbuf& operator=(basic_stringbuf&& rhs);
\end{itemdecl}
%\added{noexcept(allocator_traits<Allocator>::propagate_on_container_move_assignment::value ||}\\
%      \added{allocator_traits<Allocator>::is_always_equal::value)}@;

\begin{itemdescr}
\pnum
\effects \added{Move assigns \tcode{buf} from \tcode{std::move(rhs.buf)}}. After the move assignment \tcode{*this} has the observable state it would
have had if it had been move constructed from \tcode{rhs} (see~\ref{stringbuf.cons}).

\pnum
\returns \tcode{*this}.
\end{itemdescr}

\indexlibrarymember{swap}{basic_stringbuf}%
\begin{itemdecl}
void swap(basic_stringbuf& rhs);
\end{itemdecl}
%@\added{noexcept(allocator_traits<Allocator>::propagate_on_container_swap::value ||}\\
%       \added{allocator_traits<Allocator>::is_always_equal::value)}@;

\begin{itemdescr}
\pnum
\effects Exchanges the state of \tcode{*this}
and \tcode{rhs}.
\end{itemdescr}

\indexlibrarymember{swap}{basic_stringbuf}%
\begin{itemdecl}
template <class charT, class traits, class Allocator>
  void swap(basic_stringbuf<charT, traits, Allocator>& x,
            basic_stringbuf<charT, traits, Allocator>& y);
\end{itemdecl}
%@\added{noexcept(noexcept(x.swap(y)))}@;

\begin{itemdescr}
\pnum
\effects As if by \tcode{x.swap(y)}.
\end{itemdescr}




\section{30.8.2.3 Member functions [stringbuf.members]}
Add the definition of the \tcode{get_allocator} function:
\begin{addedblock}
%\indexlibrarymember{get_allocator}{basic_stringbuf}%
\begin{itemdecl}
allocator_type get_allocator() const noexcept;
\end{itemdecl}

\begin{itemdescr}
\pnum
\returns \tcode{buf.get_allocator()}.

\end{itemdescr}
\end{addedblock}

%% basic_istringstream
\section{30.8.3 Adjust synopsis of basic\_istringstream [istringstream]}
Change each of the non-move, non-deleted constructors to add a const-ref \tcode{Allocator} parameter as last parameter with a default constructed \tcode{Allocator} as default argument. Allow a string with a different allocator type here as well.
\begin{codeblock}
explicit basic_istringstream(
             ios_base::openmode which = ios_base::in@\added{,}@
             @\added{const Allocator\& a = Allocator()}@);
@\added{template <class SAllocator>}@
explicit basic_istringstream(
             const basic_string<charT, traits, @\added{S}@Allocator>& str,
             ios_base::openmode which = ios_base::in@\added{,}@
             @\added{const Allocator\& a = Allocator()}@);
\end{codeblock}
\begin{addedblock}\begin{codeblock}
template <class SAllocator>
explicit basic_istringstream(
             const basic_string<charT, traits,SAllocator>& str,
             const Allocator& a)
         : basic_istringstream(str, ios_base::in, a) {}
\end{codeblock}\end{addedblock}


%\textit{Append a paragraph p2 to the text following the synopsis (Pablo thinks this is superfluous):}
%
%\begin{insrt}
%\pnum
%In every specialization \tcode{basic_istringstream<charT, traits, Allocator>}, the type \tcode{allocator_traits<Allocator>::value_type} shall name the same type as \tcode{charT}. Every object of type \tcode{basic_istringstream<charT, traits, Allocator>} shall use an object of type \tcode{Allocator} to allocate and free storage for the internal buffer of \tcode{charT} objects as needed. The \tcode{Allocator} object used shall be obtained as described in 23.2.1 [container.requirements.general].
%%\begin{note}
%%Implementations using \tcode{basic_string} internally, will simply pass the allocator parameter to the corresponding \tcode{basic_string} constructors.
%%\end{note}
%\begin{note}
%Access to the current allocator can be obtained via \tcode{rdbuf()->get_allocator()}.
%\end{note}
%\end{insrt}


\subsection{30.8.3.1 \tcode{basic_istringstream} constructors [istringstream.cons]}
Adjust the constructor specifications taking the additional Allocator parameter and adjust the delegation to basic_stringbuf constructors in the Effects clauses in p1 and p2 to pass on the given allocator object.

\begin{itemdecl}
explicit basic_istringstream(
  ios_base::openmode which = ios_base::in@\added{,}@
  @\added{const Allocator\& a = Allocator()}@);
\end{itemdecl}
\begin{itemdescr}
\pnum
\effects
Constructs an object of class
\tcode{basic_istringstream<charT, traits>},
initializing the base class with
\tcode{basic_istream(\&sb)}
and initializing \tcode{sb} with
\tcode{basic_stringbuf<charT, traits, Alloca\-tor>(which | ios_base::in\added{, a}))}~({30.8.2.1}).
\end{itemdescr}

\begin{itemdecl}
@\added{template <class SAllocator>}@
explicit basic_istringstream(
  const basic_string<charT, traits, @\added{S}@Allocator>& str,
  ios_base::openmode which = ios_base::in@\added{,}@
  @\added{const Allocator\& a = Allocator()}@);
\end{itemdecl}

\begin{itemdescr}
\pnum
\effects
Constructs an object of class
\tcode{basic_istringstream<charT, traits>},
initializing the base class with
\tcode{basic_istream(\&sb)}
and initializing \tcode{sb} with
\tcode{basic_stringbuf<charT, traits, Alloca\-tor>(str, which | ios_base::in\added{, a)})}~({30.8.2.1}).
\end{itemdescr}



%% basic_ostringstream
\section{30.8.4 Adjust synopsis of \tcode{basic_ostringstream} [ostringstream]}
Change each of the non-move, non-deleted constructors to add a const-ref \tcode{Allocator} parameter as last parameter with a default constructed \tcode{Allocator} as default argument. 
\begin{codeblock}
explicit basic_ostringstream(
             ios_base::openmode which = ios_base::out@\added{,}@
             @\added{const Allocator\& a = Allocator()}@);
@\added{template <class SAllocator>}@
explicit basic_ostringstream(
             const basic_string<charT, traits, @\added{S}@Allocator>& str,
             ios_base::openmode which = ios_base::out@\added{,}@
             @\added{const Allocator\& a = Allocator()}@);
\end{codeblock}
\begin{addedblock}\begin{codeblock}
template <class SAllocator>
explicit basic_ostringstream(
             const basic_string<charT, traits,SAllocator>& str,
             const Allocator& a)
         : basic_ostringstream(std, ios_base::out, a) {}
\end{codeblock}\end{addedblock}

%\textit{Append a paragraph p2 to the text following the synopsis:}
%
%\begin{insrt}
%\pnum
%In every specialization \tcode{basic_ostringstream<charT, traits, Allocator>}, the type \tcode{allocator_traits<Allocator>::value_type} shall name the same type as \tcode{charT}. Every object of type \tcode{basic_ostringstream<charT, traits, Allocator>} shall use an object of type \tcode{Allocator} to allocate and free storage for the internal buffer of \tcode{charT} objects as needed. The \tcode{Allocator} object used shall be obtained as described in 23.2.1 [container.requirements.general].
%%\begin{note}
%%Implementations using \tcode{basic_string} internally, will simply pass the allocator parameter to the corresponding \tcode{basic_string} constructors.
%%\end{note}
%\begin{note}
%Access to the current allocator can be obtained via \tcode{rdbuf()->get_allocator()}.
%\end{note}
%\end{insrt}


\subsection{30.8.4.1 \tcode{basic_ostringstream} constructors [ostringstream.cons]}
Adjust the constructor specifications taking the additional Allocator parameter and adjust the delegation to basic_stringbuf constructors in the Effects clauses in p1 and p2 to pass on the given allocator object.

\begin{itemdecl}
explicit basic_ostringstream(
  ios_base::openmode which = ios_base::out@\added{,}@
  @\added{const Allocator\& a = Allocator()}@);
\end{itemdecl}

\begin{itemdescr}
\pnum
\effects
Constructs an object of class
\tcode{basic_ostringstream},
initializing the base class with
\tcode{basic_ostream(\brk{}\&sb)}
and initializing \tcode{sb} with
\tcode{basic_stringbuf<charT, traits, Allocator>(which | \brk{}ios_base::out\added{, a}))}~({30.8.2.1}).
\end{itemdescr}

\begin{itemdecl}
@\added{template <class SAllocator>}@
explicit basic_ostringstream(
  const basic_string<charT, traits, @\added{S}@Allocator>& str,
  ios_base::openmode which = ios_base::out@\added{,}@
  @\added{const Allocator\& a = Allocator()}@);
\end{itemdecl}

\begin{itemdescr}
\pnum
\effects
Constructs an object of class
\tcode{basic_ostringstream<charT, traits>},
initializing the base class with
\tcode{basic_ostream(\&sb)}
and initializing \tcode{sb} with
\tcode{basic_stringbuf<charT, traits, Alloca\-tor>(str, which | ios_base::out\added{, a}))}~({30.8.2.1}).
\end{itemdescr}


%% basic_stringstream
\section{30.8.5 Adjust synopsis of \tcode{basic_stringstream} [stringstream]}
Change each of the non-move, non-deleted constructors to add a const-ref \tcode{Allocator} parameter as last parameter with a default constructed \tcode{Allocator} as default argument. 
\begin{codeblock}
explicit basic_stringstream(
             ios_base::openmode which = ios_base::out | ios_base::in@\added{,}@
             @\added{const Allocator\& a = Allocator()}@);
@\added{template <class SAllocator>}@
explicit basic_ostringstream(
             const basic_string<charT, traits, @\added{S}@Allocator>& str,
             ios_base::openmode which = ios_base::out | ios_base::in@\added{,}@
             @\added{const Allocator\& a = Allocator()}@);
\end{codeblock}
\begin{addedblock}\begin{codeblock}
template <class SAllocator>
explicit basic_stringstream(
             const basic_string<charT, traits,SAllocator>& str,
             const Allocator& a)
         : basic_stringstream(std, ios_base::openmode(ios_base::in | ios_base::out), a) {}
\end{codeblock}\end{addedblock}

%\textit{Append a paragraph p2 to the text following the synopsis:}
%
%\begin{insrt}
%\pnum
%In every specialization \tcode{basic_stringstream<charT, traits, Allocator>}, the type \tcode{allocator_traits<Allocator>::value_type} shall name the same type as \tcode{charT}. Every object of type \tcode{basic_stringstream<charT, traits, Allocator>} shall use an object of type \tcode{Allocator} to allocate and free storage for the internal buffer of \tcode{charT} objects as needed. The \tcode{Allocator} object used shall be obtained as described in 23.2.1 [container.requirements.general].
%\begin{note}
%Implementations using \tcode{basic_string} internally, will simply pass the allocator parameter to the corresponding \tcode{basic_string} constructors.
%\end{note}
%\begin{note}
%Access to the current allocator can be obtained via \tcode{rdbuf()->get_allocator()}.
%\end{note}
%\end{insrt}

\subsection{30.8.5.1 \tcode{basic_stringstream} constructors [stringstream.cons]}
Adjust the constructor specifications taking the additional Allocator parameter and adjust the delegation to basic_stringbuf constructors in the Effects clauses in p1 and p2 to pass on the given allocator object.
\begin{itemdecl}
explicit basic_stringstream(
  ios_base::openmode which = ios_base::out | ios_base::in@\added{,}@
  @\added{const Allocator\& a = Allocator()}@);
\end{itemdecl}

\begin{itemdescr}
\pnum
\effects
Constructs an object of class
\tcode{basic_stringstream<charT, traits>},
initializing the base class with
\tcode{basic_iostream(\&sb)}
and initializing
\tcode{sb}
with
\tcode{basic_stringbuf<charT, traits, Alloca\-tor>(which\added{, a})}.
\end{itemdescr}

\indexlibrary{\idxcode{basic_stringstream}!constructor}%
\begin{itemdecl}
@\added{template <class SAllocator>}@
explicit basic_stringstream(
  const basic_string<charT, traits, @\added{S}@Allocator>& str,
  ios_base::openmode which = ios_base::out | ios_base::in@\added{,}@
  @\added{const Allocator\& a = Allocator()}@);
\end{itemdecl}

\begin{itemdescr}
\pnum
\effects
Constructs an object of class
\tcode{basic_stringstream<charT, traits>},
initializing the base class with
\tcode{basic_iostream(\&sb)}
and initializing
\tcode{sb}
with
\tcode{basic_stringbuf<charT, traits, Alloca\-tor>(str, which\added{, a})}.
\end{itemdescr}



\chapter{Appendix: Example Implementations}
An implementation of the additional constructor parameter was done by the author in the \tcode{<sstream>} header of gcc 6.1. It seems trivial, since all significant relevant usage is within \tcode{basic_string}. 
\end{document}

