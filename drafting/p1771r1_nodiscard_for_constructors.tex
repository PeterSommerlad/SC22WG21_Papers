\documentclass[ebook,11pt,article]{memoir}
\usepackage{geometry}  % See geometry.pdf to learn the layout options. There are lots.
\geometry{a4paper}  % ... or a4paper or a5paper or ... 
%\geometry{landscape}  % Activate for for rotated page geometry
%%% from std.tex
%\usepackage[american]
%           {babel}        % needed for iso dates
%\usepackage[iso,american]
%           {isodate}      % use iso format for dates
\usepackage[final]
           {listings}     % code listings
%\usepackage{longtable}    % auto-breaking tables
%\usepackage{ltcaption}    % fix captions for long tables
\usepackage{relsize}      % provide relative font size changes
%\usepackage{textcomp}     % provide \text{l,r}angle
\usepackage{underscore}   % remove special status of '_' in ordinary text
%\usepackage{parskip}      % handle non-indented paragraphs "properly"
\usepackage{array}        % new column definitions for tables
\usepackage[normalem]{ulem}
\usepackage{enumitem}
\usepackage{color}        % define colors for strikeouts and underlines
%\usepackage{amsmath}      % additional math symbols
%\usepackage{mathrsfs}     % mathscr font
\usepackage[final]{microtype}
%\usepackage{multicol}
\usepackage{xspace}
%\usepackage{lmodern}
\usepackage[T1]{fontenc} % makes tilde work! and is better for umlauts etc.
%\usepackage[pdftex, final]{graphicx}
\usepackage[pdftex,
%            pdftitle={C++ International Standard},
%            pdfsubject={C++ International Standard},
%            pdfcreator={Richard Smith},
            bookmarks=true,
            bookmarksnumbered=true,
            pdfpagelabels=true,
            pdfpagemode=UseOutlines,
            pdfstartview=FitH,
            linktocpage=true,
            colorlinks=true,
            linkcolor=blue,
            plainpages=false
           ]{hyperref}
%\usepackage{memhfixc}     % fix interactions between hyperref and memoir
%\usepackage[active,header=false,handles=false,copydocumentclass=false,generate=std-gram.ext,extract-cmdline={gramSec},extract-env={bnftab,simplebnf,bnf,bnfkeywordtab}]{extract} % Grammar extraction
%
\usepackage{threeparttable} % allow for table footnotes..
\renewcommand\RSsmallest{5.5pt}  % smallest font size for relsize

\usepackage{todonotes}

%%%% reuse all three from std.tex:
\input{macros}
\input{layout}
\input{styles}

\pagestyle{myheadings}

\newcommand{\papernumber}{P1771R1}
\newcommand{\paperdate}{2019-07-19}

\markboth{\papernumber{} \paperdate{}}{\papernumber{} \paperdate{}}

\title{\papernumber{} - [[nodiscard]] for constructors}
\author{Peter Sommerlad}
\date{\paperdate}                % Activate to display a given date or no date
\setsecnumdepth{subsection}

\begin{document}
\maketitle
\begin{center}
\begin{tabular}[t]{|l|p{8cm}|}\hline 
Document Number:&  \papernumber \\\hline
Date: & \paperdate \\\hline
Project: & Programming Language C++ \newline Programming Language Vulnerabilities C++\\\hline 
Audience: & EWGI/EWG/CWG\\\hline
\end{tabular}
\end{center}


\chapter{Introduction}

The  paper p0189 that introduced the \tcode{[[nodiscard]]} attribute did not consider constructors. However, gcc for example implements the checking for constructors, even so it warns about putting \tcode{[[nodiscard]]} on a constructor definition. Here I propose to allow  \tcode{[[nodiscard]]} also on constructors (which it implicitly is allowed by the current wording) and suggest checking it for cast expressions so that we can put it on things like \tcode{scoped_lock} etc.

The need is more obvious in C++ 17 and later, where CTAD allows for fewer factory functions and thus the easy to make mistake by just typing the type and constructor arguments instead of defining a local variable.

Since this change is editorial only, it might be considered to be applied for the current working paper.

R1 of this paper extends the example to demonstrate the added cases and integrates feedback by CWG in Cologne 2019.

Reviewers, please note that a constructor declaration is a function declaration.

Thanks to EWG, CWG, Mike Miller, and Alisdair Meredith for helping with this paper and giving feedback.

\chapter{Wording}

The following changes are relative to n4820. 

Change in Table 16 ([tab:cpp.cond.ha]) in section 15.1 ([cpp.cond]) the entry \tcode{nodiscard} from \tcode{201603L} to \added{\tcode{201907L}}.

Change section [dcl.attr.nodiscard] as follows. 

\rSec2[dcl.attr.nodiscard]{Nodiscard attribute}%
%\indextext{attribute!nodiscard}

\pnum
The \grammarterm{attribute-token} \tcode{nodiscard}
may be applied to the \grammarterm{declarator-id}
in a function declaration or to the declaration of a class or enumeration.
It shall appear at most once in each \grammarterm{attribute-list} and
no \grammarterm{attribute-argument-clause} shall be present.

\pnum
\added{A \emph{nodiscard type} is a (possibly cv-qualified) class or enumeration type
marked \tcode{nodiscard} in a reachable declaration.}
A \emph{nodiscard call} is \added{either}
\begin{itemize}
\item a function call expression \added{(7.6.1.2 [expr.call])}
that calls a function 
\removed{previously} declared \tcode{nodiscard}\added{ in a reachable declaration}, or
whose return type is a 
\removed{possibly cv-qualified class or enumeration type marked \tcode{nodiscard}}
\added{nodiscard type, or}\removed{.}
\item
\added{an explicit type conversion (7.6.1.8 [expr.static.cast], 7.6.3 [expr.cast], 7.6.1.3 [expr.type.conv])
that constructs an object through a constructor declared \tcode{nodiscard}, or
that initializes an object of a nodiscard type.}
\end{itemize} 

\pnum
\begin{note}
Appearance of a nodiscard call as
a potentially-evaluated discarded-value expression (7.2)%\iref{expr.prop}
is discouraged unless explicitly cast to \tcode{void}.
Implementations should issue a warning in such cases.
This is typically because discarding the return value
of a nodiscard call has surprising consequences.
\end{note}

\pnum
\begin{example}\begin{addedblock}\begin{codeblock}
struct [[nodiscard]] my_scopeguard { @\commentellip@ };
struct my_unique {
  my_unique() = default; // does not acquire resource
  [[nodiscard]] my_unique(int fd) { @\commentellip@ } // acquires resource
  ~my_unique() noexcept { @\commentellip@ } // releases resource, if any
  @\commentellip@
};
\end{codeblock}\end{addedblock}
\begin{codeblock}
struct [[nodiscard]] error_info { @\commentellip@ };
error_info enable_missile_safety_mode();
void launch_missiles();
void test_missiles() {
  @\added{my_scopeguard();}@              // \added{warning encouraged}
  @\added{void(my_scopeguard()),}@        // \added{warning not encouraged, cast to void}
    @\added{launch_missiles();}@          // \added{comma operator, statement continues}
  @\added{my_unique(42);}@                // \added{warning encouraged}
  @\added{my_unique();}@                  // \added{warning not encouraged}
  enable_missile_safety_mode(); // warning encouraged
  launch_missiles();
}
error_info &foo();
void f() { foo(); }             // warning not encouraged: not a nodiscard call, because neither
                                // the (reference) return type nor the function is declared nodiscard
\end{codeblock}
\end{example}


\end{document}

