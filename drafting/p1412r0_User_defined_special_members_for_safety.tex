\documentclass[ebook,11pt,article]{memoir}
\usepackage{geometry}  % See geometry.pdf to learn the layout options. There are lots.
\geometry{a4paper}  % ... or a4paper or a5paper or ... 
%\geometry{landscape}  % Activate for for rotated page geometry
%%% from std.tex
%\usepackage[american]
%           {babel}        % needed for iso dates
%\usepackage[iso,american]
%           {isodate}      % use iso format for dates
\usepackage[final]
           {listings}     % code listings
%\usepackage{longtable}    % auto-breaking tables
%\usepackage{ltcaption}    % fix captions for long tables
\usepackage{relsize}      % provide relative font size changes
%\usepackage{textcomp}     % provide \text{l,r}angle
\usepackage{underscore}   % remove special status of '_' in ordinary text
%\usepackage{parskip}      % handle non-indented paragraphs "properly"
%\usepackage{array}        % new column definitions for tables
\usepackage[normalem]{ulem}
\usepackage{enumitem}
\usepackage{color}        % define colors for strikeouts and underlines
%\usepackage{amsmath}      % additional math symbols
%\usepackage{mathrsfs}     % mathscr font
\usepackage[final]{microtype}
%\usepackage{multicol}
\usepackage{xspace}
%\usepackage{lmodern}
\usepackage[T1]{fontenc} % makes tilde work! and is better for umlauts etc.
%\usepackage[pdftex, final]{graphicx}
\usepackage[pdftex,
%            pdftitle={C++ International Standard},
%            pdfsubject={C++ International Standard},
%            pdfcreator={Richard Smith},
            bookmarks=true,
            bookmarksnumbered=true,
            pdfpagelabels=true,
            pdfpagemode=UseOutlines,
            pdfstartview=FitH,
            linktocpage=true,
            colorlinks=true,
            linkcolor=blue,
            plainpages=false
           ]{hyperref}
%\usepackage{memhfixc}     % fix interactions between hyperref and memoir
%\usepackage[active,header=false,handles=false,copydocumentclass=false,generate=std-gram.ext,extract-cmdline={gramSec},extract-env={bnftab,simplebnf,bnf,bnfkeywordtab}]{extract} % Grammar extraction
%
\usepackage{threeparttable} % allow for table footnotes..
\renewcommand\RSsmallest{5.5pt}  % smallest font size for relsize


%%%% reuse all three from std.tex:
\input{macros}
\input{layout}
\input{styles}

\pagestyle{myheadings}

\newcommand{\papernumber}{p1412}
\newcommand{\paperdate}{2019-01-21}

\markboth{\papernumber{} \paperdate{}}{\papernumber{} \paperdate{}}

\title{\papernumber{} - On user-declared and user-defined special member functions in safety-critical code}
\author{Peter Sommerlad}
\date{\paperdate}                % Activate to display a given date or no date
\setsecnumdepth{subsection}

\begin{document}
\maketitle
\begin{center}
\begin{tabular}[t]{|l|l|}\hline 
Document Number:&  \papernumber \\\hline
Date: & \paperdate \\\hline
Project: & Programming Language C++ / Programming Language Vulnerabilities C++\\\hline 
Audience: & SG12 / ISO SC22 WG23, SG20, Misra C++\\\hline
\end{tabular}
\end{center}
\chapter{Introduction}

C++ is a complex language providing a lot of flexibility in its use. However, not all programs are valid or behave as a developer thinks. Because of its deterministic resource management C++ is used in many systems that have high quality and safety-related requirements. While high software quality does not make a system safe or fault-tolerant, it can be an enabling factor to achieve them. Therefore, in safety-critical environments adhering to programming guidelines restricting the language use to avoid safety risks, especially undefined behavior, are one of the required practices.

Over the decades of C++ evolution, best practices and what is considered good style changed a lot. Typical systems built with the language often outlive the style they were written in. With the introduction of a three years release cycle and the frequent innovation and updates it becomes hard to keep up. Some style guides try to achieve that by taking a looking class and even proposing to use features of the language or library that are not yet standardized or even implemented (\cite{CoreGuidelines}). Other relevant guidelines that are industry standards stem from a past and provide a style that is considered no longer recommendable.

One area, where classic and more modern C++ versions differ are the recommended practices of declaring and defining special member functions in a user-defined class type. The introduction of move operations with C++11 and the guaranteed copy-elision of C++17 are game changers to the rules viable for C++03, and the world has not become simpler. This paper explains the situation by repeating Howard Hinnant's classical special member functions table and suggest a set of special member function declaration/definition combinations the author recommends. The underlying philosophy, such as categorizing class types into values, managing types, and referring types, is explained below together with other attributes the author believes are important for modern high-quality C++ code.

%Type system.

Promote value types/regular types.

UB


level of developer expertise

context of audience.

\chapter{history}

rule of zero and related



Properties to be discussed. 

Potential dangers.

\chapter{Forces for Safety in Source Code}
As a pattern (book) author I would like to introduce so-called "forces" that are used in a pattern's problem description to denote design constraints that influence the pattern's solution. Often such forces are not absolute and a pattern make conscious trade-offs. That is also a reason, why often conflicting patterns for a problem exist that resolve to different solutions.

Here I collect forces that in my observation have influenced existing programming guidelines.
\begin{itemize}
\item Simplicity
\item Familiarity
\item Code Evolution (aka Maintenance)
\end{itemize}

\chapter{Howard Hinnant's special member function overview}
\begin{table}[htp]
\caption{Howard Hinnant's special member functions table}
\begin{center}
\begin{threeparttable}
\begin{tabular}{|c||c|c|c|c|c|c||c|}
 &\multicolumn{6}{c}{What the compiler provides for class X}& \\
 user\newline{}declares   & {\tcode{X()}} & {\tcode{\~X()}} & {\tcode{X(X const\&)}} & {\tcode{=(X const\&)}} & {\tcode{X(X \&\&)}} & {\tcode{=(X \&\&)}} &   OK? \\
\hline
 nothing & \tcode{=default} & \tcode{=default} & \tcode{=default} & \tcode{=default} & \tcode{=default} & \tcode{=default} & OK \\
\hline
\tcode{X(T)} & not decl& \tcode{=default} & \tcode{=default} & \tcode{=default} & \tcode{=default} & \tcode{=default} & OK \\
\hline
\tcode{X()} & \textit{declared} & \tcode{=default} & \tcode{=default} & \tcode{=default} & \tcode{=default} & \tcode{=default} & (OK) \\
\hline
\tcode{\~X()} & \tcode{=default} & \textit{declared} & \color{red}\tcode{=default} & \color{red}\tcode{=default} & not decl& not decl& \color{red}\textbf{BAD} \\
\hline
\tcode{X(X const\&)} & not decl& \tcode{=default} & \textit{declared} & \color{red}\tcode{=default} & not decl& not decl& \color{red}\textbf{BAD} \\
\hline
\tcode{=(X const\&)} & \tcode{=default} & \tcode{=default} & \color{red}\tcode{=default} & \textit{declared} & not decl& not decl& \color{red}\textbf{BAD} \\
\hline
\tcode{X(X\&\&)} & not decl& \tcode{=default} & \tcode{=delete} &  \tcode{=delete} & \textit{declared} & not decl& \color{red}\textbf{BAD} \\
\hline
\tcode{=(X\&\&)} & \tcode{=default} & \tcode{=default} & \tcode{=delete} &  \tcode{=delete} & not decl& \textit{declared} & {(BAD)} \\
\hline
\end{tabular}
\end{threeparttable}
\end{center}
\label{default}
\end{table}%

\begin{table}[htp]
\caption{Safe and Sane combinations of Special Member Functions (TODO)}
\begin{center}
\begin{tabular}{|c||c|c|c|c|c|c||c|}
 &\multicolumn{6}{c}{declared or provided}& \\
type category & {\tcode{X()}} & {\tcode{\~X()}} & {\tcode{X(X const\&)}} & {\tcode{=(X const\&)}} & {\tcode{X(X \&\&)}} & {\tcode{=(X \&\&)}}  \\
\hline
value/aggregate & \tcode{=default} & \tcode{=default} & \tcode{=default} & \tcode{=default} & \tcode{=default} & \tcode{=default} & OK \\
\hline
value & not decl& \tcode{=default} & \tcode{=default} & \tcode{=default} & \tcode{=default} & \tcode{=default} & OK \\
\hline
\tcode{X()} & \textit{declared} & \tcode{=default} & \tcode{=default} & \tcode{=default} & \tcode{=default} & \tcode{=default} & (OK) \\
\hline
OO & \tcode{=default} & \textit{declared} & \color{red}\tcode{=default} & \color{red}\tcode{=default} & not decl& not decl& \color{red}\textbf{BAD} \\
\hline
\tcode{X(X const\&)} & not decl& \tcode{=default} & \textit{declared} & \color{red}\tcode{=default} & not decl& not decl& \color{red}\textbf{BAD} \\
\hline
\tcode{=(X const\&)} & \tcode{=default} & \tcode{=default} & \color{red}\tcode{=default} & \textit{declared} & not decl& not decl& \color{red}\textbf{BAD} \\
\hline
\tcode{X(X\&\&)} & not decl& \tcode{=default} & \tcode{=delete} &  \tcode{=delete} & \textit{declared} & not decl& \color{red}\textbf{BAD} \\
\hline
\tcode{=(X\&\&)} & \tcode{=default} & \tcode{=default} & \tcode{=delete} &  \tcode{=delete} & not decl& \textit{declared} & {(BAD)} \\
\hline
\end{tabular}
\end{center}
\label{default}
\end{table}%



\section{Items to be discussed}
Things I am unsure
\begin{itemize}
\item Are there further useful and safe exceptions?
\end{itemize}


\chapter{Categories of safe user-defined classes}
\section{Plain Value Types}
\section{Monomorphic Object Types (better name) - Encapsulation Types}
\section{Polymorphic Object Types - Class Hierarchies}
\section{Resource Managing Types}

%%%%%%

\chapter{Bibliography}
Core Guidelines

MISRA

Rule of Zero

Howard's table
\end{document}

