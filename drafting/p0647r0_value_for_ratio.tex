\documentclass[ebook,11pt,article]{memoir}
\usepackage{geometry}  % See geometry.pdf to learn the layout options. There are lots.
\geometry{a4paper}  % ... or a4paper or a5paper or ... 
%\geometry{landscape}  % Activate for for rotated page geometry
%\usepackage[parfill]{parskip}  % Activate to begin paragraphs with an empty line rather than an indent

%\usepackage[final]
%           {listings}     % code listings
%\usepackage{color}        % define colors for strikeouts and underlines
%\usepackage{underscore}   % remove special status of '_' in ordinary text
%\usepackage{xspace}
%\usepackage[normalem]{ulem}
\usepackage{enumitem}
%%%%% from std.tex:
\usepackage[final]
           {listings}     % code listings
\usepackage{longtable}    % auto-breaking tables
\usepackage{ltcaption}    % fix captions for long tables
\usepackage{relsize}      % provide relative font size changes
\usepackage{textcomp}     % provide \text{l,r}angle
\usepackage{underscore}   % remove special status of '_' in ordinary text
\usepackage{parskip}      % handle non-indented paragraphs "properly"
\usepackage{array}        % new column definitions for tables
\usepackage[normalem]{ulem}
\usepackage{color}        % define colors for strikeouts and underlines
\usepackage{amsmath}      % additional math symbols
\usepackage{mathrsfs}     % mathscr font
\usepackage{microtype}
\usepackage{multicol}
\usepackage{xspace}
\usepackage{lmodern}
\usepackage[T1]{fontenc}
\usepackage[pdftex, final]{graphicx}
\input{macros}
\input{styles}
\input{layout}
%%%%%
\pagestyle{myheadings}

\newcommand{\papernumber}{p0647r0}
\newcommand{\paperdate}{2017-05-20}

%\definecolor{insertcolor}{rgb}{0,0.5,0.25}
%\newcommand{\del}[1]{\textcolor{red}{\sout{#1}}}
%\newcommand{\ins}[1]{\textcolor{insertcolor}{\underline{#1}}}
%
%\newenvironment{insrt}{\color{insertcolor}}{\color{black}}


\markboth{\papernumber{} \paperdate{}}{\papernumber{} \paperdate{}}

\title{\papernumber{} - Floating point value access for std::ratio}
\author{Peter Sommerlad}
\date{\paperdate}                        % Activate to display a given date or no date
\setsecnumdepth{subsection}

\begin{document}
\maketitle
\begin{tabular}[t]{|l|l|}\hline 
Document Number:& \papernumber  \\\hline
Date: & \paperdate \\\hline
Project: & Programming Language C++\\\hline 
Audience: & LWG/LEWG\\\hline
Target: & C++20\\\hline
\end{tabular}

\chapter{Motivation}

Preparing for standardizing units and using std::ratio for keeping track of fractions often one needs to get the quotient as a floating point number or as a number of a type underlying a quantity, e.g., a fixpoint type. Doing that manually means adding a cast before doing the division. This is tedious and it would be nice to just access the value, as one can do with std::integral_constant. I believe that omission is just a historical accident, because it was not possible to do compile-time computation with floating point values when ratio was invented. There are some options on how to access the fraction as a compile-time entity. I chose to make the value member a long double and provide a templatized explicit conversion operator for accessing the fraction.

\chapter{Acknowledgements}
\begin{itemize}
\item Authors of N2661: Howard Hinnant, Walter Brown, Jeff Garland, Marc Paterno.
\item Members of the LiaW workshop "Towards Units" at C++Now 2017: Billy Baker, Charles Wilson, Daniel Pfeifer, Dave Jenkins, Manuel Bergler, Morris Hafner, Nicolas Holthaus, Peter Bindels, Steven Watanabe, Tuan Tran.

\end{itemize}

\newpage
\chapter{Changes Proposed}
Modify section 23.26.3 by inserting floating point access to the fractional value represented.
\rSec2[ratio.ratio]{Class template \tcode{ratio}}

\indexlibrary{\idxcode{ratio}}%
\begin{codeblock}
namespace std {
  template <intmax_t N, intmax_t D = 1>
  class ratio {
  public:
    static constexpr intmax_t num;
    static constexpr intmax_t den;
    
    using type = ratio<num, den>;
\end{codeblock}
\begin{addedblock}
\begin{codeblock}
    static constexpr long double value{ static_cast<long double>(num)/den }; 
    template<typename R>   
    explicit constexpr operator R() const noexcept { 
        return static_cast<R>(num)/static_cast<R>(den);
    }
\end{codeblock}
\end{addedblock}
\begin{codeblock}
  };
}
\end{codeblock}





\end{document}
