\documentclass[ebook,11pt,article]{memoir}
\usepackage{geometry}	% See geometry.pdf to learn the layout options. There are lots.
\geometry{a4paper}	% ... or a4paper or a5paper or ... 
%\geometry{landscape}	% Activate for for rotated page geometry
%\usepackage[parfill]{parskip}	% Activate to begin paragraphs with an empty line rather than an indent


\usepackage[final]
           {listings}     % code listings
\usepackage{color}        % define colors for strikeouts and underlines
\usepackage{underscore}   % remove special status of '_' in ordinary text
\usepackage{xspace}
\pagestyle{myheadings}
\markboth{N4189 2014-10-08}{N4189 2014-10-08}

\title{N4189 - Generic Scope Guard and RAII Wrapper for the Standard Library}
\author{Peter Sommerlad and Andrew L. Sandoval}
\date{2014-10-08}                                           % Activate to display a given date or no date
%!TEX root = std.tex
% Definitions and redefinitions of special commands

%%--------------------------------------------------
%% Difference markups
\definecolor{addclr}{rgb}{0,.6,.3} %% 0,.6,.6 was to blue for my taste :-)
\definecolor{remclr}{rgb}{1,0,0}
\definecolor{noteclr}{rgb}{0,0,1}

\renewcommand{\added}[1]{\textcolor{addclr}{\uline{#1}}}
\newcommand{\removed}[1]{\textcolor{remclr}{\sout{#1}}}
\renewcommand{\changed}[2]{\removed{#1}\added{#2}}

\newcommand{\nbc}[1]{[#1]\ }
\newcommand{\addednb}[2]{\added{\nbc{#1}#2}}
\newcommand{\removednb}[2]{\removed{\nbc{#1}#2}}
\newcommand{\changednb}[3]{\removednb{#1}{#2}\added{#3}}
\newcommand{\remitem}[1]{\item\removed{#1}}

\newcommand{\ednote}[1]{\textcolor{noteclr}{[Editor's note: #1] }}
% \newcommand{\ednote}[1]{}

\newenvironment{addedblock}
{
\color{addclr}
}
{
\color{black}
}
\newenvironment{removedblock}
{
\color{remclr}
}
{
\color{black}
}

%%--------------------------------------------------
%% Grammar extraction.
\def\gramSec[#1]#2{}

\makeatletter
\newcommand{\FlushAndPrintGrammar}{%
\immediate\closeout\XTR@out%
\immediate\openout\XTR@out=std-gram-dummy.tmp%
\def\gramSec[##1]##2{\rSec1[##1]{##2}}%
\input{std-gram.ext}%
}
\makeatother

%%--------------------------------------------------
% Escaping for index entries. Replaces ! with "! throughout its argument.
\def\indexescape#1{\doindexescape#1\stopindexescape!\doneindexescape}
\def\doindexescape#1!{#1"!\doindexescape}
\def\stopindexescape#1\doneindexescape{}

%%--------------------------------------------------
%% Cross references.
\newcommand{\addxref}[1]{%
 \glossary[xrefindex]{\indexescape{#1}}{(\ref{\indexescape{#1}})}%
}

%%--------------------------------------------------
%% Sectioning macros.
% Each section has a depth, an automatically generated section
% number, a name, and a short tag.  The depth is an integer in
% the range [0,5].  (If it proves necessary, it wouldn't take much
% programming to raise the limit from 5 to something larger.)

% Set the xref label for a clause to be "Clause n", not just "n".
\makeatletter
\newcommand{\customlabel}[2]{%
\protected@write \@auxout {}{\string \newlabel {#1}{{#2}{\thepage}{#2}{#1}{}} }%
\hypertarget{#1}{}%
}
\makeatother
\newcommand{\clauselabel}[1]{\customlabel{#1}{Clause \thechapter}}
\newcommand{\annexlabel}[1]{\customlabel{#1}{Annex \thechapter}}

% The basic sectioning command.  Example:
%    \Sec1[intro.scope]{Scope}
% defines a first-level section whose name is "Scope" and whose short
% tag is intro.scope.  The square brackets are mandatory.
\def\Sec#1[#2]#3{%
\ifcase#1\let\s=\chapter\let\l=\clauselabel
      \or\let\s=\section\let\l=\label
      \or\let\s=\subsection\let\l=\label
      \or\let\s=\subsubsection\let\l=\label
      \or\let\s=\paragraph\let\l=\label
      \or\let\s=\subparagraph\let\l=\label
      \fi%
\s[#3]{#3\hfill[#2]}\l{#2}\addxref{#2}}

% A convenience feature (mostly for the convenience of the Project
% Editor, to make it easy to move around large blocks of text):
% the \rSec macro is just like the \Sec macro, except that depths
% relative to a global variable, SectionDepthBase.  So, for example,
% if SectionDepthBase is 1,
%   \rSec1[temp.arg.type]{Template type arguments}
% is equivalent to
%   \Sec2[temp.arg.type]{Template type arguments}
\newcounter{SectionDepthBase}
\newcounter{SectionDepth}

\def\rSec#1[#2]#3{%
\setcounter{SectionDepth}{#1}
\addtocounter{SectionDepth}{\value{SectionDepthBase}}
\Sec{\arabic{SectionDepth}}[#2]{#3}}

%%--------------------------------------------------
% Indexing

% locations
\newcommand{\indextext}[1]{\index[generalindex]{#1}}
\newcommand{\indexlibrary}[1]{\index[libraryindex]{#1}}
\newcommand{\indexhdr}[1]{\indextext{\idxhdr{#1}}\index[headerindex]{\idxhdr{#1}}}
\newcommand{\indexgram}[1]{\index[grammarindex]{#1}}

% Collation helper: When building an index key, replace all macro definitions
% in the key argument with a no-op for purposes of collation.
\newcommand{\nocode}[1]{#1}
\newcommand{\idxmname}[1]{\_\_#1\_\_}
\newcommand{\idxCpp}{C++}

% \indeximpldef synthesizes a collation key from the argument; that is, an
% invocation \indeximpldef{arg} emits an index entry `key@arg`, where `key`
% is derived from `arg` by replacing the folowing list of commands with their
% bare content. This allows, say, collating plain text and code.
\newcommand{\indeximpldef}[1]{%
\let\otextup\textup%
\let\textup\nocode%
\let\otcode\tcode%
\let\tcode\nocode%
\let\ogrammarterm\grammarterm%
\let\grammarterm\nocode%
\let\omname\mname%
\let\mname\idxmname%
\let\oCpp\Cpp%
\let\Cpp\idxCpp%
\let\oBreakableUnderscore\BreakableUnderscore%  See the "underscore" package.
\let\BreakableUnderscore\textunderscore%
\edef\x{#1}%
\let\tcode\otcode%
\let\grammarterm\gterm%
\let\mname\omname%
\let\Cpp\oCpp%
\let\BreakableUnderscore\oBreakableUnderscore%
\index[impldefindex]{\x@#1}%
\let\grammarterm\ogrammarterm%
\let\textup\otextup%
}

\newcommand{\indexdefn}[1]{\indextext{#1}}
\newcommand{\idxbfpage}[1]{\textbf{\hyperpage{#1}}}
\newcommand{\indexgrammar}[1]{\indextext{#1}\indexgram{#1|idxbfpage}}
% This command uses the "cooked" \indeximpldef command to emit index
% entries; thus they only work for simple index entries that do not contain
% special indexing instructions.
\newcommand{\impldef}[1]{\indeximpldef{#1}implementation-defined}
% \impldefplain passes the argument directly to the index, allowing you to
% use special indexing instructions (!, @, |).
\newcommand{\impldefplain}[1]{\index[impldefindex]{#1}implementation-defined}

% appearance
\newcommand{\idxcode}[1]{#1@\tcode{#1}}
\newcommand{\idxhdr}[1]{#1@\tcode{<#1>}}
\newcommand{\idxgram}[1]{#1@\gterm{#1}}
\newcommand{\idxxname}[1]{__#1@\xname{#1}}

% class member library index
\newcommand{\indexlibrarymember}[2]{\indexlibrary{\idxcode{#1}!\idxcode{#2}}\indexlibrary{\idxcode{#2}!\idxcode{#1}}}

%%--------------------------------------------------
% General code style
\newcommand{\CodeStyle}{\ttfamily}
\newcommand{\CodeStylex}[1]{\texttt{#1}}

% General grammar style
\newcommand{\GrammarStyle}{\itfamily}
\newcommand{\GrammarStylex}[1]{\textit{#1}}

% Code and definitions embedded in text.
\newcommand{\tcode}[1]{\CodeStylex{#1}}
\newcommand{\term}[1]{\textit{#1}}
\newcommand{\gterm}[1]{\GrammarStylex{#1}}
\newcommand{\fakegrammarterm}[1]{\gterm{#1}}
\newcommand{\grammarterm}[1]{\indexgram{\idxgram{#1}}\gterm{#1}}
\newcommand{\grammartermnc}[1]{\indexgram{\idxgram{#1}}\gterm{#1\nocorr}}
\newcommand{\placeholder}[1]{\textit{#1}}
\newcommand{\placeholdernc}[1]{\textit{#1\nocorr}}
\newcommand{\defnxname}[1]{\indextext{\idxxname{#1}}\xname{#1}}
\newcommand{\defnlibxname}[1]{\indexlibrary{\idxxname{#1}}\xname{#1}}

% Non-compound defined term.
\newcommand{\defn}[1]{\defnx{#1}{#1}}
% Defined term with different index entry.
\newcommand{\defnx}[2]{\indexdefn{#2}\textit{#1}}
% Compound defined term with 'see' for primary term.
% Usage: \defnadj{trivial}{class}
\newcommand{\defnadj}[2]{\indextext{#1 #2|see{#2, #1}}\indexdefn{#2!#1}\textit{#1 #2}}

%%--------------------------------------------------
%% allow line break if needed for justification
\newcommand{\brk}{\discretionary{}{}{}}

%%--------------------------------------------------
%% Macros for funky text
\newcommand{\Cpp}{\texorpdfstring{C\kern-0.05em\protect\raisebox{.35ex}{\textsmaller[2]{+\kern-0.05em+}}}{C++}}
\newcommand{\CppIII}{\Cpp{} 2003}
\newcommand{\CppXI}{\Cpp{} 2011}
\newcommand{\CppXIV}{\Cpp{} 2014}
\newcommand{\CppXVII}{\Cpp{} 2017}
\newcommand{\opt}[1]{#1\ensuremath{_\mathit{opt}}}
\newcommand{\dcr}{-{-}}
\newcommand{\bigoh}[1]{\ensuremath{\mathscr{O}(#1)}}

% Make all tildes a little larger to avoid visual similarity with hyphens.
\renewcommand{\~}{\textasciitilde}
\let\OldTextAsciiTilde\textasciitilde
\renewcommand{\textasciitilde}{\protect\raisebox{-0.17ex}{\larger\OldTextAsciiTilde}}
\newcommand{\caret}{\char`\^}

%%--------------------------------------------------
%% States and operators
\newcommand{\state}[2]{\tcode{#1}\ensuremath{_{#2}}}
\newcommand{\bitand}{\ensuremath{\mathbin{\mathsf{bitand}}}}
\newcommand{\bitor}{\ensuremath{\mathbin{\mathsf{bitor}}}}
\newcommand{\xor}{\ensuremath{\mathbin{\mathsf{xor}}}}
\newcommand{\rightshift}{\ensuremath{\mathbin{\mathsf{rshift}}}}
\newcommand{\leftshift}[1]{\ensuremath{\mathbin{\mathsf{lshift}_{#1}}}}

%% Notes and examples
\newcommand{\noteintro}[1]{[\textit{#1}:\space}
\newcommand{\noteoutro}[1]{\textit{\,---\,end #1}\kern.5pt]}
\newenvironment{note}[1][Note]{\noteintro{#1}}{\noteoutro{note}\space}
\newenvironment{example}[1][Example]{\noteintro{#1}}{\noteoutro{example}\space}

%% Library function descriptions
\newcommand{\Fundescx}[1]{\textit{#1}}
\newcommand{\Fundesc}[1]{\Fundescx{#1:}\space}
\newcommand{\required}{\Fundesc{Required behavior}}
\newcommand{\requires}{\Fundesc{Requires}}
\newcommand{\constraints}{\Fundesc{Constraints}}
\newcommand{\mandates}{\Fundesc{Mandates}}
\newcommand{\expects}{\Fundesc{Expects}}
\newcommand{\effects}{\Fundesc{Effects}}
\newcommand{\ensures}{\Fundesc{Ensures}}
\newcommand{\postconditions}{\ensures}
\newcommand{\returns}{\Fundesc{Returns}}
\newcommand{\throws}{\Fundesc{Throws}}
\newcommand{\default}{\Fundesc{Default behavior}}
\newcommand{\complexity}{\Fundesc{Complexity}}
\newcommand{\remarks}{\Fundesc{Remarks}}
\newcommand{\errors}{\Fundesc{Error conditions}}
\newcommand{\sync}{\Fundesc{Synchronization}}
\newcommand{\implimits}{\Fundesc{Implementation limits}}
\newcommand{\replaceable}{\Fundesc{Replaceable}}
\newcommand{\returntype}{\Fundesc{Return type}}
\newcommand{\cvalue}{\Fundesc{Value}}
\newcommand{\ctype}{\Fundesc{Type}}
\newcommand{\ctypes}{\Fundesc{Types}}
\newcommand{\dtype}{\Fundesc{Default type}}
\newcommand{\ctemplate}{\Fundesc{Class template}}
\newcommand{\templalias}{\Fundesc{Alias template}}

%% Cross reference
\newcommand{\xref}{\textsc{See also:}\space}
\newcommand{\xrefc}[1]{\xref{} ISO C #1}

%% Inline parenthesized reference
\newcommand{\iref}[1]{\nolinebreak[3] (\ref{#1})}

%% Inline non-parenthesized table reference (override memoir's \tref)
\renewcommand{\tref}[1]{\tablerefname \nolinebreak[3] \ref{tab:#1}}

%% NTBS, etc.
\newcommand{\NTS}[1]{\textsc{#1}}
\newcommand{\ntbs}{\NTS{ntbs}}
\newcommand{\ntmbs}{\NTS{ntmbs}}
% The following are currently unused:
% \newcommand{\ntwcs}{\NTS{ntwcs}}
% \newcommand{\ntcxvis}{\NTS{ntc16s}}
% \newcommand{\ntcxxxiis}{\NTS{ntc32s}}

%% Code annotations
\newcommand{\EXPO}[1]{\textit{#1}}
\newcommand{\expos}{\EXPO{exposition only}}
\newcommand{\impdef}{\EXPO{implementation-defined}}
\newcommand{\impdefnc}{\EXPO{implementation-defined\nocorr}}
\newcommand{\impdefx}[1]{\indeximpldef{#1}\EXPO{implementation-defined}}
\newcommand{\notdef}{\EXPO{not defined}}

\newcommand{\UNSP}[1]{\textit{\texttt{#1}}}
\newcommand{\UNSPnc}[1]{\textit{\texttt{#1}\nocorr}}
\newcommand{\unspec}{\UNSP{unspecified}}
\newcommand{\unspecnc}{\UNSPnc{unspecified}}
\newcommand{\unspecbool}{\UNSP{unspecified-bool-type}}
\newcommand{\seebelow}{\UNSP{see below}}
\newcommand{\seebelownc}{\UNSPnc{see below}}
\newcommand{\unspecuniqtype}{\UNSP{unspecified unique type}}
\newcommand{\unspecalloctype}{\UNSP{unspecified allocator type}}

%% Manual insertion of italic corrections, for aligning in the presence
%% of the above annotations.
\newlength{\itcorrwidth}
\newlength{\itletterwidth}
\newcommand{\itcorr}[1][]{%
 \settowidth{\itcorrwidth}{\textit{x\/}}%
 \settowidth{\itletterwidth}{\textit{x\nocorr}}%
 \addtolength{\itcorrwidth}{-1\itletterwidth}%
 \makebox[#1\itcorrwidth]{}%
}

%% Double underscore
\newcommand{\ungap}{\kern.5pt}
\newcommand{\unun}{\textunderscore\ungap\textunderscore}
\newcommand{\xname}[1]{\tcode{\unun\ungap#1}}
\newcommand{\mname}[1]{\tcode{\unun\ungap#1\ungap\unun}}

%% An elided code fragment, /* ... */, that is formatted as code.
%% (By default, listings typeset comments as body text.)
%% Produces 9 output characters.
\newcommand{\commentellip}{\tcode{/* ...\ */}}

%% Concepts
\newcommand{\oldconcept}[1]{\textit{Cpp17#1}}
\newcommand{\oldconceptdefn}[1]{\defn{Cpp17#1}}
\newcommand{\idxoldconcept}[1]{Cpp17#1@\textit{Cpp17#1}}
\newcommand{\libconcept}[1]{\tcode{#1}}

%% Ranges
\newcommand{\Range}[4]{\tcode{#1#3,\penalty2000{} #4#2}}
\newcommand{\crange}[2]{\Range{[}{]}{#1}{#2}}
\newcommand{\brange}[2]{\Range{(}{]}{#1}{#2}}
\newcommand{\orange}[2]{\Range{(}{)}{#1}{#2}}
\newcommand{\range}[2]{\Range{[}{)}{#1}{#2}}

%% Change descriptions
\newcommand{\diffdef}[1]{\hfill\break\textbf{#1:}\space}
\newcommand{\diffref}[1]{\pnum\textbf{Affected subclause:} \ref{#1}}
\newcommand{\diffrefs}[2]{\pnum\textbf{Affected subclauses:} \ref{#1}, \ref{#2}}
% \nodiffref swallows a following \change and removes the preceding line break.
\def\nodiffref\change{\pnum\textbf{Change:}\space}
\newcommand{\change}{\diffdef{Change}}
\newcommand{\rationale}{\diffdef{Rationale}}
\newcommand{\effect}{\diffdef{Effect on original feature}}
\newcommand{\difficulty}{\diffdef{Difficulty of converting}}
\newcommand{\howwide}{\diffdef{How widely used}}

%% Miscellaneous
\newcommand{\uniquens}{\placeholdernc{unique}}
\newcommand{\stage}[1]{\item[Stage #1:]}
\newcommand{\doccite}[1]{\textit{#1}}
\newcommand{\cvqual}[1]{\textit{#1}}
\newcommand{\cv}{\ifmmode\mathit{cv}\else\cvqual{cv}\fi}
\newcommand{\numconst}[1]{\textsl{#1}}
\newcommand{\logop}[1]{{\footnotesize #1}}

%%--------------------------------------------------
%% Environments for code listings.

% We use the 'listings' package, with some small customizations.  The
% most interesting customization: all TeX commands are available
% within comments.  Comments are set in italics, keywords and strings
% don't get special treatment.

\lstset{language=C++,
        basicstyle=\small\CodeStyle,
        keywordstyle=,
        stringstyle=,
        xleftmargin=1em,
        showstringspaces=false,
        commentstyle=\itshape\rmfamily,
        columns=fullflexible,
        keepspaces=true,
        texcl=true}

% Our usual abbreviation for 'listings'.  Comments are in
% italics.  Arbitrary TeX commands can be used if they're
% surrounded by @ signs.
\newcommand{\CodeBlockSetup}{%
\lstset{escapechar=@, aboveskip=\parskip, belowskip=0pt,
        midpenalty=500, endpenalty=-50,
        emptylinepenalty=-250, semicolonpenalty=0}%
\renewcommand{\tcode}[1]{\textup{\CodeStylex{##1}}}%
\renewcommand{\term}[1]{\textit{##1}}%
\renewcommand{\grammarterm}[1]{\gterm{##1}}%
}

\lstnewenvironment{codeblock}{\CodeBlockSetup}{}

% An environment for command / program output that is not C++ code.
\lstnewenvironment{outputblock}{\lstset{language=}}{}

% A code block in which single-quotes are digit separators
% rather than character literals.
\lstnewenvironment{codeblockdigitsep}{
 \CodeBlockSetup
 \lstset{deletestring=[b]{'}}
}{}

% Permit use of '@' inside codeblock blocks (don't ask)
\makeatletter
\newcommand{\atsign}{@}
\makeatother

%%--------------------------------------------------
%% Indented text
\newenvironment{indented}[1][]
{\begin{indenthelper}[#1]\item\relax}
{\end{indenthelper}}

%%--------------------------------------------------
%% Library item descriptions
\lstnewenvironment{itemdecl}
{
 \lstset{escapechar=@,
 xleftmargin=0em,
 midpenalty=500,
 semicolonpenalty=-50,
 endpenalty=3000,
 aboveskip=2ex,
 belowskip=0ex	% leave this alone: it keeps these things out of the
				% footnote area
 }
}
{
}

\newenvironment{itemdescr}
{
 \begin{indented}[beginpenalty=3000, endpenalty=-300]}
{
 \end{indented}
}


%%--------------------------------------------------
%% Bnf environments
\newlength{\BnfIndent}
\setlength{\BnfIndent}{\leftmargini}
\newlength{\BnfInc}
\setlength{\BnfInc}{\BnfIndent}
\newlength{\BnfRest}
\setlength{\BnfRest}{2\BnfIndent}
\newcommand{\BnfNontermshape}{\small\rmfamily\itshape}
\newcommand{\BnfTermshape}{\small\ttfamily\upshape}

\newenvironment{bnfbase}
 {
 \newcommand{\nontermdef}[1]{{\BnfNontermshape##1\itcorr}\indexgrammar{\idxgram{##1}}\textnormal{:}}
 \newcommand{\terminal}[1]{{\BnfTermshape ##1}}
 \newcommand{\descr}[1]{\textnormal{##1}}
 \newcommand{\bnfindent}{\hspace*{\bnfindentfirst}}
 \newcommand{\bnfindentfirst}{\BnfIndent}
 \newcommand{\bnfindentinc}{\BnfInc}
 \newcommand{\bnfindentrest}{\BnfRest}
 \newcommand{\br}{\hfill\\*}
 \widowpenalties 1 10000
 \frenchspacing
 }
 {
 \nonfrenchspacing
 }

\newenvironment{simplebnf}
{
 \begin{bnfbase}
 \BnfNontermshape
 \begin{indented}[before*=\setlength{\rightmargin}{-\leftmargin}]
}
{
 \end{indented}
 \end{bnfbase}
}

\newenvironment{bnf}
{
 \begin{bnfbase}
 \begin{bnflist}
 \BnfNontermshape
 \item\relax
}
{
 \end{bnflist}
 \end{bnfbase}
}

% non-copied versions of bnf environments
\let\ncsimplebnf\simplebnf
\let\endncsimplebnf\endsimplebnf
\let\ncbnf\bnf
\let\endncbnf\endbnf

%%--------------------------------------------------
%% Drawing environment
%
% usage: \begin{drawing}{UNITLENGTH}{WIDTH}{HEIGHT}{CAPTION}
\newenvironment{drawing}[4]
{
\newcommand{\mycaption}{#4}
\begin{figure}[h]
\setlength{\unitlength}{#1}
\begin{center}
\begin{picture}(#2,#3)\thicklines
}
{
\end{picture}
\end{center}
\caption{\mycaption}
\end{figure}
}

%%--------------------------------------------------
%% Environment for imported graphics
% usage: \begin{importgraphic}{CAPTION}{TAG}{FILE}

\newenvironment{importgraphic}[3]
{%
\newcommand{\cptn}{#1}
\newcommand{\lbl}{#2}
\begin{figure}[htp]\centering%
\includegraphics[scale=.35]{#3}
}
{
\caption{\cptn}\label{\lbl}%
\end{figure}}

%% enumeration display overrides
% enumerate with lowercase letters
\newenvironment{enumeratea}
{
 \renewcommand{\labelenumi}{\alph{enumi})}
 \begin{enumerate}
}
{
 \end{enumerate}
}

%%--------------------------------------------------
%% Definitions section for "Terms and definitions"
\newcounter{termnote}
\newcommand{\nocontentsline}[3]{}
\newcommand{\definition}[2]{%
\addxref{#2}%
\setcounter{termnote}{0}%
\let\oldcontentsline\addcontentsline%
\let\addcontentsline\nocontentsline%
\ifcase\value{SectionDepth}
         \let\s=\section
      \or\let\s=\subsection
      \or\let\s=\subsubsection
      \or\let\s=\paragraph
      \or\let\s=\subparagraph
      \fi%
\s[#1]{\hfill[#2]}\vspace{-.3\onelineskip}\label{#2} \textbf{#1}\\*%
\let\addcontentsline\oldcontentsline%
}
\newcommand{\defncontext}[1]{\textlangle#1\textrangle}
\newenvironment{defnote}{\addtocounter{termnote}{1}\noteintro{Note \thetermnote{} to entry}}{\noteoutro{note}\space}
%PS:
\newenvironment{insrt}{\begin{addedblock}}{\end{addedblock}}

\setsecnumdepth{subsection}

\begin{document}
\maketitle
\begin{tabular}[t]{|l|l|}\hline 
Document Number: N4189 &   (update of N3949, N3830, N3677)\\\hline
Date: & 2014-10-08 \\\hline
Project: & Programming Language C++\\\hline 
\end{tabular}

\chapter{History}
\section{Changes from N3949}
\begin{itemize}
\item renamed \tcode{scope_guard} to \tcode{scope_exit} and the factory to \tcode{make_scope_exit}. Reason for make_ is to teach users to save the result in a local variable instead of just have a temporary that gets destroyed immediately. Similarly for unique resources, \tcode{unique_resource}, \tcode{make_unique_resource} and \tcode{make_unique_resource_checked}.
\item renamed editorially \tcode{scope_exit::deleter} to \tcode{scope_exit::exit_function}.
\item changed the factories to use forwarding for the \tcode{deleter}/\tcode{exit_function} but not deduce a reference.
\item get rid of \tcode{invoke}'s parameter and rename it to \tcode{reset()} and provide a \tcode{noexcept} specification for it.
\end{itemize}


\section{Changes from N3830}
\begin{itemize}
\item rename to \tcode{unique_resource_t} and factory to \tcode{unique_resource}, resp. \tcode{unique_resource_checked}
\item provide scope guard functionality through type \tcode{scope_guard_t} and \tcode{scope_guard} factory
\item remove multiple-argument case in favor of simpler interface, lambda can deal with complicated release APIs requiring multiple arguments.
\item make function/functor position the last argument of the factories for lambda-friendliness.

\end{itemize}

\section{Changes from N3677}
\begin{itemize}
\item Replace all 4 proposed classes with a single class covering all use cases, using variadic templates, as determined in the Fall 2013 LEWG meeting.
\item The conscious decision was made to name the factory functions without "make", because they actually do not allocate any resources, like \tcode{std::make_unique} or \tcode{std::make_shared} do
\end{itemize}

\chapter{Introduction}
The Standard Template Library provides RAII classes for managing pointer types, such as \tcode{std::unique_ptr} and \tcode{std::shared_ptr}.  This proposal seeks to add a two generic RAII wrappers classes which tie zero or one resource to a clean-up/completion routine which is bound by scope, ensuring execution at scope exit (as the object is destroyed) unless released early or in the case of a single resource: executed early or returned by moving its value.

\chapter{Acknowledgements}
\begin{itemize}
\item This proposal incorporates what Andrej Alexandrescu described as scope_guard long ago and explained again at C++ Now 2012 (%\url{
%https://onedrive.live.com/view.aspx?resid=F1B8FF18A2AEC5C5!1158&app=WordPdf&wdo=2&authkey=!APo6bfP5sJ8EmH4}
).
\item This proposal would not have been possible without the impressive work of Peter Sommerlad who produced the sample implementation during the Fall 2013 committee meetings in Chicago.  Peter took what Andrew Sandoval produced for N3677 and demonstrated the possibility of using C++14 features to make a single, general purpose RAII wrapper capable of fulfilling all of the needs presented by the original 4 classes (from N3677) with none of the compromises.
\item Gratitude is also owed to members of the LEWG participating in the February 2014 (Issaquah) and Fall 2013 (Chicago) meeting for their support, encouragement, and suggestions that have led to this proposal.
\item Special thanks and recognition goes to OpenSpan, Inc. (http://www.openspan.com) for supporting the production of this proposal, and for sponsoring Andrew L. Sandoval's first proposal (N3677) and the trip to Chicago for the Fall 2013 LEWG meeting. \emph{Note: this version abandons the over-generic version from N3830 and comes back to two classes with one or no resource to be managed.}
\item Thanks also to members of the mailing lists who gave feedback. Especially Zhihao Yuan, and Ville Voutilainen.
\item Special thanks to Daniel Kr\"ugler for his deliberate review of the draft version of this paper (D3949).
\end{itemize}

\chapter{Motivation and Scope}
The quality of C++ code can often be improved through the use of "smart" holder objects.  For example, using \tcode{std::unique_ptr} or \tcode{std::shared_ptr} to manage pointers can prevent common mistakes that lead to memory leaks, as well as the less common leaks that occur when exceptions unwind.  The latter case is especially difficult to diagnose and debug and is a commonly made mistake -- especially on systems where unexpected events (such as access violations) in third party libraries may cause deep unwinding that a developer did not expect.  (One example would be on Microsoft Windows with Structured Exception Handling and libraries like MFC that issue callbacks to user-defined code wrapped in a \tcode{try/catch(...)} block.  The developer is usually unaware that their code is wrapped with an exception handler that depending on compile-time options will quietly unwind their code, masking any exceptions that occur.)

While \tcode{std::unique_ptr} can be tweaked by using a custom deleter type to almost a perfect handler for resources, it is awkward to use for handle types that are not pointers and for the use case of a scope guard. As a smart pointer  \tcode{std::unique_ptr} can be used syntactically like a pointer, but requires the use of \tcode{get()} to pass the underlying pointer value to legacy APIs.

This proposal introduces a new RAII "smart" resource container called \tcode{unique_resource} which can bind a resource to "clean-up" code regardless of type of the argument required by the "clean-up" function.

\section {Without Coercion}
Existing smart pointer types can often be coerced into providing the needed functionality.  For example, \tcode{std::unique_ptr} could be coerced into invoking a function used to close an opaque handle type.  For example, given the following system APIs, \tcode{std::unique_ptr} can be used to ensure the file handle is not leaked on scope exit:

\begin{codeblock}
typedef void *HANDLE;        // System defined opaque handle type
typedef unsigned long DWORD;
#define INVALID_HANDLE_VALUE reinterpret_cast<HANDLE>(-1)	
// Can't help this, that's from the OS

// System defined APIs
void CloseHandle(HANDLE hObject);

HANDLE CreateFile(const char *pszFileName, 
	DWORD dwDesiredAccess, 
	DWORD dwShareMode, 
	DWORD dwCreationDisposition, 
	DWORD dwFlagsAndAttributes, 
	HANDLE hTemplateFile);

bool ReadFile(HANDLE hFile, 
	void *pBuffer, 
	DWORD nNumberOfBytesToRead, 
	DWORD*pNumberOfBytesRead);

// Using std::unique_ptr to ensure file handle is closed on scope-exit:
void CoercedExample()
{
	// Initialize hFile ensure it will be "closed" (regardless of value) on scope-exit
	std::unique_ptr<void, decltype(&CloseHandle)> hFile(
		CreateFile("test.tmp", 
			FILE_ALL_ACCESS, 
			FILE_SHARE_READ, 
			OPEN_EXISTING, 
			FILE_ATTRIBUTE_NORMAL,
			nullptr), 
		CloseHandle);

	// Read some data using the handle
	std::array<char, 1024> arr = { };
	DWORD dwRead = 0;
	ReadFile(hFile.get(),	// Must use std::unique_ptr::get()
		&arr[0], 
		static_cast<DWORD>(arr.size()), 
		&dwRead);
}
\end{codeblock}

While this works, there are a few problems with coercing \tcode{std::unique_ptr} into handling the resource in this manner:
\begin{itemize}
\item The type used by the \tcode{std::unique_ptr} does not match the type of the resource.  \tcode{void} is not a \tcode{HANDLE}.  (Thus the word coercion is used to describe it.)
\item There is no convenient way to check the value returned by \tcode{CreateFile} and assigned to the \tcode{std::unique_ptr<void>} to prevent calling \tcode{CloseHandle} when an invalid handle value is returned.  \tcode{std::unique_ptr} will check for a null pointer, but the \tcode{CreateFile} API may return another pre-defined value to signal an error.
\item Because hFile does not have a cast operator that converts the contained "pointer" to a \tcode{HANDLE}, the \tcode{get()} method must be used when invoking other system APIs needing the underlying \tcode{HANDLE}.
\end{itemize}

Each of these problems is solved by \tcode{unique_resource} as shown in the following example:

\begin{codeblock}
void ScopedResourceExample1()
{
	// Initialize hFile ensure it will be "closed" (regardless of value) on scope-exit
	auto hFile = std::make_unique_resource_checked(
		CreateFile("test.tmp", 
			FILE_ALL_ACCESS, 
			FILE_SHARE_READ, 
			OPEN_EXISTING, 
			FILE_ATTRIBUTE_NORMAL,
			nullptr),           // The resource
		INVALID_HANDLE_VALUE,   // Don't call CloseHandle if it failed!
		CloseHandle);           // Clean-up API, lambda-friendly position

	// Read some data using the handle
	std::array<char, 1024> arr = { };
	DWORD dwRead = 0;
	// cast operator makes it seamless to use with other APIs needing a HANDLE
	ReadFile(hFile,
		&arr[0], 
		static_cast<DWORD>(arr.size()), 
		&dwRead);
}
\end{codeblock}

\subsection{Non-Pointer Handle Types}
While \tcode{std::unique_ptr} can deal with the above pointer handle type, as well as \tcode{<cstdio>}'s \tcode{FILE *}, it is non-intuitive to use with handle's like \tcode{<fcntl.h>}'s and \tcode{<unistd.h>}'s \tcode{int} file handles. See the following code examples on using \tcode{unique_resource} with \tcode{int} and \tcode{FILE *} handle types.

\begin{codeblock}
void demonstrate_unique_resource_with_stdio() {
	const std::string filename = "hello.txt";
	{
		auto file=make_unique_resource(::fopen(filename.c_str(),"w"),&::fclose);
		::fputs("Hello World!\n", file);
		ASSERT(file.get()!= NULL);
	}
	{
		std::ifstream input { filename.c_str() };
		std::string line { };
		getline(input, line);
		ASSERT_EQUAL("Hello World!", line);
		getline(input, line);
		ASSERT(input.eof());
	}
	::unlink(filename.c_str());
	{
		auto file = make_unique_resource_checked(::fopen("nonexistingfile.txt", "r"), 
		            (FILE*) NULL, &::fclose);
		ASSERT_EQUAL((FILE*)NULL, file.get());
	}

}
\end{codeblock}

\newpage
\begin{codeblock}
void demontrate_unique_resource_with_POSIX_IO() {
	const std::string filename = "./hello1.txt";
	{
		auto file=make_unique_resource(::open(filename.c_str(),
                     O_CREAT|O_RDWR|O_TRUNC,0666), &::close);
		
		::write(file, "Hello World!\n", 12u);
		ASSERT(file.get() != -1);
	}
	{
		std::ifstream input { filename.c_str() };
		std::string line { };
		getline(input, line);
		ASSERT_EQUAL("Hello World!", line);
		getline(input, line);
		ASSERT(input.eof());
	}
	::unlink(filename.c_str());
	{
		auto file = make_unique_resource_checked(::open("nonexistingfile.txt", 
                       O_RDONLY), -1, &::close);
		ASSERT_EQUAL(-1, file.get());
	}

}\end{codeblock}
\newpage
\section{Multiple Parameters}
This feature was abandoned due to feedback by LEWG in Issaquah. A lambda as deleter can have the same effect without complicating \tcode{unique_resource}.

\section{Lambdas, etc.}
It is also possible to use lambdas instead of a function pointer to initialize a \tcode{unique_resource}.  The following is a very simple and otherwise useless example:

\begin{codeblock}
void TalkToTheWorld(std::ostream& out, std::string const farewell="Uff Wiederluege...")
{
	// Always say goodbye before returning,
	// but if given a non-empty farewell message use it...
	auto goodbye = make_scope_exit([&out]() ->void
	{
		out << "Goodbye world..." << std::endl;
	});
	auto altgoodbye = make_scope_exit([&out,farewell]() ->void
	{
		out << farewell << std::endl;
	});


	if(farewell.empty())
	{
		altgoodbye.release();		// Don't use farewell!
	}
	else
	{
		goodbye.release();	// Don't use the alternate
	}
}
\end{codeblock}
\begin{codeblock}

void testTalkToTheWorld(){
	std::ostringstream out;
	TalkToTheWorld(out,"");
	ASSERT_EQUAL("Goodbye world...\n",out.str());
	out.str("");
	TalkToTheWorld(out);
	ASSERT_EQUAL("Uff Wiederluege...\n",out.str());
}
\end{codeblock}
The example also shows that a scope guard can be released early (that is the clean-up function is not called).

\section{Other Functionality}
In addition to the basic features shown above, \tcode{unique_resource} also provides various operators (cast, \tcode{->}, \tcode{()}, \tcode{*}, and accessor methods (\tcode{get}, \tcode{get_deleter}).  The most complicated of these is the \tcode{invoke()} member function which allows the "clean-up" function to be executed early, just as it would be at scope exit.  This function takes a parameter indicating whether or not the function should again be executed at scope exit.  The \tcode{reset(R\&\& resource)} member function that allows the resource value to be reset.

As already shown in the examples, the expected method of construction is to use one of the two generator functions:
\begin{itemize}
\item \tcode{unique_resource(resources,deleter)} - non-checking instance, allows multiple parameters.
\item \tcode{unique_resource_checked(resource, invalid_value,deleter)} - checked instance, allowing a resource which is validated to inhibit the call to the deleter function if invalid.
\end{itemize}

\section{What's not included}
\tcode{unique_resource} does not do reference counting like \tcode{shared_ptr} does.  Though there is very likely a need for a class similar to \tcode{unique_resource} that includes reference counting it is beyond the scope of this proposal.

One other limitation with \tcode{unique_resource} is that while the resources themselves may be \tcode{reset()}, the "deleter" or "clean-up" function/lambda can not be altered, because they are part of the type.  Generally there should be no need to reset the deleter, and especially with lambdas type matching would be difficult or impossible.

\chapter{Impact on the Standard}
This proposal is a pure library extension. Two new headers, \tcode{<scope_guard>} and \tcode{<unique_resource>} are proposed, but it does not require changes to any standard classes or functions. It does not require any changes in the core language, and it has been implemented in standard C++ conforming to C++14. Depending on the timing of the acceptance of this proposal, it might go into library fundamentals TS under the namespace std::experimental or directly in the working paper of the standard, once it is open again for future additions.

\chapter{Design Decisions}
\section{General Principles}
The following general principles are formulated for \tcode{unique_resource}, and are valid for \tcode{scope_exit} correspondingly.
\begin{itemize}
\item Simplicity - Using \tcode{unique_resource} should be nearly as simple as using an unwrapped type.  The generator functions, cast operator, and accessors all enable this.
\item Transparency - It should be obvious from a glance what each instance of a \tcode{unique_resource} object does.  By binding the resource to it's clean-up routine, the declaration of \tcode{unique_resource} makes its intention clear.
\item Resource Conservation and Lifetime Management - Using \tcode{unique_resource} makes it possible to "allocate it and forget about it" in the sense that deallocation is always accounted for after the \tcode{unique_resource} has been initialized.
\item Exception Safety - Exception unwinding is one of the primary reasons that \tcode{unique_resource} is needed.  Nevertheless the goal is to introduce a new container that will not throw during construction of the \tcode{unique_resource} itself. However, there are no intentions to provide safeguards for piecemeal construction of resource and deleter. If either fails, no unique_resource will be created, because the factory function unique_resource will not be called. It is not recommended to use \tcode{unique_resource()} factory with resource construction, functors or lambda capture types where creation, copying or moving might throw.
\item Flexibility - \tcode{unique_resource} is designed to be flexible, allowing the use of lambdas or existing functions for clean-up of resources. 
\end{itemize}

\section{Prior Implementations}
Please see N3677 from the May 2013 mailing (or http://www.andrewlsandoval.com/scope_exit/) for the previously proposed solution and implementation.  Discussion of N3677 in the (Chicago) Fall 2013 LEWG meeting led to the creation of \tcode{unique_resource} with the general agreement that such an implementation would be vastly superior to N3677 and would find favor with the LEWG.  Professor Sommerlad produced the implementation backing this proposal during the days following that discussion.

N3677 has a more complete list of other prior implementations.

N3830 provided an alternative approach to allow an arbitrary number of resources which was abandoned due to LEWG feedback 

The following issues have been discussed by LEWG already:
\begin{itemize}
\item \textit{Should there be a companion class for sharing the resource \tcode{shared_resource} ?  (Peter thinks no. Ville thinks it could be provided later anyway.) } LEWG: NO.
\item \textit{Should \tcode{~scope_exit()} and \tcode{unique_resource::invoke()} guard against deleter functions that throw with \tcode{try{ deleter(); }catch(...){}} (as now) or not?} LEWG: NO, but provide noexcept in detail.
\item \textit{Does \tcode{scope_exit} need to be move-assignable? } LEWG: NO.
\end{itemize}


\section{Open Issues to be Discussed}
\begin{itemize}
\item Should we make the regular constructors private and friend the factory functions only?
\item Should we provide a factory for type-erasing the deleter/exit_function using std::function?
\end{itemize}


\chapter{Technical Specifications}
The following formulation is based on inclusion to the draft of the C++ standard. However, if it is decided to go into the Library Fundamentals TS, the position of the texts and the namespaces will have to be adapted accordingly, i.e., instead of namespace \tcode{std::} we suppose namespace \tcode{std::experimental::}.

\section{Header}
In section [utilities.general] add two extra rows to table 44 
%%TODO clearer specification 
\begin{table}[htdp]
\caption{Table 44 - General utilities library summary}
\begin{center}
\begin{tabular}{|c|c|}
\hline
Subclause & Header\\
\hline
20.nn Scope Guard Support & \tcode{<scope_exit>}\\
\hline
20.nn+1 Unique Resource Wrapper & \tcode{<unique_resource>}\\
\hline
\end{tabular}
\end{center}
\label{utilities}
\end{table}%

\section{Additional sections}
Add a two new sections to chapter 20 introducing the contents of the headers \tcode{<scope_exit>} and \tcode{<unique_resource>}.

%\rSec1[utilities.scope_exit]{Scope Guard}
\section{Scope Guard Support [utilities.scope_exit]}
This subclause contains infrastructure for a generic scope guard.\\
\\
\synopsis{Header \tcode{<scope_exit>} synopsis}

\pnum
The header  \tcode{<scope_exit>} defines the class template \tcode{scope_exit} and the function template \tcode{make_scope_exit()} to create its instances.

\begin{codeblock}
namespace std {
template <typename EF>
struct scope_exit {
	// construction
	explicit
	scope_exit(EF &&f) noexcept
	:exit_function(std::move(f))
	,execute_on_destruction{true}{}
	// move
	scope_exit(scope_exit  &&rhs) noexcept
	:exit_function(std::move(rhs.exit_function))
	,execute_on_destruction{rhs.execute_on_destruction}{
		rhs.release();
	}
	// release
	~scope_exit() noexcept(noexcept(this->exit_function())){
		if (execute_on_destruction)
				this->exit_function();
	}
	void release() noexcept { this->execute_on_destruction=false;}

private:
	scope_exit(scope_exit const &)=delete;
	void operator=(scope_exit const &)=delete;
	scope_exit& operator=(scope_exit &&)=delete;
	EF exit_function;
	bool execute_on_destruction; // exposition only
};
// factory function
template <typename EF>
auto make_scope_exit(EF &&exit_function) noexcept {
	return scope_exit<std::remove_reference_t<EF>>(std::forward<EF>(exit_function));
}

} // namespace std
\end{codeblock}

\pnum
\enternote
\tcode{scope_exit} is meant to be a universal scope guard to call its deleter function on scope exit.
\exitnote

%\rSec2[scope_exit.scope_exit]{Class Template \tcode{scope_exit}}
\subsection {Class Template \tcode{scope_exit} [scope_exit.scope_exit]}

\pnum
\requires \tcode{EF} shall be a MoveConstructible function object type or reference to such, 
the expression \tcode{exit_function()} shall be valid.
%and shall not throw an exception. %% thanks to Daniel %% removed again
Move construction of \tcode{EF} shall not throw an exception.


\begin{itemdecl}
explicit
scope_exit(EF &&exit_function) noexcept;
\end{itemdecl}


\pnum
\effects constructs a scope_exit object that will call \tcode{exit_function()} on its destruction if not \tcode{release()}ed prior to that. \tcode{execute_on_destruction} is set to \tcode{true}.

\begin{itemdecl}
~scope_exit();
\end{itemdecl}

\pnum
\effects If and only if \tcode{execute_on_destruction} is \tcode{true}, calls \tcode{exit_function()}.


\begin{itemdecl}
void release() noexcept;
\end{itemdecl}

\pnum
\effects \tcode{execute_on_destruction=false;}

\begin{itemdecl}
scope_exit(scope_exit  &&rhs) noexcept;
\end{itemdecl}

\pnum
\effects Move constructs \tcode{exit_function} from \tcode{rhs.exit_function}. \tcode{execute_on_destruction=rhs.execute_on_destruction;}\tcode{rhs.release();}

%\rSec2[scope_exit.make_scope_exit]{Factory Function \tcode{make_scope_exit}}
\subsection {Factory Function \tcode{make_scope_exit} [scope_exit.make_scope_exit]}

\begin{itemdecl}
template <typename EF>
scope_exit<remove_reference_t<EF>> make_scope_exit(EF && exit_function) noexcept;
\end{itemdecl}

\pnum
\returns \tcode{scope_exit<std::remove_reference_t<EF>>(std::forward<EF>(exit_function))}



%%%--------- unique_resource

%\rSec1[utilities.unique_resource]{Unique Resource Wrapper}
\section{Unique Resource Wrapper [utilities.unique_resource]}
This subclause contains infrastructure for a generic RAII resource wrapper.


\synopsis{Header \tcode{<unique_resource>} synopsis}

\pnum
The header  \tcode{<unique_resource>} defines the class template \tcode{unique_resource}, the enumeration \tcode{invoke_it} and function templates \tcode{make_unique_resource()} and \tcode{make_unique_resource_checked()} to create its instances.

\begin{codeblock}
namespace std {

template<typename R,typename D>
class unique_resource {
	R resource; // exposition only
	D deleter; // exposition only
	bool execute_on_destruction; // exposition only
	unique_resource& operator=(unique_resource const &)=delete;
	unique_resource(unique_resource const &)=delete; 
public:
	// construction
	explicit
	unique_resource(R && resource, D && deleter, bool shouldRun=true) noexcept;
	// move
	unique_resource(unique_resource &&other) noexcept;
	unique_resource& operator=(unique_resource  &&other) noexcept ;
	
    	// resource release
	~unique_resource() noexcept(noexcept(this->reset()));
	void reset() noexcept(noexcept(this->get_deleter()(resource)));
	void reset(R && newresource) noexcept(noexcept(this->reset())) ;
	R const & release() noexcept;
	// resource access
	R const & get() const noexcept ;
	operator  R const &() const noexcept ;
	R operator->() const noexcept ;
 	@\seebelow@ operator*() const;
	// deleter access
	const D &	get_deleter() const noexcept;
};

//factories
template<typename R,typename D>
unique_resource<R,remove_reference_t<D>>
make_unique_resource( R && r,D &&d) noexcept;
template<typename R,typename D>
unique_resource<R,D>
make_unique_resource_checked(R r, R invalid, D d) noexcept;

} // namespace std
\end{codeblock}

\pnum
\enternote
\tcode{unique_resource} is meant to be a universal RAII wrapper for resource handles provided by an operating system or platform.
Typically, such resource handles come with a factory function and a deleter function and are of trivial type.
The deleter function together with the result of the factory function is used to create a \tcode{unique_resource} variable, that on destruction will call the release function. Access to the underlying resource handle is achieved through a set of convenience functions or type conversion.
\exitnote

%\rSec2[unique_resource.unique_resource]{Class Template \tcode{unique_resource}}
\subsection {Class Template \tcode{unique_resource} [unique_resource.unique_resource]}

\pnum
\requires  \tcode{D} and \tcode{R} shall be a MoveConstructible and MoveAssignable.
\tcode{D} shall be a function object type or reference to such. 
The expression \tcode{deleter(resource)} shall be valid.
%and shall not throw an exception.  %% thanks to Daniel
Move construction and move assignment of \tcode{D} and \tcode{R} shall not throw an exception.


\begin{itemdecl}
explicit
unique_resource(R && resource, D && deleter, bool shouldRun=true) noexcept;
\end{itemdecl}

\pnum
\effects constructs a \tcode{unique_resource} by moving \tcode{resource} and then \tcode{deleter}. The constructed object will call \tcode{deleter(resource)} on its destruction if not \tcode{release()}ed prior to that. \tcode{execute_on_destruction} is set to \tcode{true}.

\begin{itemdecl}
unique_resource(unique_resource &&other) noexcept;
\end{itemdecl}

\pnum
\effects move-constructs a \tcode{unique_resource} from \tcode{other}'s members then calls\tcode{other.release()}.

\begin{itemdecl}
unique_resource& operator=(unique_resource  &&other) noexcept ;
\end{itemdecl}

\pnum
\effects \tcode{this->reset();} Move-assigns members from \tcode{other} then calls \tcode{other.release()}.

\begin{itemdecl}
~unique_resource() ;
\end{itemdecl}

\pnum
\effects \tcode{this->reset();}

\begin{itemdecl}
void reset() noexcept(noexcept(this->get_deleter()(resource)));
\end{itemdecl}

\pnum
\effects 
\begin{codeblock}
if (execute_on_destruction)  {
    this->execute_on_destruction = false;
	this->get_deleter()(resource);
}
\end{codeblock}

\begin{itemdecl}
void reset(R && newresource) noexcept ;
\end{itemdecl}

\pnum
\effects 
\begin{codeblock}
this->reset(); 
this->resource=std::move(newresource);
this->execute_on_destruction = true;
\end{codeblock}

\pnum
\enternote This function takes the role of an assignment of a new resource.
\exitnote

\begin{itemdecl}
R const & release() noexcept;
\end{itemdecl}

\pnum
\effects \tcode{execute_on_destruction=false;}

\pnum
\returns \tcode{resource}


\begin{itemdecl}
R const & get() const noexcept ;
operator  R const &() const noexcept ;
R operator->() const noexcept ;
\end{itemdecl}

\pnum
\requires \tcode{operator->} is only available if \\
\tcode{is_pointer<R>::value \&\& }\\\tcode{(is_class<remove_pointer_t<R>>::value || }\tcode{is_union<remove_pointer_t<R>>::value)} is \tcode{true}. 

\pnum
\returns \tcode{resource}.

\begin{itemdecl}
@\seebelow@  operator*() const noexcept;
\end{itemdecl}

\pnum
\requires This function is only available if \tcode{is_pointer<R>::value} is \tcode{true}. 

\pnum
\returns \tcode{*this->get()}. 
\\Return type is \tcode{std::add_lvalue_reference_t<std::remove_pointer_t<R>>}


\begin{itemdecl}
const DELETER & get_deleter() const noexcept;
\end{itemdecl}

\pnum
\returns \tcode{deleter}

%\rSec2[unique_resource.unique_resource]{Factories \tcode{unique_resource}}
\subsection {Factories for \tcode{unique_resource} [unique_resource.unique_resource]}
\begin{itemdecl}
template<typename R,typename D>
unique_resource<R,remove_reference_t<D>>
make_unique_resource( R && r,D &&d) noexcept;
\end{itemdecl}

\pnum
\returns \tcode{unique_resource<R,remove_reference_t<D>>(std::move(r),}\\
\tcode{			std::forward<remove_reference_t<D>>(d),true)}


\begin{itemdecl}
template<typename R,typename D>
unique_resource<R,D>
make_unique_resource_checked(R r, R invalid, D d ) noexcept;
\end{itemdecl}

\pnum
\requires \tcode{R} is EqualityComparable

\pnum
\returns \tcode{unique_resource<R,D>(std::move(r), std::move(d), not bool(r==invalid))}



\chapter{Appendix: Example Implementations}
\section{Scope Guard Helper}
\begin{codeblock}
#ifndef SCOPE_EXIT_H_
#define SCOPE_EXIT_H_

// modeled slightly after Andrescu's talk and article(s)

namespace std{
namespace experimental{

template <typename EF>
struct scope_exit {
	// construction
	explicit
	scope_exit(EF &&f) noexcept
	:exit_function(std::move(f))
	,execute_on_destruction{true}{}
	// move
	scope_exit(scope_exit  &&rhs) noexcept
	:exit_function(std::move(rhs.exit_function))
	,execute_on_destruction{rhs.execute_on_destruction}{
		rhs.release();
	}
	// release
	~scope_exit() noexcept(noexcept(this->exit_function())){
		if (execute_on_destruction)
				this->exit_function();
	}
	void release() noexcept { this->execute_on_destruction=false;}

private:
	scope_exit(scope_exit const &)=delete;
	void operator=(scope_exit const &)=delete;
	scope_exit& operator=(scope_exit &&)=delete;
	EF exit_function;
	bool execute_on_destruction; // exposition only
};

template <typename EF>
auto make_scope_exit(EF &&exit_function) noexcept {
	return scope_exit<std::remove_reference_t<EF>>(std::forward<EF>(exit_function));
}

}
}

#endif /* SCOPE_EXIT_H_ */
\end{codeblock}


\section{Unique Resource}
\begin{codeblock}
#ifndef UNIQUE_RESOURCE_H_
#define UNIQUE_RESOURCE_H_

namespace std{
namespace experimental{

template<typename R,typename D>
class unique_resource{
	R resource;
	D deleter;
	bool execute_on_destruction; // exposition only
	unique_resource& operator=(unique_resource const &)=delete;
	unique_resource(unique_resource const &)=delete; // no copies!
public:
	// construction
	explicit
	unique_resource(R && resource, D && deleter, bool shouldrun=true) noexcept
		:  resource(std::move(resource))
		,  deleter(std::move(deleter))
		, execute_on_destruction{shouldrun}{}
	// move
	unique_resource(unique_resource &&other) noexcept
	:resource(std::move(other.resource))
	,deleter(std::move(other.deleter))
	,execute_on_destruction{other.execute_on_destruction}{
		other.release();
	}
	unique_resource& 
	operator=(unique_resource  &&other) noexcept(noexcept(this->reset())) {
		this->reset();
		this->deleter=std::move(other.deleter);
		this->resource=std::move(other.resource);
		this->execute_on_destruction=other.execute_on_destruction;
		other.release();
		return *this;
	}
    // resource release
	~unique_resource() noexcept(noexcept(this->reset())){
		this->reset();
	}
	void reset() noexcept(noexcept(this->get_deleter()(resource))) {
		if (execute_on_destruction) {
			this->execute_on_destruction = false;
			this->get_deleter()(resource);
		}
	}
	void reset(R && newresource) noexcept(noexcept(this->reset())) {
		this->reset();
		this->resource = std::move(newresource);
		this->execute_on_destruction = true;
	}
	R const & release() noexcept{
		this->execute_on_destruction = false;
		return this->get();
	}

	// resource access
	R const & get() const noexcept {
		return this->resource;
	}
	operator  R const &() const noexcept {
		return this->resource;
	}
	R
	operator->() const noexcept {
		return this->resource;
	}
	std::add_lvalue_reference_t<
		std::remove_pointer_t<R>>
	operator*() const {
		return * this->resource;
	}

	// deleter access
	const D &
	get_deleter() const noexcept {
		return this->deleter;
	}
};

//factories
template<typename R,typename D>
auto
make_unique_resource( R && r,D &&d) noexcept {
	return unique_resource<R,std::remove_reference_t<D>>(
			std::move(r)
			,std::forward<std::remove_reference_t<D>>(d)
			,true);
}

template<typename R,typename D>
auto
make_unique_resource_checked(R r, R invalid, D d ) noexcept {
	bool shouldrun = not bool(r == invalid);
	return unique_resource<R,D>(std::move(r), std::move(d), shouldrun);
}



}}
#endif /* UNIQUE_RESOURCE_H_ */
\end{codeblock}

\section{Test cases}
\begin{codeblock}
#include "cute.h"
#include "ide_listener.h"
#include "xml_listener.h"
#include "cute_runner.h"
#include "scoped_resource.h"

#include <iostream>
#include <unistd.h>
#include <memory>

void thisIsATest() {
	using namespace std;
	string const msg { " deleted resource\n" };
	auto res = make_scoped_resource([msg] {cout << msg<< flush;});
	auto file = make_scoped_resource(
			[msg](int i) {write(i,msg.data(),msg.size());}, 2);
	auto duo = make_scoped_resource(
			[msg](int j, double d) {cout << j <<", "<< d << msg<< flush;}, 42,
			3.14);
	cout << "handle: " << file.get() << "\n";
	cout << "handle again casted:" << static_cast<int>(file) << endl;

	ASSERTM("start writing tests", true);
}
void test_scoped_function_semantics() {
	std::ostringstream out { };
	{
		auto cleanup = make_scoped_resource([&out] {out << "cleaned";});
	}
	ASSERT_EQUAL("cleaned", out.str());
}
void test_scoped_function_semantics_invoke_false_release() {
	std::ostringstream out { };
	{
		auto cleanup = make_scoped_resource([&out] {out << "cleaned";});
		cleanup.invoke(invoke_it::again);
		cleanup.invoke(invoke_it::again);
		cleanup.release();
	}
	ASSERT_EQUAL("cleanedcleaned", out.str());
}
void test_scoped_function_semantics_invoke_twice_only_does_once() {
	std::ostringstream out { };
	{
		auto cleanup = make_scoped_resource([&out] {out << "cleaned";});
		cleanup.invoke();
		cleanup.invoke();
	}
	ASSERT_EQUAL("cleaned", out.str());
}
void test_scoped_function_semantics_release_does_nothing() {
	std::ostringstream out { };
	{
		auto cleanup = make_scoped_resource([&out] {out << "cleaned";});
		cleanup.release();
	}
	ASSERT_EQUAL("", out.str());
}
void test_scoped_resource_semantics() {
	std::ostringstream out { };
	{
		auto cleanup = make_scoped_resource(
				[&out](int i) {out << "cleaned " << i;}, 1);
	}
	ASSERT_EQUAL("cleaned 1", out.str());
}
void test_scoped_resource_semantics_reset() {
	std::ostringstream out { };
	{
		auto cleanup = make_scoped_resource(
				[&out](int i,int j) {out << "cleaned " << i;}, 1, 2);
		cleanup.reset(2, 1);
	}
	ASSERT_EQUAL("cleaned 1cleaned 2", out.str());
}
void test_scoped_resource_semantics_reset_move() {
	std::ostringstream out { };
	{
		auto cleanup = make_scoped_resource(
				[&out](auto const &) {out << "cleaned ";},
				std::make_unique<int>(42));
		cleanup.reset(nullptr);
	}
	ASSERT_EQUAL("cleaned cleaned ", out.str());
}
void test_scoped_resource_semantics_release() {
	std::ostringstream out { };
	{
		auto cleanup = make_scoped_resource(
				[&out](int i) {out << "cleaned " << i;}, 5);
		ASSERT_EQUAL(5, cleanup.release());
	}
	ASSERT_EQUAL("", out.str());
}

void test_scope_resource_with_two_values() {
	std::ostringstream out { };
	{
		auto cleanup = make_scoped_resource([&out](int i, double d) {
			out << "cleaned " << i << ", "<<d;
		}, 42, 3.14);
		ASSERT_EQUAL(3.14, cleanup.get<1>());
		cleanup.reset(1, cleanup.get<1>());
		ASSERT_EQUAL(1, cleanup.get());
	}
	ASSERT_EQUAL("cleaned 42, 3.14cleaned 1, 3.14", out.str());

}
void test_scoped_resource_with_pointer() {
	std::ostringstream out { };
	{
		using namespace std::literals;
		auto cleanup = make_scoped_resource(
				[&out](char const *s) {out << "cleaned " << s;},
				static_cast<char const * const >("hello"));
		ASSERT_EQUAL('h', *cleanup);
	}
	ASSERT_EQUAL("cleaned hello", out.str());
}

void test_scoped_resource_address_of() {
	std::ostringstream out { };
	{
		int j { 4 };
		auto cleanup = make_scoped_resource(
				[&out](int &i) {out << "cleaned " << i;}, j);
		ASSERT_EQUAL(j, *&cleanup);
	}
	ASSERT_EQUAL("cleaned 4", out.str());
}

void test_scoped_resource_with_failure_value() {
	std::ostringstream out { };
	{
		auto cleanup = make_scoped_resource_checked(
				[&out](int i) {out << "cleaned " << i;}, -1, -1);
		ASSERT_EQUAL(-1, cleanup.release());
	}
	ASSERT_EQUAL("", out.str());
}
void test_scoped_resource_move_enable() {
	std::ostringstream out { };
	{
		auto cleanup = make_scoped_resource(
				[&out](int i) {out << "cleaned " << i;}, -1);
		auto cleanup2 = std::move(cleanup);
		cleanup2.release();
		cleanup2.reset(42);
	}
	ASSERT_EQUAL("cleaned 42", out.str());
}

auto pass_scoped_resource(std::ostream &out) {
	auto cleanup = make_scoped_resource(
			[&out](std::unique_ptr<int>const& i) {out << "cleaned " << *i;},
			std::unique_ptr<int>(new int { 42 }));
	return cleanup;
}

void test_scoped_resource_can_be_moved() {
	std::ostringstream out { };
	{
		auto cleanup = pass_scoped_resource(out);
		ASSERT_EQUAL(42, *cleanup.get());
	}
	ASSERT_EQUAL("cleaned 42", out.str());
}

void thrower() noexcept(false){
	throw 42;
}

void test_scoped_resource_noexcept_deleter(){
//		auto cleanup = make_scoped_resource(thrower);
// will terminate if run....
}

void runAllTests(int argc, char const *argv[]) {
	cute::suite s;
	//TODO add your test here
	s.push_back(CUTE(thisIsATest));
	s.push_back(CUTE(test_scoped_function_semantics));
	s.push_back(CUTE(test_scoped_function_semantics_invoke_false_release));
	s.push_back(CUTE(test_scoped_function_semantics_invoke_twice_only_does_once));
	s.push_back(CUTE(test_scoped_function_semantics_release_does_nothing));
	s.push_back(CUTE(test_scoped_resource_semantics_reset));
	s.push_back(CUTE(test_scoped_resource_semantics));
	s.push_back(CUTE(test_scope_resource_with_two_values));
	s.push_back(CUTE(test_scoped_resource_with_pointer));
	s.push_back(CUTE(test_scoped_resource_semantics_release));
	s.push_back(CUTE(test_scoped_resource_address_of));
	s.push_back(CUTE(test_scoped_resource_with_failure_value));
	s.push_back(CUTE(test_scoped_resource_move_enable));
	s.push_back(CUTE(test_scoped_resource_can_be_moved));
	s.push_back(CUTE(test_scoped_resource_semantics_reset_move));
	s.push_back(CUTE(test_scoped_resource_noexcept_deleter));
	cute::xml_file_opener xmlfile(argc, argv);
	cute::xml_listener<cute::ide_listener<> > lis(xmlfile.out);
	cute::makeRunner(lis, argc, argv)(s, "AllTests");
}

int main(int argc, char const *argv[]) {
	runAllTests(argc, argv);
	return 0;
}

\end{codeblock}

\end{document}

