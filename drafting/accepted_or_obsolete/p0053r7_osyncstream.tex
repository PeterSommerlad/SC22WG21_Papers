\documentclass[ebook,11pt,article]{memoir}
\usepackage{geometry}  % See geometry.pdf to learn the layout options. There are lots.
\geometry{a4paper}  % ... or a4paper or a5paper or ... 
%\geometry{landscape}  % Activate for for rotated page geometry
%%% from std.tex
%\usepackage[american]
%           {babel}        % needed for iso dates
%\usepackage[iso,american]
%           {isodate}      % use iso format for dates
\usepackage[final]
           {listings}     % code listings
%\usepackage{longtable}    % auto-breaking tables
%\usepackage{ltcaption}    % fix captions for long tables
\usepackage{relsize}      % provide relative font size changes
%\usepackage{textcomp}     % provide \text{l,r}angle
\usepackage{underscore}   % remove special status of '_' in ordinary text
%\usepackage{parskip}      % handle non-indented paragraphs "properly"
%\usepackage{array}        % new column definitions for tables
\usepackage[normalem]{ulem}
\usepackage{enumitem}
\usepackage{color}        % define colors for strikeouts and underlines
%\usepackage{amsmath}      % additional math symbols
%\usepackage{mathrsfs}     % mathscr font
\usepackage[final]{microtype}
%\usepackage{multicol}
\usepackage{xspace}
%\usepackage{lmodern}
\usepackage[T1]{fontenc} % makes tilde work! and is better for umlauts etc.
%\usepackage[pdftex, final]{graphicx}
\usepackage[pdftex,
%            pdftitle={C++ International Standard},
%            pdfsubject={C++ International Standard},
%            pdfcreator={Richard Smith},
            bookmarks=true,
            bookmarksnumbered=true,
            pdfpagelabels=true,
            pdfpagemode=UseOutlines,
            pdfstartview=FitH,
            linktocpage=true,
            colorlinks=true,
            linkcolor=blue,
            plainpages=false
           ]{hyperref}
%\usepackage{memhfixc}     % fix interactions between hyperref and memoir
%\usepackage[active,header=false,handles=false,copydocumentclass=false,generate=std-gram.ext,extract-cmdline={gramSec},extract-env={bnftab,simplebnf,bnf,bnfkeywordtab}]{extract} % Grammar extraction
%
\renewcommand\RSsmallest{5.5pt}  % smallest font size for relsize


%%% use space after & * not before
%%% 

%%%% reuse all three from std.tex:
%!TEX root = std.tex
% Definitions and redefinitions of special commands

%%--------------------------------------------------
%% Difference markups
\definecolor{addclr}{rgb}{0,.6,.3} %% 0,.6,.6 was to blue for my taste :-)
\definecolor{remclr}{rgb}{1,0,0}
\definecolor{noteclr}{rgb}{0,0,1}

\renewcommand{\added}[1]{\textcolor{addclr}{\uline{#1}}}
\newcommand{\removed}[1]{\textcolor{remclr}{\sout{#1}}}
\renewcommand{\changed}[2]{\removed{#1}\added{#2}}

\newcommand{\nbc}[1]{[#1]\ }
\newcommand{\addednb}[2]{\added{\nbc{#1}#2}}
\newcommand{\removednb}[2]{\removed{\nbc{#1}#2}}
\newcommand{\changednb}[3]{\removednb{#1}{#2}\added{#3}}
\newcommand{\remitem}[1]{\item\removed{#1}}

\newcommand{\ednote}[1]{\textcolor{noteclr}{[Editor's note: #1] }}
% \newcommand{\ednote}[1]{}

\newenvironment{addedblock}
{
\color{addclr}
}
{
\color{black}
}
\newenvironment{removedblock}
{
\color{remclr}
}
{
\color{black}
}

%%--------------------------------------------------
%% Grammar extraction.
\def\gramSec[#1]#2{}

\makeatletter
\newcommand{\FlushAndPrintGrammar}{%
\immediate\closeout\XTR@out%
\immediate\openout\XTR@out=std-gram-dummy.tmp%
\def\gramSec[##1]##2{\rSec1[##1]{##2}}%
\input{std-gram.ext}%
}
\makeatother

%%--------------------------------------------------
% Escaping for index entries. Replaces ! with "! throughout its argument.
\def\indexescape#1{\doindexescape#1\stopindexescape!\doneindexescape}
\def\doindexescape#1!{#1"!\doindexescape}
\def\stopindexescape#1\doneindexescape{}

%%--------------------------------------------------
%% Cross references.
\newcommand{\addxref}[1]{%
 \glossary[xrefindex]{\indexescape{#1}}{(\ref{\indexescape{#1}})}%
}

%%--------------------------------------------------
%% Sectioning macros.
% Each section has a depth, an automatically generated section
% number, a name, and a short tag.  The depth is an integer in
% the range [0,5].  (If it proves necessary, it wouldn't take much
% programming to raise the limit from 5 to something larger.)

% Set the xref label for a clause to be "Clause n", not just "n".
\makeatletter
\newcommand{\customlabel}[2]{%
\protected@write \@auxout {}{\string \newlabel {#1}{{#2}{\thepage}{#2}{#1}{}} }%
\hypertarget{#1}{}%
}
\makeatother
\newcommand{\clauselabel}[1]{\customlabel{#1}{Clause \thechapter}}
\newcommand{\annexlabel}[1]{\customlabel{#1}{Annex \thechapter}}

% The basic sectioning command.  Example:
%    \Sec1[intro.scope]{Scope}
% defines a first-level section whose name is "Scope" and whose short
% tag is intro.scope.  The square brackets are mandatory.
\def\Sec#1[#2]#3{%
\ifcase#1\let\s=\chapter\let\l=\clauselabel
      \or\let\s=\section\let\l=\label
      \or\let\s=\subsection\let\l=\label
      \or\let\s=\subsubsection\let\l=\label
      \or\let\s=\paragraph\let\l=\label
      \or\let\s=\subparagraph\let\l=\label
      \fi%
\s[#3]{#3\hfill[#2]}\l{#2}\addxref{#2}}

% A convenience feature (mostly for the convenience of the Project
% Editor, to make it easy to move around large blocks of text):
% the \rSec macro is just like the \Sec macro, except that depths
% relative to a global variable, SectionDepthBase.  So, for example,
% if SectionDepthBase is 1,
%   \rSec1[temp.arg.type]{Template type arguments}
% is equivalent to
%   \Sec2[temp.arg.type]{Template type arguments}
\newcounter{SectionDepthBase}
\newcounter{SectionDepth}

\def\rSec#1[#2]#3{%
\setcounter{SectionDepth}{#1}
\addtocounter{SectionDepth}{\value{SectionDepthBase}}
\Sec{\arabic{SectionDepth}}[#2]{#3}}

%%--------------------------------------------------
% Indexing

% locations
\newcommand{\indextext}[1]{\index[generalindex]{#1}}
\newcommand{\indexlibrary}[1]{\index[libraryindex]{#1}}
\newcommand{\indexhdr}[1]{\indextext{\idxhdr{#1}}\index[headerindex]{\idxhdr{#1}}}
\newcommand{\indexgram}[1]{\index[grammarindex]{#1}}

% Collation helper: When building an index key, replace all macro definitions
% in the key argument with a no-op for purposes of collation.
\newcommand{\nocode}[1]{#1}
\newcommand{\idxmname}[1]{\_\_#1\_\_}
\newcommand{\idxCpp}{C++}

% \indeximpldef synthesizes a collation key from the argument; that is, an
% invocation \indeximpldef{arg} emits an index entry `key@arg`, where `key`
% is derived from `arg` by replacing the folowing list of commands with their
% bare content. This allows, say, collating plain text and code.
\newcommand{\indeximpldef}[1]{%
\let\otextup\textup%
\let\textup\nocode%
\let\otcode\tcode%
\let\tcode\nocode%
\let\ogrammarterm\grammarterm%
\let\grammarterm\nocode%
\let\omname\mname%
\let\mname\idxmname%
\let\oCpp\Cpp%
\let\Cpp\idxCpp%
\let\oBreakableUnderscore\BreakableUnderscore%  See the "underscore" package.
\let\BreakableUnderscore\textunderscore%
\edef\x{#1}%
\let\tcode\otcode%
\let\grammarterm\gterm%
\let\mname\omname%
\let\Cpp\oCpp%
\let\BreakableUnderscore\oBreakableUnderscore%
\index[impldefindex]{\x@#1}%
\let\grammarterm\ogrammarterm%
\let\textup\otextup%
}

\newcommand{\indexdefn}[1]{\indextext{#1}}
\newcommand{\idxbfpage}[1]{\textbf{\hyperpage{#1}}}
\newcommand{\indexgrammar}[1]{\indextext{#1}\indexgram{#1|idxbfpage}}
% This command uses the "cooked" \indeximpldef command to emit index
% entries; thus they only work for simple index entries that do not contain
% special indexing instructions.
\newcommand{\impldef}[1]{\indeximpldef{#1}implementation-defined}
% \impldefplain passes the argument directly to the index, allowing you to
% use special indexing instructions (!, @, |).
\newcommand{\impldefplain}[1]{\index[impldefindex]{#1}implementation-defined}

% appearance
\newcommand{\idxcode}[1]{#1@\tcode{#1}}
\newcommand{\idxhdr}[1]{#1@\tcode{<#1>}}
\newcommand{\idxgram}[1]{#1@\gterm{#1}}
\newcommand{\idxxname}[1]{__#1@\xname{#1}}

% class member library index
\newcommand{\indexlibrarymember}[2]{\indexlibrary{\idxcode{#1}!\idxcode{#2}}\indexlibrary{\idxcode{#2}!\idxcode{#1}}}

%%--------------------------------------------------
% General code style
\newcommand{\CodeStyle}{\ttfamily}
\newcommand{\CodeStylex}[1]{\texttt{#1}}

% General grammar style
\newcommand{\GrammarStyle}{\itfamily}
\newcommand{\GrammarStylex}[1]{\textit{#1}}

% Code and definitions embedded in text.
\newcommand{\tcode}[1]{\CodeStylex{#1}}
\newcommand{\term}[1]{\textit{#1}}
\newcommand{\gterm}[1]{\GrammarStylex{#1}}
\newcommand{\fakegrammarterm}[1]{\gterm{#1}}
\newcommand{\grammarterm}[1]{\indexgram{\idxgram{#1}}\gterm{#1}}
\newcommand{\grammartermnc}[1]{\indexgram{\idxgram{#1}}\gterm{#1\nocorr}}
\newcommand{\placeholder}[1]{\textit{#1}}
\newcommand{\placeholdernc}[1]{\textit{#1\nocorr}}
\newcommand{\defnxname}[1]{\indextext{\idxxname{#1}}\xname{#1}}
\newcommand{\defnlibxname}[1]{\indexlibrary{\idxxname{#1}}\xname{#1}}

% Non-compound defined term.
\newcommand{\defn}[1]{\defnx{#1}{#1}}
% Defined term with different index entry.
\newcommand{\defnx}[2]{\indexdefn{#2}\textit{#1}}
% Compound defined term with 'see' for primary term.
% Usage: \defnadj{trivial}{class}
\newcommand{\defnadj}[2]{\indextext{#1 #2|see{#2, #1}}\indexdefn{#2!#1}\textit{#1 #2}}

%%--------------------------------------------------
%% allow line break if needed for justification
\newcommand{\brk}{\discretionary{}{}{}}

%%--------------------------------------------------
%% Macros for funky text
\newcommand{\Cpp}{\texorpdfstring{C\kern-0.05em\protect\raisebox{.35ex}{\textsmaller[2]{+\kern-0.05em+}}}{C++}}
\newcommand{\CppIII}{\Cpp{} 2003}
\newcommand{\CppXI}{\Cpp{} 2011}
\newcommand{\CppXIV}{\Cpp{} 2014}
\newcommand{\CppXVII}{\Cpp{} 2017}
\newcommand{\opt}[1]{#1\ensuremath{_\mathit{opt}}}
\newcommand{\dcr}{-{-}}
\newcommand{\bigoh}[1]{\ensuremath{\mathscr{O}(#1)}}

% Make all tildes a little larger to avoid visual similarity with hyphens.
\renewcommand{\~}{\textasciitilde}
\let\OldTextAsciiTilde\textasciitilde
\renewcommand{\textasciitilde}{\protect\raisebox{-0.17ex}{\larger\OldTextAsciiTilde}}
\newcommand{\caret}{\char`\^}

%%--------------------------------------------------
%% States and operators
\newcommand{\state}[2]{\tcode{#1}\ensuremath{_{#2}}}
\newcommand{\bitand}{\ensuremath{\mathbin{\mathsf{bitand}}}}
\newcommand{\bitor}{\ensuremath{\mathbin{\mathsf{bitor}}}}
\newcommand{\xor}{\ensuremath{\mathbin{\mathsf{xor}}}}
\newcommand{\rightshift}{\ensuremath{\mathbin{\mathsf{rshift}}}}
\newcommand{\leftshift}[1]{\ensuremath{\mathbin{\mathsf{lshift}_{#1}}}}

%% Notes and examples
\newcommand{\noteintro}[1]{[\textit{#1}:\space}
\newcommand{\noteoutro}[1]{\textit{\,---\,end #1}\kern.5pt]}
\newenvironment{note}[1][Note]{\noteintro{#1}}{\noteoutro{note}\space}
\newenvironment{example}[1][Example]{\noteintro{#1}}{\noteoutro{example}\space}

%% Library function descriptions
\newcommand{\Fundescx}[1]{\textit{#1}}
\newcommand{\Fundesc}[1]{\Fundescx{#1:}\space}
\newcommand{\required}{\Fundesc{Required behavior}}
\newcommand{\requires}{\Fundesc{Requires}}
\newcommand{\constraints}{\Fundesc{Constraints}}
\newcommand{\mandates}{\Fundesc{Mandates}}
\newcommand{\expects}{\Fundesc{Expects}}
\newcommand{\effects}{\Fundesc{Effects}}
\newcommand{\ensures}{\Fundesc{Ensures}}
\newcommand{\postconditions}{\ensures}
\newcommand{\returns}{\Fundesc{Returns}}
\newcommand{\throws}{\Fundesc{Throws}}
\newcommand{\default}{\Fundesc{Default behavior}}
\newcommand{\complexity}{\Fundesc{Complexity}}
\newcommand{\remarks}{\Fundesc{Remarks}}
\newcommand{\errors}{\Fundesc{Error conditions}}
\newcommand{\sync}{\Fundesc{Synchronization}}
\newcommand{\implimits}{\Fundesc{Implementation limits}}
\newcommand{\replaceable}{\Fundesc{Replaceable}}
\newcommand{\returntype}{\Fundesc{Return type}}
\newcommand{\cvalue}{\Fundesc{Value}}
\newcommand{\ctype}{\Fundesc{Type}}
\newcommand{\ctypes}{\Fundesc{Types}}
\newcommand{\dtype}{\Fundesc{Default type}}
\newcommand{\ctemplate}{\Fundesc{Class template}}
\newcommand{\templalias}{\Fundesc{Alias template}}

%% Cross reference
\newcommand{\xref}{\textsc{See also:}\space}
\newcommand{\xrefc}[1]{\xref{} ISO C #1}

%% Inline parenthesized reference
\newcommand{\iref}[1]{\nolinebreak[3] (\ref{#1})}

%% Inline non-parenthesized table reference (override memoir's \tref)
\renewcommand{\tref}[1]{\tablerefname \nolinebreak[3] \ref{tab:#1}}

%% NTBS, etc.
\newcommand{\NTS}[1]{\textsc{#1}}
\newcommand{\ntbs}{\NTS{ntbs}}
\newcommand{\ntmbs}{\NTS{ntmbs}}
% The following are currently unused:
% \newcommand{\ntwcs}{\NTS{ntwcs}}
% \newcommand{\ntcxvis}{\NTS{ntc16s}}
% \newcommand{\ntcxxxiis}{\NTS{ntc32s}}

%% Code annotations
\newcommand{\EXPO}[1]{\textit{#1}}
\newcommand{\expos}{\EXPO{exposition only}}
\newcommand{\impdef}{\EXPO{implementation-defined}}
\newcommand{\impdefnc}{\EXPO{implementation-defined\nocorr}}
\newcommand{\impdefx}[1]{\indeximpldef{#1}\EXPO{implementation-defined}}
\newcommand{\notdef}{\EXPO{not defined}}

\newcommand{\UNSP}[1]{\textit{\texttt{#1}}}
\newcommand{\UNSPnc}[1]{\textit{\texttt{#1}\nocorr}}
\newcommand{\unspec}{\UNSP{unspecified}}
\newcommand{\unspecnc}{\UNSPnc{unspecified}}
\newcommand{\unspecbool}{\UNSP{unspecified-bool-type}}
\newcommand{\seebelow}{\UNSP{see below}}
\newcommand{\seebelownc}{\UNSPnc{see below}}
\newcommand{\unspecuniqtype}{\UNSP{unspecified unique type}}
\newcommand{\unspecalloctype}{\UNSP{unspecified allocator type}}

%% Manual insertion of italic corrections, for aligning in the presence
%% of the above annotations.
\newlength{\itcorrwidth}
\newlength{\itletterwidth}
\newcommand{\itcorr}[1][]{%
 \settowidth{\itcorrwidth}{\textit{x\/}}%
 \settowidth{\itletterwidth}{\textit{x\nocorr}}%
 \addtolength{\itcorrwidth}{-1\itletterwidth}%
 \makebox[#1\itcorrwidth]{}%
}

%% Double underscore
\newcommand{\ungap}{\kern.5pt}
\newcommand{\unun}{\textunderscore\ungap\textunderscore}
\newcommand{\xname}[1]{\tcode{\unun\ungap#1}}
\newcommand{\mname}[1]{\tcode{\unun\ungap#1\ungap\unun}}

%% An elided code fragment, /* ... */, that is formatted as code.
%% (By default, listings typeset comments as body text.)
%% Produces 9 output characters.
\newcommand{\commentellip}{\tcode{/* ...\ */}}

%% Concepts
\newcommand{\oldconcept}[1]{\textit{Cpp17#1}}
\newcommand{\oldconceptdefn}[1]{\defn{Cpp17#1}}
\newcommand{\idxoldconcept}[1]{Cpp17#1@\textit{Cpp17#1}}
\newcommand{\libconcept}[1]{\tcode{#1}}

%% Ranges
\newcommand{\Range}[4]{\tcode{#1#3,\penalty2000{} #4#2}}
\newcommand{\crange}[2]{\Range{[}{]}{#1}{#2}}
\newcommand{\brange}[2]{\Range{(}{]}{#1}{#2}}
\newcommand{\orange}[2]{\Range{(}{)}{#1}{#2}}
\newcommand{\range}[2]{\Range{[}{)}{#1}{#2}}

%% Change descriptions
\newcommand{\diffdef}[1]{\hfill\break\textbf{#1:}\space}
\newcommand{\diffref}[1]{\pnum\textbf{Affected subclause:} \ref{#1}}
\newcommand{\diffrefs}[2]{\pnum\textbf{Affected subclauses:} \ref{#1}, \ref{#2}}
% \nodiffref swallows a following \change and removes the preceding line break.
\def\nodiffref\change{\pnum\textbf{Change:}\space}
\newcommand{\change}{\diffdef{Change}}
\newcommand{\rationale}{\diffdef{Rationale}}
\newcommand{\effect}{\diffdef{Effect on original feature}}
\newcommand{\difficulty}{\diffdef{Difficulty of converting}}
\newcommand{\howwide}{\diffdef{How widely used}}

%% Miscellaneous
\newcommand{\uniquens}{\placeholdernc{unique}}
\newcommand{\stage}[1]{\item[Stage #1:]}
\newcommand{\doccite}[1]{\textit{#1}}
\newcommand{\cvqual}[1]{\textit{#1}}
\newcommand{\cv}{\ifmmode\mathit{cv}\else\cvqual{cv}\fi}
\newcommand{\numconst}[1]{\textsl{#1}}
\newcommand{\logop}[1]{{\footnotesize #1}}

%%--------------------------------------------------
%% Environments for code listings.

% We use the 'listings' package, with some small customizations.  The
% most interesting customization: all TeX commands are available
% within comments.  Comments are set in italics, keywords and strings
% don't get special treatment.

\lstset{language=C++,
        basicstyle=\small\CodeStyle,
        keywordstyle=,
        stringstyle=,
        xleftmargin=1em,
        showstringspaces=false,
        commentstyle=\itshape\rmfamily,
        columns=fullflexible,
        keepspaces=true,
        texcl=true}

% Our usual abbreviation for 'listings'.  Comments are in
% italics.  Arbitrary TeX commands can be used if they're
% surrounded by @ signs.
\newcommand{\CodeBlockSetup}{%
\lstset{escapechar=@, aboveskip=\parskip, belowskip=0pt,
        midpenalty=500, endpenalty=-50,
        emptylinepenalty=-250, semicolonpenalty=0}%
\renewcommand{\tcode}[1]{\textup{\CodeStylex{##1}}}%
\renewcommand{\term}[1]{\textit{##1}}%
\renewcommand{\grammarterm}[1]{\gterm{##1}}%
}

\lstnewenvironment{codeblock}{\CodeBlockSetup}{}

% An environment for command / program output that is not C++ code.
\lstnewenvironment{outputblock}{\lstset{language=}}{}

% A code block in which single-quotes are digit separators
% rather than character literals.
\lstnewenvironment{codeblockdigitsep}{
 \CodeBlockSetup
 \lstset{deletestring=[b]{'}}
}{}

% Permit use of '@' inside codeblock blocks (don't ask)
\makeatletter
\newcommand{\atsign}{@}
\makeatother

%%--------------------------------------------------
%% Indented text
\newenvironment{indented}[1][]
{\begin{indenthelper}[#1]\item\relax}
{\end{indenthelper}}

%%--------------------------------------------------
%% Library item descriptions
\lstnewenvironment{itemdecl}
{
 \lstset{escapechar=@,
 xleftmargin=0em,
 midpenalty=500,
 semicolonpenalty=-50,
 endpenalty=3000,
 aboveskip=2ex,
 belowskip=0ex	% leave this alone: it keeps these things out of the
				% footnote area
 }
}
{
}

\newenvironment{itemdescr}
{
 \begin{indented}[beginpenalty=3000, endpenalty=-300]}
{
 \end{indented}
}


%%--------------------------------------------------
%% Bnf environments
\newlength{\BnfIndent}
\setlength{\BnfIndent}{\leftmargini}
\newlength{\BnfInc}
\setlength{\BnfInc}{\BnfIndent}
\newlength{\BnfRest}
\setlength{\BnfRest}{2\BnfIndent}
\newcommand{\BnfNontermshape}{\small\rmfamily\itshape}
\newcommand{\BnfTermshape}{\small\ttfamily\upshape}

\newenvironment{bnfbase}
 {
 \newcommand{\nontermdef}[1]{{\BnfNontermshape##1\itcorr}\indexgrammar{\idxgram{##1}}\textnormal{:}}
 \newcommand{\terminal}[1]{{\BnfTermshape ##1}}
 \newcommand{\descr}[1]{\textnormal{##1}}
 \newcommand{\bnfindent}{\hspace*{\bnfindentfirst}}
 \newcommand{\bnfindentfirst}{\BnfIndent}
 \newcommand{\bnfindentinc}{\BnfInc}
 \newcommand{\bnfindentrest}{\BnfRest}
 \newcommand{\br}{\hfill\\*}
 \widowpenalties 1 10000
 \frenchspacing
 }
 {
 \nonfrenchspacing
 }

\newenvironment{simplebnf}
{
 \begin{bnfbase}
 \BnfNontermshape
 \begin{indented}[before*=\setlength{\rightmargin}{-\leftmargin}]
}
{
 \end{indented}
 \end{bnfbase}
}

\newenvironment{bnf}
{
 \begin{bnfbase}
 \begin{bnflist}
 \BnfNontermshape
 \item\relax
}
{
 \end{bnflist}
 \end{bnfbase}
}

% non-copied versions of bnf environments
\let\ncsimplebnf\simplebnf
\let\endncsimplebnf\endsimplebnf
\let\ncbnf\bnf
\let\endncbnf\endbnf

%%--------------------------------------------------
%% Drawing environment
%
% usage: \begin{drawing}{UNITLENGTH}{WIDTH}{HEIGHT}{CAPTION}
\newenvironment{drawing}[4]
{
\newcommand{\mycaption}{#4}
\begin{figure}[h]
\setlength{\unitlength}{#1}
\begin{center}
\begin{picture}(#2,#3)\thicklines
}
{
\end{picture}
\end{center}
\caption{\mycaption}
\end{figure}
}

%%--------------------------------------------------
%% Environment for imported graphics
% usage: \begin{importgraphic}{CAPTION}{TAG}{FILE}

\newenvironment{importgraphic}[3]
{%
\newcommand{\cptn}{#1}
\newcommand{\lbl}{#2}
\begin{figure}[htp]\centering%
\includegraphics[scale=.35]{#3}
}
{
\caption{\cptn}\label{\lbl}%
\end{figure}}

%% enumeration display overrides
% enumerate with lowercase letters
\newenvironment{enumeratea}
{
 \renewcommand{\labelenumi}{\alph{enumi})}
 \begin{enumerate}
}
{
 \end{enumerate}
}

%%--------------------------------------------------
%% Definitions section for "Terms and definitions"
\newcounter{termnote}
\newcommand{\nocontentsline}[3]{}
\newcommand{\definition}[2]{%
\addxref{#2}%
\setcounter{termnote}{0}%
\let\oldcontentsline\addcontentsline%
\let\addcontentsline\nocontentsline%
\ifcase\value{SectionDepth}
         \let\s=\section
      \or\let\s=\subsection
      \or\let\s=\subsubsection
      \or\let\s=\paragraph
      \or\let\s=\subparagraph
      \fi%
\s[#1]{\hfill[#2]}\vspace{-.3\onelineskip}\label{#2} \textbf{#1}\\*%
\let\addcontentsline\oldcontentsline%
}
\newcommand{\defncontext}[1]{\textlangle#1\textrangle}
\newenvironment{defnote}{\addtocounter{termnote}{1}\noteintro{Note \thetermnote{} to entry}}{\noteoutro{note}\space}
%PS:
\newenvironment{insrt}{\begin{addedblock}}{\end{addedblock}}

%!TEX root = std.tex
%% layout.tex -- set overall page appearance

%%--------------------------------------------------
%%  set page size, type block size, type block position

\setstocksize{11in}{8.5in}
\settrimmedsize{11in}{8.5in}{*}
\setlrmarginsandblock{1in}{1in}{*}
\setulmarginsandblock{1in}{*}{1.618}

%%--------------------------------------------------
%%  set header and footer positions and sizes

\setheadfoot{\onelineskip}{2\onelineskip}
\setheaderspaces{*}{2\onelineskip}{*}

%%--------------------------------------------------
%%  make miscellaneous adjustments, then finish the layout
\setmarginnotes{7pt}{7pt}{0pt}
\checkandfixthelayout

%%--------------------------------------------------
%% Paragraph and bullet numbering

\newcounter{Paras}
\counterwithin{Paras}{chapter}
\counterwithin{Paras}{section}
\counterwithin{Paras}{subsection}
\counterwithin{Paras}{subsubsection}
\counterwithin{Paras}{paragraph}
\counterwithin{Paras}{subparagraph}

\newcounter{Bullets1}[Paras]
\newcounter{Bullets2}[Bullets1]
\newcounter{Bullets3}[Bullets2]
\newcounter{Bullets4}[Bullets3]

\makeatletter
\newcommand{\parabullnum}[2]{%
\stepcounter{#1}%
\noindent\makebox[0pt][l]{\makebox[#2][r]{%
\scriptsize\raisebox{.7ex}%
{%
\ifnum \value{Paras}>0
\ifnum \value{Bullets1}>0 (\fi%
                          \arabic{Paras}%
\ifnum \value{Bullets1}>0 .\arabic{Bullets1}%
\ifnum \value{Bullets2}>0 .\arabic{Bullets2}%
\ifnum \value{Bullets3}>0 .\arabic{Bullets3}%
\fi\fi\fi%
\ifnum \value{Bullets1}>0 )\fi%
\fi%
}%
\hspace{\@totalleftmargin}\quad%
}}}
\makeatother

\def\pnum{\parabullnum{Paras}{0pt}}

% Leave more room for section numbers in TOC
\cftsetindents{section}{1.5em}{3.0em}

%!TEX root = std.tex
%% styles.tex -- set styles for:
%     chapters
%     pages
%     footnotes

%%--------------------------------------------------
%%  create chapter style

\makechapterstyle{cppstd}{%
  \renewcommand{\beforechapskip}{\onelineskip}
  \renewcommand{\afterchapskip}{\onelineskip}
  \renewcommand{\chapternamenum}{}
  \renewcommand{\chapnamefont}{\chaptitlefont}
  \renewcommand{\chapnumfont}{\chaptitlefont}
  \renewcommand{\printchapternum}{\chapnumfont\thechapter\quad}
  \renewcommand{\afterchapternum}{}
}

%%--------------------------------------------------
%%  create page styles

\makepagestyle{cpppage}
\makeevenhead{cpppage}{\copyright\,\textsc{ISO/IEC}}{}{\textbf{\docno}}
\makeoddhead{cpppage}{\copyright\,\textsc{ISO/IEC}}{}{\textbf{\docno}}
\makeevenfoot{cpppage}{\leftmark}{}{\thepage}
\makeoddfoot{cpppage}{\leftmark}{}{\thepage}

\makeatletter
\makepsmarks{cpppage}{%
  \let\@mkboth\markboth
  \def\chaptermark##1{\markboth{##1}{##1}}%
  \def\sectionmark##1{\markboth{%
    \ifnum \c@secnumdepth>\z@
      \textsection\space\thesection
    \fi
    }{\rightmark}}%
  \def\subsectionmark##1{\markboth{%
    \ifnum \c@secnumdepth>\z@
      \textsection\space\thesubsection
    \fi
    }{\rightmark}}%
  \def\subsubsectionmark##1{\markboth{%
    \ifnum \c@secnumdepth>\z@
      \textsection\space\thesubsubsection
    \fi
    }{\rightmark}}%
  \def\paragraphmark##1{\markboth{%
    \ifnum \c@secnumdepth>\z@
      \textsection\space\theparagraph
    \fi
    }{\rightmark}}}
\makeatother

\aliaspagestyle{chapter}{cpppage}

%%--------------------------------------------------
%%  set heading styles for main matter
\newcommand{\beforeskip}{-.7\onelineskip plus -1ex}
\newcommand{\afterskip}{.3\onelineskip minus .2ex}

\setbeforesecskip{\beforeskip}
\setsecindent{0pt}
\setsecheadstyle{\large\bfseries\raggedright}
\setaftersecskip{\afterskip}

\setbeforesubsecskip{\beforeskip}
\setsubsecindent{0pt}
\setsubsecheadstyle{\large\bfseries\raggedright}
\setaftersubsecskip{\afterskip}

\setbeforesubsubsecskip{\beforeskip}
\setsubsubsecindent{0pt}
\setsubsubsecheadstyle{\normalsize\bfseries\raggedright}
\setaftersubsubsecskip{\afterskip}

\setbeforeparaskip{\beforeskip}
\setparaindent{0pt}
\setparaheadstyle{\normalsize\bfseries\raggedright}
\setafterparaskip{\afterskip}

\setbeforesubparaskip{\beforeskip}
\setsubparaindent{0pt}
\setsubparaheadstyle{\normalsize\bfseries\raggedright}
\setaftersubparaskip{\afterskip}

%%--------------------------------------------------
% set heading style for annexes
\newcommand{\Annex}[3]{\chapter[#2]{(#3)\protect\\#2\hfill[#1]}\relax\label{#1}}
\newcommand{\infannex}[2]{\Annex{#1}{#2}{informative}\addxref{#1}}
\newcommand{\normannex}[2]{\Annex{#1}{#2}{normative}\addxref{#1}}

%%--------------------------------------------------
%%  set footnote style
\footmarkstyle{\smaller#1) }

%%--------------------------------------------------
% set style for main text
\setlength{\parindent}{0pt}
\setlength{\parskip}{1ex}

% set style for lists (itemizations, enumerations)
\setlength{\partopsep}{0pt}
\newlist{indenthelper}{itemize}{1}
\setlist[itemize]{parsep=\parskip, partopsep=0pt, itemsep=0pt, topsep=0pt}
\setlist[enumerate]{parsep=\parskip, partopsep=0pt, itemsep=0pt, topsep=0pt}
\setlist[indenthelper]{parsep=\parskip, partopsep=0pt, itemsep=0pt, topsep=0pt, label={}}

%%--------------------------------------------------
%%  set caption style and delimiter
\captionstyle{\centering}
\captiondelim{ --- }
% override longtable's caption delimiter to match
\makeatletter
\def\LT@makecaption#1#2#3{%
  \LT@mcol\LT@cols c{\hbox to\z@{\hss\parbox[t]\LTcapwidth{%
    \sbox\@tempboxa{#1{#2 --- }#3}%
    \ifdim\wd\@tempboxa>\hsize
      #1{#2 --- }#3%
    \else
      \hbox to\hsize{\hfil\box\@tempboxa\hfil}%
    \fi
    \endgraf\vskip\baselineskip}%
  \hss}}}
\makeatother

%%--------------------------------------------------
%% set global styles that get reset by \mainmatter
\newcommand{\setglobalstyles}{
  \counterwithout{footnote}{chapter}
  \counterwithout{table}{chapter}
  \counterwithout{figure}{chapter}
  \renewcommand{\chaptername}{}
  \renewcommand{\appendixname}{Annex }
}

%%--------------------------------------------------
%% change list item markers to number and em-dash

\renewcommand{\labelitemi}{---\parabullnum{Bullets1}{\labelsep}}
\renewcommand{\labelitemii}{---\parabullnum{Bullets2}{\labelsep}}
\renewcommand{\labelitemiii}{---\parabullnum{Bullets3}{\labelsep}}
\renewcommand{\labelitemiv}{---\parabullnum{Bullets4}{\labelsep}}

%%--------------------------------------------------
%% set section numbering limit, toc limit
\maxsecnumdepth{subparagraph}
\setcounter{tocdepth}{1}


\pagestyle{myheadings}

\newcommand{\papernumber}{p0053r7}
\newcommand{\paperdate}{2017-11-07}

\markboth{\papernumber{} \paperdate{}}{\papernumber{} \paperdate{}}

\title{\papernumber{} - C++ Synchronized Buffered Ostream}
\author{Lawrence Crowl, Peter Sommerlad, Nicolai Josuttis, Pablo Halpern}
\date{\paperdate}                % Activate to display a given date or no date
\setsecnumdepth{subsection}

\begin{document}
\maketitle
\begin{center}
\begin{tabular}[t]{|l|l|}\hline 
Document Number:&  \papernumber \\\hline
Date: & \paperdate \\\hline
Project: & Programming Language C++\\\hline 
Audience: & LWG/LEWG\\\hline
\end{tabular}
\end{center}
\chapter{Introduction}
At present, some stream output operations guarantee that they will not produce race conditions, but do not guarantee that the effect will be sensible. Some form of external synchronization is required. Unfortunately, without a standard mechanism for synchronizing, independently developed software will be unable to synchronize.

\href{https://wg21.link/n3535}{N3535 C++ Stream Mutexes} proposed a standard mechanism for finding and sharing a mutex on streams. At the Spring 2013 standards meeting, the Concurrency Study Group requested a change away from a full mutex definition to a definition that also enabled buffering.

\href{https://wg21.link/N3678}{N3678 C++ Stream Guards} proposed a standard mechanism for batching operations on a stream. That batching may be implemented as mutexes, as buffering, or some combination of both. It was the response to the Concurrency Study Group. A draft of that paper was reviewed in the Library Working Group, who found too many open issues on what was reasonably exposed to the 'buffering' part.

\href{https://wg21.link/N3665}{N3665 Uninterleaved String Output Streaming} proposed making streaming of strings of length less than BUFSIZ appear uninterleaved in the output. It was a "minimal" functionality change to the existing standard to address the problem. The full Committee chose not to adopt that minimal solution.

\href{https://wg21.link/N3750}{N3750 C++ Ostream Buffers} proposed an explicit buffering, at the direction of the general consensus in the July 2013 meeting of the Concurrency Study Group. In November 2014 a further updated version \href{https://wg21.link/N4187}{N4187} was discussed in Library Evolution in Urbana and it was consensus to work with a library expert to get the naming and wording better suited to LWG expectations. Nico Josuttis volunteered to help the original authors. More information on the discussion is available at \href{http://wiki.edg.com/twiki/bin/view/Wg21urbana-champaign/N4187}{LEWG wiki} and the corresponding \href{https://issues.isocpp.org/show_bug.cgi?id=20}{LEWG bug tracker entry (20)}. This paper address issues raised with \href{https://wg21.link/N4187}{N4187}. This paper has a change of title to reflect a change in class name, but contains the same fundamental approach.

\section{Solution}
We propose a \tcode{basic_osyncstream}, that buffers output operations for a wrapped stream. The \tcode{basic_osyncstream} will atomically transfer the contents of an internal stream buffer to a \tcode{basic_ostream}'s stream buffer on destruction of the \tcode{basic_osyncstream}.

The transfer on destruction simplifies the code and ensures at least some output in the presence of an exception.

The intent is that the \tcode{basic_osyncstream} is an automatic-duration variable with a relatively small scope which constructs the text to appear uninterleaved. For example,

\begin{codeblock}
{
  osyncstream bout(cout);
  bout << "Hello, " << "World!" << '\n';
}
\end{codeblock}
or with a single output statement:
\begin{codeblock}
  osyncstream{cout} << "The answer is " << 6*7 << endl;
\end{codeblock}


\chapter{Design}
Here we list a set of previous objections to our proposal in the form of questions. We then give our reasons why other potential solutions do not work as well as our proposed solution.
\begin{description}[style=nextline]
\item[Why can I not just use cout? It should be thread-safe.] 
You will not get data races for cout, but that is not true for most other streams. In addition there is no guarantee that output from different threads appears in any sensible order. Coherent output order is the main reason for this proposal.
\item[Why must osyncstream be an ostream? Could a simple proxy wrapper work?] 
To support all existing and user-defined output operators, osyncstream must be an ostream subclass.
\item[Can you make a flush of the osyncstream mean transfer the characters and flush the underlying stream?
] 
No, because the point of the osyncstream is to collect text into an atomic unit. Flushes are often emitted in calls where the body is not visible, and hence unintentionally break up the text. Furthermore, there may be more than one flush within the lifetime of an osyncstream, which would impose a performance loss when an atomic unit of text needs only one flush. The design decision is only to flush the underlying stream if the osyncstream was flushed, and only once per atomic transfer of the character sequence.
\added{ p0053r5 introduced manipulators to allow both ways, but the default remains not to flush immediately.}

\item[Can flush just transfer the characters and not flush the underlying stream?] 
The flush intends an effect on visible to the user. So, we should preserve at least one flush. The flush may not be visible to the code around the osyncstream, and so its programmer cannot do a manual flush, because attempting to flush the underlying stream that is shared among threads will introduce a data race.

\item[Why do you specify \tcode{basic_syncbuf}? LWG and LEWG thought you wouldn't need it.] 
Users use the streambuf interface. Without access to the \tcode{basic_syncbuf} they would not be able call \tcode{emit()} on the underlying streambuf responsible the synchronization. Every stream in the standard is defined as a pair of streambuf/stream, for good reason. Ask Pablo Halpern if you need more to be convinced.
\item[Where will the required lock/mutex be put? Will it be in every \tcode{streambuf} object changing the ABI?] 
That is one of the reasons why this must be put into the standard library. It is possible to implement with a global map from \tcode{streambuf*} to mutexes as the example code does, however, existing standard library implementations might have already a space for the mutexes (not breaking their ABI), because \tcode{cout}/\tcode{cerr} seem to require one and those might be the most important ones to wrap.
\end{description}

The design follows from the principles of the iostream library. If discussed a person knowledgable about iostream's implementation is favorable, because of its many legacy design decisions, that would no longer be taken by modern C++ class designers. 
 
As with all existing stream classes, using a stream object or a streambuf object from multiple threads can result in a data race, this is also true for \tcode{osyncstream}. Its use enables sharing the wrapped streambuf object/output stream across several threads, if all concurrently using threads apply a local osyncstream around it.

We follow typical stream conventions of \tcode{basic_} prefixes and typedefs.

The constructor for \tcode{osyncstream} takes a non-const reference to a \tcode{basic_ostream} obtaining its stream buffer or a \tcode{basic_streambuf*} or can be put around such a stream buffer directly. Calling the \tcode{emit()} member function or destroying the \tcode{osyncstream} may write to the stream buffer obtained at construction.


The wording below permits implementation of \tcode{basic_osyncstream} with either a stream_mutex from \href{https://wg21.link/n3535}{N3535} or with implementations suitable for \href{https://wg21.link/N3665}{N3665}, e.g. with Posix file locks \href{http://pubs.opengroup.org/onlinepubs/009695399/functions/flockfile.html}{[PSL]}

\section{Items to be discussed by LWG}
\begin{itemize}
\item \removed{Naming of the manipulators} (moved to separate paper p0753r0)
\item What should be the delivery vehicle for this feature? I suggest both C++20 and the concurrency TS?
\textbf{should be moved to C++-next working paper in Albuquerque.}
\end{itemize}


\chapter{Wording}

This wording is relative to the current C++ working draft. It could be integrated into a Concurrency TS  accordingly.

Changes will happen in section 30 mainly.

In section 20.5.1.1 Library contents [contents] add an entry to table 16 (cpp.library.headers) for the new header \tcode{<syncstream>}. 

\section{30.1 General [input.output.general]}
Insert a new entry in table 106 Input/output library summary (tab:iostreams.lib.summary) 

\begin{center} 
\begin{tabular}{|lll|}
%\ref{syncstream}             
\textbf{30.x }& Synchronized output stream                & \tcode{<syncstream>} \\ 
\end{tabular}
\end{center}

\section{30.3.1 Header <iosfwd> synopsis [iosfwd.syn]}

Insert the following declarations to the synopsis of \tcode{<iosfwd>} in the namespace std.

\begin{addedblock}
\begin{codeblock}
template <class charT,
          class traits = char_traits<charT>,
          class Allocator = allocator<charT>>
  class basic_syncbuf;
using syncbuf  = basic_syncbuf<char>;
using wsyncbuf = basic_syncbuf<wchar_t>;

template <class charT,
          class traits = char_traits<charT>,
          class Allocator = allocator<charT> >
  class basic_osyncstream;
using osyncstream = basic_osyncstream<char>;
using wosyncstream = basic_osyncstream<wchar_t>; 
\end{codeblock}
\end{addedblock}


\section{30.x Synchronized output stream [syncstream]}
\emph{Insert this new section 30.x in chapter 30 [input.output] }
\subsection{30.x.1 Header <syncstream> synopsis [syncstream.syn]}
\begin{addedblock}
\begin{codeblock}
namespace std {
template <class charT,
          class traits,
          class Allocator>
  class basic_syncbuf;
using syncbuf  = basic_syncbuf<char>;
using wsyncbuf = basic_syncbuf<wchar_t>;

template <class charT,
          class traits,
          class Allocator>
  class basic_osyncstream;
using osyncstream = basic_osyncstream<char>;
using wosyncstream = basic_osyncstream<wchar_t>; 
}
\end{codeblock}
\end{addedblock}

\begin{addedblock}
\pnum
The header \tcode{<syncstream>} provides a mechanism to synchronize execution agents writing to the same stream. 
It defines class templates \tcode{basic_osyncstream} and \tcode{basic_syncbuf}. The latter buffers output and transfer the buffered content into an object of type \tcode{basic_streambuf<charT, traits>} atomically with respect to such transfers by other \tcode{basic_syncbuf<charT, traits, Allocator>} objects referring to the same \tcode{basic_streambuf<charT, traits>} object. The transfer occurs when \tcode{emit()} is called and when the \tcode{basic_syncbuf<charT, traits, Allocator>} object is destroyed.
\end{addedblock}

\newpage
\subsection{30.x.2 Class template basic_syncbuf [syncstream.syncbuf]}
\begin{addedblock}
\begin{codeblock}
template <class charT,
          class traits,
          class Allocator>
class basic_syncbuf
  : public basic_streambuf<charT, traits> {

public:
  using char_type      = charT;
  using int_type       = typename traits::int_type;
  using pos_type       = typename traits::pos_type;
  using off_type       = typename traits::off_type;
  using traits_type    = traits;
  using allocator_type = Allocator;

  using streambuf_type = basic_streambuf<charT, traits>;

  explicit
  basic_syncbuf(streambuf_type* obuf = nullptr)
    : basic_syncbuf(obuf, Allocator()) { }
  basic_syncbuf(streambuf_type*, const Allocator&);
  basic_syncbuf(basic_syncbuf&&);
  ~basic_syncbuf();

  basic_syncbuf& operator=(basic_syncbuf&&);
  void swap(basic_syncbuf&);

  bool emit();
  streambuf_type* get_wrapped()   const noexcept;
  allocator_type  get_allocator() const noexcept;
  void            set_emit_on_sync(bool) noexcept;

protected:
  int sync() override;

private:
  streambuf_type* wrapped;         // exposition only
  bool            emit_on_sync{}; // exposition only
};

template <class charT, class traits, class Allocator>
void swap(basic_syncbuf<charT, traits, Allocator>&,
          basic_syncbuf<charT, traits, Allocator>&);

\end{codeblock}
\end{addedblock}

\newpage
\section{30.x.2.1 \tcode{basic_syncbuf} constructors [syncstream.syncbuf.cons]}
%\rSec3[syncbuf.cons]{\tcode{basic_syncbuf}  constructors}

\indexlibrary{\idxcode{basic_syncbuf}!constructor}%
\begin{addedblock}
\begin{itemdecl}
basic_syncbuf(streambuf_type* obuf, const Allocator& allocator);
\end{itemdecl}

\begin{itemdescr}
\pnum
\effects
Constructs the \tcode{basic_syncbuf} object and sets \tcode{wrapped} to \tcode{obuf} which will be the final destination of associated output.

\pnum %% was missing
\remarks
A copy of \tcode{allocator} is used to allocate memory for internal buffers holding the associated output.

\pnum
\throws
Nothing unless constructing a mutex or allocating memory throws.

\pnum
\postconditions
\tcode{get_wrapped() == obuf \&\& get_allocator() == allocator}.

\end{itemdescr}


\begin{itemdecl}
basic_syncbuf(basic_syncbuf&& other);
\end{itemdecl}

\begin{itemdescr}
\pnum
\effects
Move constructs from \tcode{other} (Table 23 [moveconstructable]).

\pnum
\postconditions
The value returned by \tcode{this->get_wrapped()} is the value returned by \tcode{other.get_wrapped()} prior to calling this constructor. Output stored in \tcode{other} prior to calling this constructor will be stored in \tcode{*this} afterwards. \tcode{other.rdbuf()->pbase() == other.rdbuf()->pptr()} and \tcode{other.get_wrapped() == nullptr}.


\pnum
\begin{remarks}
This constructor disassociates \tcode{other} from its wrapped stream buffer, ensuring destruction of \tcode{other} produces no output. 
\end{remarks}
\end{itemdescr}

\end{addedblock}

\section{30.x.2.2 \tcode{basic_syncbuf} destructor [syncstream.syncbuf.dtor]}
%\rSec3[syncbuf.dtor]{\tcode{basic_syncbuf}  destructors}

\indexlibrary{\idxcode{basic_syncbuf}!destructor}%
\begin{addedblock}
\begin{itemdecl}
~basic_syncbuf();
\end{itemdecl}

\begin{itemdescr}
\pnum
\effects
Calls \tcode{emit()}.

\pnum
\throws
Nothing. 
If an exception is thrown from \tcode{emit()}, that exception is caught and ignored.

\end{itemdescr}

\end{addedblock}

\subsection{30.x.2.3 \tcode{basic_syncbuf} assign and swap [syncstream.syncbuf.assign]}

%\rSec3[syncstream.syncbuf.assign]{\tcode{basic_syncbuf} assign and swap }
%\indexlibrarymember{operator=}{basic_syncbuf}%

\begin{addedblock}
\begin{itemdecl}
basic_syncbuf& operator=(basic_syncbuf&& rhs) noexcept;
\end{itemdecl}

\begin{itemdescr}
\pnum
\effects 
Calls \tcode{emit()} then move assigns from \tcode{rhs}. After the move assignment \tcode{*this} has the observable state it would have had if it had been move constructed from \tcode{rhs} (see [syncstream.syncbuf.ctor]).


\pnum
\returns \tcode{*this}.

\pnum
\postconditions
\tcode{rhs.get_wrapped() == nullptr}. If \tcode{allocator_traits<Allocator>::propagate_on_container_move_assignment::value == true}, then \tcode{this->get_allocator() == rhs.get_allocator();} otherwise the allocator is unchanged.

\pnum
\begin{remarks}
This assignment operator disassociates \tcode{rhs} from its wrapped stream buffer ensuring destruction of \tcode{rhs} produces no output. 
\end{remarks}
\end{itemdescr}

\begin{itemdecl}
void swap(basic_syncbuf& other) noexcept;
\end{itemdecl}
\begin{itemdescr}
\pnum
\requires
\tcode{allocator_traits<Allocator>::propagate_on_container_swap::value ||}\\
\tcode{this->get_allocator() == other.get_allocator()}.

\pnum
\effects
Exchanges the state of \tcode{*this} and \tcode{other}.
\end{itemdescr}

\begin{itemdecl}
template <class charT, class traits, class Allocator>
void swap(basic_syncbuf<charT, traits, Allocator>& a, 
          basic_syncbuf<charT, traits, Allocator>& b) noexcept;
\end{itemdecl}
\begin{itemdescr}

\pnum
\effects
Equivalent to \tcode{a.swap(b)}.
\end{itemdescr}

\end{addedblock}

\subsection{30.x.2.4 \tcode{basic_syncbuf} member functions [syncstream.syncbuf.mfun]}

%\rSec3[syncstream.syncbuf.mfun]{\tcode{basic_syncbuf} member functions }

\begin{addedblock}
\begin{itemdecl}
bool emit();
\end{itemdecl}

\begin{itemdescr}
\pnum
\effects 
Atomically transfers the contents of the internal buffer to the stream buffer \tcode{*wrapped}, so that they appear in the output stream as a contiguous sequence of characters. If and only if a call was made to \tcode{sync()} since the last call of \tcode{emit()},  \tcode{wrapped->pubsync()} is called.
%roger orr: is called afterwards.

\pnum
\returns \tcode{true} if all of the following conditions hold; otherwise false:
\begin{itemize}
\item \tcode{wrapped != nullptr}.
\item All of the characters in the associated output were successfully transferred.
\item The call to \tcode{wrapped->pubsync()} (if any) succeeded.
\end{itemize}

\pnum
\postconditions
On success the associated output is empty.

\pnum
{\Fundesc{Synchronization}}
All \tcode{emit()} calls transferring characters to the same stream buffer object appear to execute in a total order consistent with \emph{happens-before} where each \tcode{emit()} call \emph{synchronizes-with} subsequent \tcode{emit()} calls in that total order.


\pnum
\remarks
May call member functions of \tcode{wrapped} while holding a lock uniquely associated with \tcode{wrapped}.
\end{itemdescr}

\begin{itemdecl}
streambuf_type* get_wrapped() const noexcept;
\end{itemdecl}

\begin{itemdescr}
\pnum
\returns 
\tcode{wrapped}.
\end{itemdescr}

\begin{itemdecl}
allocator_type get_allocator() const noexcept;
\end{itemdecl}

\begin{itemdescr}
\pnum
\returns 
A copy of the allocator set in the constructor or from assignment.
\end{itemdescr}

\begin{itemdecl}
void set_emit_on_sync(bool b) noexcept;
\end{itemdecl}

\begin{itemdescr}
\pnum
\effects 
\tcode{emit_on_sync=b}.
\end{itemdescr}

\end{addedblock}

\subsection{30.x.2.5 \tcode{basic_syncbuf} overridden virtual member functions [syncstream.syncbuf.virtuals]}

%\rSec3[syncstream.syncbuf.virtuals]{\tcode{basic_syncbuf} overridden virtual member functions }

\begin{addedblock}
\begin{itemdecl}
int sync() override;
\end{itemdecl}
\begin{itemdescr}
\pnum
\effects 
Record that the wrapped stream buffer is to be flushed. 
Then, if \tcode{emit_on_sync == true}, calls \tcode{emit()}.
\begin{note}
If \tcode{emit_on_sync == false}, the actual flush is delayed until a call to \tcode{emit()}.
\end{note}

\pnum
\returns
If \tcode{emit()} was called and returned \tcode{false}, returns \tcode{-1}; otherwise \tcode{0}.
\end{itemdescr}

\end{addedblock}

\newpage
%\rSec2[syncstream.osyncstream]{Class template \tcode{basic_osyncstream}}
\section{30.x.3 Class template \tcode{basic_osyncstream} [syncstream.osyncstream] }

%\indexlibrary{\idxcode{basic_osyncstream}}%

\begin{addedblock}
 
\begin{codeblock}
template <class charT,
          class traits,
          class Allocator>
class basic_osyncstream
  : public basic_ostream<charT, traits>
{
public:
  using char_type      = charT;
  using int_type       = typename traits::int_type;
  using pos_type       = typename traits::pos_type;
  using off_type       = typename traits::off_type;
  using traits_type    = traits;
  using allocator_type = Allocator;
  using streambuf_type = basic_streambuf<charT, traits>;
  using syncbuf_type   = basic_syncbuf<charT, traits, Allocator>;

  basic_osyncstream(streambuf_type*, const Allocator&);
  explicit basic_osyncstream(streambuf_type* obuf)
    : basic_osyncstream(obuf, Allocator()) { }
  basic_osyncstream(basic_ostream<charT, traits>& os, const Allocator& allocator)
    : basic_osyncstream(os.rdbuf(), allocator) { }
  explicit basic_osyncstream(basic_ostream<charT, traits>& os)
    : basic_osyncstream(os, Allocator()) { }
  basic_osyncstream(basic_osyncstream&&) noexcept;
  ~basic_osyncstream();
  basic_osyncstream& operator=(basic_osyncstream&&) noexcept;
  void            emit();
  streambuf_type* get_wrapped() const noexcept;
  syncbuf_type*   rdbuf() const noexcept { return &sb ; }
private:
  syncbuf_type sb; // exposition only
};
\end{codeblock}

\pnum
\tcode{Allocator} shall meet the allocator requirements [allocator.requirements].

\pnum
\begin{example}
Use a named variable within a block statement for streaming.
\begin{codeblock}
{
  osyncstream bout(cout);
  bout << "Hello, ";
  bout << "World!";
  bout << endl; // flush is noted
  bout << "and more!\n";
} // characters are transferred and cout is flushed
\end{codeblock}
\end{example}

\pnum
\begin{example}
Use a temporary object for streaming within a single statement. \tcode{cout} is not flushed.
\begin{codeblock}
osyncstream(cout) << "Hello, " << "World!" << '\n';
\end{codeblock}
\end{example}
\end{addedblock}


%\rSec3[osyncstream.cons]{\tcode{basic_osynctream} constructors}
\subsection{30.x.3.1 \tcode{basic_osyncstream} constructors [syncstream.osyncstream.cons]}

\begin{addedblock}

\begin{itemdecl}
basic_osyncstream(streambuf_type* buf, const Allocator& allocator);
\end{itemdecl}

\begin{itemdescr}
\pnum
\effects
Initializes \tcode{sb} from \tcode{buf} and 
\tcode{allocator} and initializes the base class with \tcode{basic_ostream(\&sb)}.

\pnum
\begin{note}
If wrapped stream buffer pointer refers to a user provided stream buffer then its implementation must be aware that its member functions might be called from \tcode{emit()} while a lock is held. 
\end{note}

\pnum
\postconditions
\tcode{get_wrapped() == buf}.
\end{itemdescr}

\begin{itemdecl}
basic_osyncstream(basic_osyncstream&& other) noexcept;
\end{itemdecl}

\begin{itemdescr}
\pnum
\effects 
Move constructs from \tcode{other}. This is accomplished by move constructing the base class, and the contained \tcode{basic_syncbuf} \tcode{sb}. Next \tcode{basic_ostream<charT, traits>::set_rdbuf(\&sb)} is called to install the \tcode{basic_syncbuf} \tcode{sb}.

\pnum
\postconditions
The value returned by \tcode{get_wrapped()} is the value returned by \tcode{os.get_wrapped()} prior to calling this constructor. \tcode{nullptr == other.get_wrapped()}.
\end{itemdescr}

\end{addedblock}

%\rSec3[osyncstream.dtor]{\tcode{basic_osynctream} destructor}
\subsection{30.x.3.2 \tcode{basic_osyncstream} destructor [syncstream.osyncstream.dtor]}

\begin{addedblock}

\begin{itemdecl}
~basic_osyncstream();
\end{itemdecl}

\begin{itemdescr}
\pnum
\effects 
Calls \tcode{emit()}.
If an exception is thrown from \tcode{emit()}, that exception is caught and ignored.
\end{itemdescr}


\end{addedblock}


%\rSec3[osyncstream.assign]{Assignment}
\subsection{30.x.3.3 Assignment [syncstream.osyncstream.assign]}
\begin{addedblock}

\begin{itemdecl}
basic_osyncstream& operator=(basic_osyncstream&& rhs) noexcept;
\end{itemdecl}

\begin{itemdescr}
\pnum
\effects 
First, calls \tcode{emit()}. 
If an exception is thrown from \tcode{emit()}, that exception is caught and ignored.
Move assigns \tcode{sb} from \tcode{rhs.sb}.
\begin{note}
This disassociates \tcode{rhs} from its wrapped stream buffer ensuring destruction of \tcode{rhs} produces no output. 
\end{note}

\pnum
\postconditions
Primarily, \tcode{nullptr == rhs.get_wrapped()}. Also, \tcode{get_wrapped()} returns the value previously returned by \tcode{rhs.get_wrapped()}.
\end{itemdescr}


\end{addedblock}

%\rSec3[osyncstream.members]{Member functions}
\subsection{30.x.3.4 Member functions [syncstream.osyncstream.members]}
\begin{addedblock}

\begin{itemdecl}
void emit();
\end{itemdecl}

\begin{itemdescr}
\pnum
\effects 
Calls \tcode{sb.emit()}. If this call returns \tcode{false}, calls \tcode{setstate(ios::badbit)}.

\begin{example}
A flush on a \tcode{basic_osyncstream} does not flush immediately:
\begin{codeblock}
{
  osyncstream bout(cout);
  bout << "Hello," << '\n'; // no flush
  bout.emit(); // characters transferred; cout not flushed
  bout << "World!" << endl; // flush noted; bout not flushed
  bout.emit(); // characters transferred; cout flushed
  bout << "Greetings." << '\n'; // no flush
} // characters transferred; cout not flushed
\end{codeblock}
\end{example}

\begin{example}
The function emit() can be used to catch exceptions from operations on the underlying stream.
\begin{codeblock}
{
  osyncstream bout(cout);
  bout << "Hello, " << "World!" << '\n';
  try {
    bout.emit();
  } catch ( ... ) {
    // stuff
  }
}
\end{codeblock}
\end{example}
\end{itemdescr}

\begin{itemdecl}
streambuf_type* get_wrapped() const noexcept;
\end{itemdecl}

\begin{itemdescr}
\pnum
\returns 
\tcode{sb.get_wrapped()}.

\begin{example}
Obtaining the wrapped stream buffer with \tcode{get_wrapped()} allows wrapping it again with an \tcode{osyncstream}. For example,
\begin{codeblock}
{
  osyncstream bout1(cout);
  bout1 << "Hello, ";
  {
    osyncstream(bout1.get_wrapped()) << "Goodbye, " << "Planet!" << '\n';
  }
  bout1 << "World!" << '\n';
}
\end{codeblock}
produces the \emph{uninterleaved} output
\begin{codeblock}
Goodbye, Planet!
Hello, World!
\end{codeblock}
\end{example}

\end{itemdescr}


\end{addedblock}

\section {Feature test macro}
For the purposes of SG10, we recommend the feature-testing macro name \tcode{__cpp_lib_syncstream}.

%%%%%%
\section{Implementation}
An example implementation is availabile on \url{https://github.com/PeterSommerlad/SC22WG21_Papers/tree/master/workspace/p0053_basic_osyncstreambuf}

\chapter{Revisions}
Each section lists the revisions in the following version from the version given in the heading.

\section{\href{https://wg21.link/P0053R7}{P0053R7}}
Final finishing touches to wording and layout based on LWG feedback. Added feature test macro

\section{\href{https://wg21.link/P0053R6}{P0053R6}}
Revisions based on feedback from LWG in Toronto. Extracted the manipulators to p0753, so that they can be reconsidered by LEWG. 

\section{\href{https://wg21.link/P0053R5}{P0053R5}}
This paper was discussed by LWG in Toronto. All recommended changes have been incorporated into the next revision.
\section{\href{http://wiki.edg.com/pub/Wg21kona2017/LibraryEvolutionWorkingGroup/p0053r4.html}{D0053R4 - P0053R4}}
This version was only published on the Kona Wiki. The manipulators were extracted into a separate paper to ease forward progress with this paper, even though the wording of the manipulator specification was already reviewed by LWG.
\begin{itemize}
\item Translate text to LaTeX
\item Added manipulator support so that logging frameworks that get an osyncstream passed can rely on flushes happening.
\item Make sure that temporary stream objects can be used safely (it is already in the standard, Pablo!)
\item Ensured section/table numbers match current working draft (as of 06/2017 before the mailing)
\end{itemize}

\section{\href{https://wg21.link/P0053R3}{P0053R3}}
\begin{itemize}
\item Takes input from Pablo Halpern and re-instantiate the stream buffer that performs the synchronization.
\item Split the constructors with a defaulted allocator parameter to one single-argument one being explicit and one non-explicit taking 2 arguments.
\end{itemize}

\section{\href{https://wg21.link/P0053R2}{P0053R2}}
\begin{itemize}
\item Remove the "may construct a mutex" notes.
\item Remove "may destroy mutex" notes.
\item Clarify \tcode{osyncstream} flush behavior in an example.
\item Make minor editorial fixes.
\end{itemize}

\section{\href{https://wg21.link/P0053R1}{P0053R1}}
\begin{itemize}
\item Provide a \tcode{typedef} for the wrapped stream buffer and use it to shorten the specification as suggested by Daniel Kr\"ugler.
\item Provide move construction and move assignment and specify the moved-from state to be detached from the wrapped stream buffer.
\item Rename \tcode{get()} to \tcode{rdbuf_wrapped()} and provide \tcode{noexcept} specification.
\item Changed to explicitly rely on wrapping a stream buffer, instead of an \tcode{ostream} object and adjust explanations accordingly.
\end{itemize}

\section{\href{https://wg21.link/P0053R0}{P0053R0}}
\begin{itemize}
\item Add remark to note that exchanging the stream buffer while the stream is wrapped causes undefined behavior and added a note to warn stream buffer implementers about the lock being held in \tcode{emit()}. Call \tcode{setstate(badbit)} if IO errors occur in \tcode{emit()}.
\item Replace code references to \tcode{basic_streambuf} by the term stream buffer introduced in [stream.buffers].
\item Provide an example implementation.
\item The lock is to be associate to the underlying \tcode{basic_streambuf} instead of the \tcode{basic_stream}.
\item Added an \tcode{Allocator} constructor parameter.
\item Moves destructor example to \tcode{emit()}.
\item Clarifies wording about synchronization and flushing (several times).
\item List the new header in corresponding table.
\item Provide type aliases in \tcode{<iosfwd>}.
\item Removed copy constructor in favor of providing \tcode{get()}.
\item Notify that move construction and assignment is deleted.
\item Moved class \tcode{noteflush_streambuf} into an implementation note.
\item Add a design subsection that states that a header test is a sufficient feature test.
\end{itemize}

\section{\href{https://wg21.link/N4187}{N4187}}
\begin{itemize}
\item Updated introduction with recent history.
\item Rename ostream_buffer to \tcode{osyncstream} to reflect its appearance is more like a stream than like a buffer.
\item Add an example of using \tcode{osyncstream} as a temporary object.
\item Add an example of a \tcode{osyncstream} constructed with another \tcode{osyncstream}.
\item Clarify the behavior of nested \tcode{osyncstream} executions.
\item Clarify the behavior of exceptions occuring with the \tcode{osyncstream} destructor.
\item Clarify the deferral of flush from the \tcode{osyncstream}'s streambuf to the final \tcode{basic_ostream}.
\item Limit the number of references to \tcode{noteflush_stringbuf} in anticipation of the committee removing it from the specification.
\item Rename \tcode{noteflush_stringbuf} to \tcode{noteflush_streambuf} to hide possible implementation details.
\item Change the base class of \tcode{noteflush_streambuf} from \tcode{basic_stringbuf} to \tcode{basic_streambuf}.
\end{itemize}
\section{\href{https://wg21.link/N4069}{N4069}}
\begin{itemize}
\item Added note to sync as suggested by BSI via email.
\end{itemize}

\section{\href{https://wg21.link/N3978}{N3978}}
\begin{itemize}
\item Added a Design section.
\item Clarify the reference capturing behavior of the \tcode{ostream_buffer} constructors.
\item Added \tcode{noexcept} and \tcode{const} as appropriate to members.
\item Added note on throwing wrapped streams.
\item Change the \tcode{noteflush_stringbuf} public member variable \tcode{needsflush} to a public member query function \tcode{flushed}.
\item Removed the public member function \tcode{noteflush_stringbuf::clear}.
\item Minor synopsis formatting changes.
\item Incorporated feedback from SG1 and Dietmar K\"uhl in specific in Rapperswil.
\end{itemize}
\section{\href{https://wg21.link/N3892}{N3892}}
\begin{itemize}
\item Flush the \tcode{ostream} if and only if the \tcode{ostream_buffer} was flushed.
\item Add the \tcode{clear_for_reuse} function.
\item Change the design from inheriting from \tcode{basic_ostream} to using a \tcode{noteflush_stringbuf}, which is a slightly modified \tcode{basic_stringbuf}. The modification is to note the flush rather than act upon it.
\end{itemize}
\section{\href{https://wg21.link/N3750}{N3750}}
\begin{itemize}
\item Change name to basic_ostream_buffer and add the usual typedefs.
\item Change interface to inherit from basic_ostringstream rather than provide access to a member of that type.
\item Add a Revisions section.
\end{itemize}


\end{document}

