\documentclass[ebook,11pt,article]{memoir}
\usepackage{geometry}                % See geometry.pdf to learn the layout options. There are lots.
\geometry{a4paper}                   % ... or a4paper or a5paper or ... 
%\geometry{landscape}                % Activate for for rotated page geometry
%\usepackage[parfill]{parskip}    % Activate to begin paragraphs with an empty line rather than an indent

\usepackage[final]
           {listings}     % code listings
\usepackage{color}        % define colors for strikeouts and underlines
\usepackage{underscore}   % remove special status of '_' in ordinary text
\usepackage{xspace}
\pagestyle{myheadings}
\markboth{N3779 2013-09-01}{N3779 2013-09-24}
%%TODO
% namespace std { inline namespace literals { inline namespace chrono_literals, string_literals, complex_literals
% put b on the side
% complex: if, i, il optional i_l

\title{apply() call a function with arguments from a tuple
}
\author{Peter Sommerlad}
\date{2013-09-27}                                           % Activate to display a given date or no date
\input{macros}
\setsecnumdepth{subsection}

\begin{document}
\maketitle
\begin{tabular}[t]{|l|l|}\hline 
Document Number: &  NXXXX \\\hline
Date: & 2013-09-27 \\\hline
Project: & Programming Language C++\\\hline 
\end{tabular}
\chapter{History}
\section{integer_sequence}
N3658 and its predecessor N3493 introduced \tcode{integer_sequence} facility and provide application of this features, for example \tcode{apply()} that is proposed in this paper.
\section{Observations}
There is a lot of history I am unaware of and several implementations posted on StackOverflow. Also the C++14 CD contains an implementation of \tcode{apply()} as an example of \tcode{std::integer_sequence} in [intseq.general].
\section{Using tuplevar... or operator...}
Mike Spertus made me aware of the proposed language extension to form a parameter pack from a tuple, i.e., by overloading an \tcode{operator...} which might make the provision of \tcode{apply()} moot. However, up to now, no such feature has been proposed to the standard committee and it is unclear if it would make it into C++17. Even if it would, it would just make the implementation of apply trivial.
\chapter{Introduction}
Tuples are great for generic programming with variadic templates. However, the standard does not define a general purpose facility that allows to call a function/functor/lambda with the tuple elements as arguments. Such a feature should be provided, because it is useful (at least for me). It even is given as an example of \tcode{std::integer_sequence} in [intseq.general] coming from N3658. 
\section{Rationale}
It is easy to create tuples from variadic templates either from types directly as \tcode{std::tuple<PACK...>} or  with \tcode{std::make_tuple()} or \tcode{std::forward_as_tuple()} the opposite mechanism of passing a tuple's elements as function arguments is not available.
%\section{Open Issues Discussed}

\section{Acknowledgements}
Acknowledgements go to Jonathan Wakely for providing integer_sequence and providing apply() as the example in the working draft.

Acknowledgements go to Mike Spertus for making me aware of the ... pack formation approaches.
%more

\chapter{Possible Implementation}
The following implementation suggestion was derived from N3658, N3690, and StackOverflow (http://stackoverflow.com/a/12650100) and some simplification. It actually seems to work with current \tcode{clang -std=c++1y}.

\begin{codeblock}
template<typename F, typename Tuple, size_t ... I>
auto apply_impl(F&& f, Tuple&& t, index_sequence<I...>) {
	return forward<F>(f)(get<I>(forward<Tuple>(t))...);
}
template<typename F, typename Tuple>
auto apply(F&& f, Tuple&& t) {
  using Indices = make_index_sequence<tuple_size<decay_t<Tuple>>::value>;
  return apply_impl(forward<F>(f), forward<Tuple>(t), Indices{});
}
\end{codeblock}

\chapter{Proposed Library Additions}

Add the following declaration in [tuple.general] in the synopsis under the group \emph{element access}:

\begin{codeblock}
template <typename F, typename Tuple>
auto apply(F&& f, Tuple&& args);
\end{codeblock}

Append the following to section [tuple.elem] after paragraph 11.

\begin{itemdecl}
template <typename F, typename Tuple>
auto apply(F&& f, Tuple&& args);
\end{itemdecl}
\begin{itemdescr}
\pnum
\requires \tcode{args} is a \tcode{std::tuple}. \tcode{f} is a function object.

\returns The result of calling \tcode{f} with the \tcode{args}' tuple elements as arguments.
\end{itemdescr}



%%%%%%%%%%% STEFANUS %%%%% 

\section{}


%%%%
%%%%%

\end{document}   